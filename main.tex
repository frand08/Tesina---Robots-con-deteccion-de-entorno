%% Para elegir si hacer el informe de imagenes o no
\newif\ifimagenes
%% Este, a su vez, es para ver si es el paper o no
\newif\ifimagenespaper
%% Este porque no se si agregar tanto de SLAM.
\newif\ifslamext
% \imagenestrue % comment out to hide imagenes
% \imagenespapertrue
% \slamexttrue

\ifimagenes
    \ifimagenespaper
        \documentclass[conference]{IEEEtran}
        \IEEEoverridecommandlockouts
    \else
        \documentclass[a4paper,10pt]{article}
    \fi
\else
    \documentclass[a4paper,10pt]{article}
\fi

\usepackage{subfig} % en lugar de subcaption
% Paquetes
% \usepackage{fontspec}
\usepackage{titlesec}
\usepackage{titling}
\usepackage{setspace}
\usepackage[hidelinks]{hyperref}
\usepackage{graphicx}
\usepackage{caption}
\usepackage{float}
\usepackage{xcolor}
\usepackage[affil-it]{authblk}
% \usepackage{biblatex}
% \addbibresource{Mendeley.bib}

% Para tablas
\usepackage{booktabs}
\usepackage{longtable}
\usepackage{rotating}

% Idioma
\usepackage[spanish,es-tabla]{babel}
\usepackage[utf8]{inputenc}
% \setlength{\parskip}{1em}
% Configuraciones
% \setmainfont{Times New Roman}
\hypersetup{
    colorlinks,
    linkcolor={red!50!black},
    citecolor={blue!50!black},
    urlcolor={blue!80!black}
}

\usepackage{mathrsfs}
\usepackage{amsmath,amssymb}
\usepackage{bm}
% \setlength{\parskip}{1em}
\numberwithin{equation}{section}

\usepackage{accents}
\newcommand\undervec[1]{\underaccent{\vec}{#1}}

% Para minipage
\usepackage{sidecap}

\ifimagenes
    \ifimagenespaper
    \else
        % Para que paragraph sea subsubsub
        \setcounter{secnumdepth}{4}
        \titleformat{\paragraph}
        {\normalfont\normalsize\bfseries}{\theparagraph}{1em}{}
        \titlespacing*{\paragraph}
        {0pt}{3.25ex plus 1ex minus .2ex}{1.5ex plus .2ex}
    \fi
\fi

% Para poner codigo
\usepackage{listings}
\usepackage{xcolor}
\definecolor{codegreen}{rgb}{0,0.6,0}
\definecolor{codegray}{rgb}{0.5,0.5,0.5}
\definecolor{codepurple}{rgb}{0.58,0,0.82}
\definecolor{backcolour}{rgb}{0.95,0.95,0.92}
\lstdefinestyle{mystyle}{
    backgroundcolor=\color{backcolour},   
    commentstyle=\color{codegreen},
    keywordstyle=\color{magenta},
    numberstyle=\tiny\color{codegray},
    stringstyle=\color{codepurple},
    basicstyle=\ttfamily\footnotesize,
    breakatwhitespace=false,         
    breaklines=true,                 
    captionpos=b,                    
    keepspaces=true,                 
    numbers=left,                    
    numbersep=5pt,                  
    showspaces=false,                
    showstringspaces=false,
    showtabs=false,                  
    tabsize=2
}
\lstset{style=mystyle}
\renewcommand{\lstlistingname}{Código} % Para que aparezca en español

% Para que no tire overflow hbox porque el texttt no corta
% \usepackage{microtype}

% Para meter PDFs
\usepackage{pdfpages}


% Para comentar codigo
\usepackage{comment}

% Para bibliografia
\usepackage[nottoc]{tocbibind}

% Para los grados ° con \textdegree
\usepackage{textcomp}

\begin{document}

\ifimagenespaper
\else
    \begin{titlepage}
\fi
\ifimagenes
    \ifimagenespaper
        \title{SLAM 3D basado en una cámara RGB-D y PCL}
        
        \author{{Francisco Domínguez}\\
        {\textit{Universidad Tecnológica Nacional}\\
        Buenos Aires, Argentina \\
        fdominguez@est.frba.utn.edu.ar}
        }
        \date{}
        \maketitle
        
        
        \ifimagenes
\ifimagenespaper
\begin{abstract}
\else
\section*{Abstract}
\fi
Para un robot móvil que explora un entorno estático desconocido, localizarse y construir un mapa al mismo tiempo es el problema conocido como localización y mapeo simultáneos (SLAM). Con el fin de poder resolver este problema, en el presente trabajo se desarrolla un enfoque de la odometría visual a partir de imágenes en colores con mapa de profundidad (RGB-D) de una cámara Microsoft Kinect bajo el entorno de simulación Gazebo. Con este fin, se propone un \textit{pipeline de registro} que tiene como objetivo encontrar la mejor estimación de movimiento de cuerpo rígido para mapear una imagen de profundidad en otra, asumiendo una escena estática tomada por una cámara en movimiento. 

El pipeline propuesto se basa en nubes de puntos organizadas, esto es, que dichas nubes se presenten como matrices 2D, tal como la estructura que presentan las cámara monoculares. Aprovechando esto, se emplea una técnica la cual asegura que la nube de puntos resultante tenga un número reducido de muestras extraídas en base a las distintas orientaciones superficiales que presenta la nube, llamada \textit{normal space sampling}, aumentando la probabilidad de que el registro converja al mínimo global. Los resultados obtenidos se asemejan a la trayectoria real simulada por el robot.
\ifimagenespaper
\end{abstract}

\begin{IEEEkeywords}
RGB-D, PCL, SLAM3D, SLAM, Gazebo
\end{IEEEkeywords}
\else
\fi

\else
\section*{Abstract}
\label{sec:1_abstract}

En los últimos años, los robots han recibido una mayor atención de la comunidad de investigación, en especial debido a la aparición de los vehículos aéreos no tripulados.

El presente trabajo describe el desarrollo de una plataforma de control de robots. Dicha plataforma cuenta con la posibilidad de expandir sus funcionalidades gracias al estándar de hardware PC/104 presente en la industria, logrando así poder desarrollar a partir de la misma tanto vehículos aéreos no tripulados (UAV) como robots terrestres, entre otros. En base a la misma, se desarrollará un robot capaz de estimar tanto su posición como el contorno del mapa en el cual se encuentra, problema conocido como el de localización y mapeo simultáneos. 

Como se utilizan en el mismo distintos sensores, en primera instancia se realiza la calibración de gran parte de los mismos, para luego utilizarlos en la generación de dos mapas: uno en dos dimensiones, basado en la IMU y el LIDAR; y el otro tridimensional, basado en la cámara RGB-D.
\fi


        \section{Introducción}
\label{sec:2_marcoteorico}
Si bien el uso de robots tiene sus orígenes principalmente en fábricas para el ensamblaje de automóviles, en los últimos años la electrónica moderna ha permitido introducirlos en otras áreas, ya sea desde electrodomésticos y juguetes de la vida cotidiana, hasta viajes espaciales con el fin de explorar planetas desconocidos. Esta expansión es debido a que los mismos permiten reducir la interacción humana no solo en tareas que presentan un riesgo a la integridad de la persona, sino también en aquellas que tienen cierto grado de repetitividad.

Yendo a un caso más concreto, en el último tiempo muchas personas han adquirido los llamados robots aspiradoras, los cuales consiguen limpiar la superficie de las casas en un tiempo medianamente razonable, aunque este tiempo no suele ser una preocupación ya que al ser el mismo completamente autónomo, uno puede seguir con sus actividades cotidianas. Ahora bien, si se analiza el recorrido que realizan la mayoría de estos robots, se puede apreciar que el mismo es completamente aleatorio, y por ende se tendería a creer que no podrá pasar por toda la superficie y limpiarla completamente. La realidad es que, como están mucho tiempo circulando, logran pasar por todos lados, pero esto genera que circulen más veces por unos lugares que por otros, haciendo ineficiente el trabajo, tal como se observa en la Figura \ref{fig:vaccumrobot}.

\begin{figure}%
    \centering
    \subfloat{{\includegraphics[width=0.47\textwidth]{Img/Roomba}}}%
    \qquad
    \subfloat{{\includegraphics[width=0.47\textwidth]{Img/VaccumRobot.png}}}%
    \caption{Ejemplo del recorrido de un robot aspiradora comercial}
    \label{fig:vaccumrobot}
\end{figure}

A pesar de que se busque reducir la interacción humana bajo estas circunstancias, no todos los robots hoy en día son autónomos, principalmente debido a la falta de robustez del algoritmo de control involucrado en el proceso. Tomando por caso el ejemplo anterior, si bien el robot aspiradora es autónomo, su eficiencia respecto a la forma óptima de realizar la tarea no es una garantía en todos los casos.

Es útil distinguir entre robots que están inmóviles, como un brazo robótico en una fábrica, y robots que son móviles, como un auto sin conductor. En este trabajo se hará hincapié en los robots móviles. Usamos este término para describir un robot impulsado por sus propios medios que puede moverse cinemáticamente entre ubicaciones en su entorno. Cuando se habla de la posición y orientación combinadas del robot, esto se define como su \textit{pose}.

Los robots móviles pueden referirse a robots que se mueven sobre el suelo, bajo el agua, a través del aire y en entornos de microgravedad, tales como los que pueden observarse en la Figura \ref{fig:mobilerobots}. Si bien este trabajo busca aplicar en parte a cualquiera de estos entornos, el enfoque del mismo se refiere principalmente a los robots móviles que permanecen en contacto con el suelo. El término vehículos terrestres no tripulados (con sus siglas en inglés, UGV) a menudo se usa más específicamente para describir robots móviles terrestres, mientras que el término vehículos aéreos no tripulados (del inglés, UAV) refieren principalmente a los drones.

Los robots móviles se mueven a través de entornos grandes y potencialmente dinámicos, lo que hace que la percepción sea mucho más difícil que los robots industriales con entornos de trabajo limitados y parámetros operativos rígidos. Los robots móviles requieren sensores adicionales, una mejor percepción y mayores grados de autonomía para operar en entornos del mundo real que cambian con frecuencia.


\begin{figure}%
    \centering
    \subfloat[Wheeled robot]{{\includegraphics[width=0.25\textwidth]{Img/WheeledRobot.jpeg}}}%
    \qquad
    \subfloat[Legged robot]{{\includegraphics[width=0.25\textwidth]{Img/LeggedRobot.jpeg}}}%
    \qquad
    \subfloat[Tracked robot]{{\includegraphics[width=0.25\textwidth]{Img/TrackedRobot.jpeg}}}%
    \qquad
    \subfloat[Drone]{{\includegraphics[width=0.3\textwidth]{Img/Drone.jpeg}}}%
    \qquad
    \subfloat[Water-based robot]{{\includegraphics[width=0.5\textwidth]{Img/WaterBasedRobot.jpg}}}%
    \caption{Tipos de robot móviles}
    \label{fig:mobilerobots}
\end{figure}

Dentro de lo que respecta a robótica móvil, uno de sus desafíos actuales es lograr la completa autonomía del vehículo. Una gran parte de las empresas de la industria automotriz han dedicado mucho tiempo y dinero para lograr el objetivo. La investigación ha avanzado hasta el punto de que en la actualidad hay autos semi-autónomos disponibles en el mercado, y se cree ampliamente que en el futuro cercano casi todos los vehículos serán completamente autónomos. Para lograr esto, obviamente hay una necesidad de sensores y funciones avanzadas. Uno de los principales problemas es el hecho de que el vehículo debe ser consciente de su posición actual dentro de un entorno desconocido. Esto se conoce como la localización y mapeo simultáneos, comúnmente abreviado como SLAM.

El SLAM es un problema difícil de solucionar, así como lo fue para los humanos en el pasado. Tiempo atrás, los marineros usaban los llamados registros de fichas para estimar su velocidad y extrapolaban esta velocidad con el tiempo para calcular su posición u orientación. Este proceso de navegación por estima (en inglés, \textit{dead reckoning}) conduce inevitablemente a errores graves de posicionamiento con el tiempo y, por lo tanto, siempre se ha respaldado mediante el uso de puntos de referencia (o \textit{landmarks}) para la navegación. Quedándonos en este caso particular, con el uso de las estrellas los marineros lograban saber donde estaba el Norte, permitiendo corregir su dirección. Sin embargo, no siempre se podía observar este patrón por días debido al cielo nublado, entonces tenían que confiar solamente en su navegación por estima, haciendo que la incertidumbre crezca día tras día. Para el caso de los robots, cuando el mismo recorre un  mapa completamente desconocido, acumulará error debido a las predicciones que debe tomar respecto a su pose y entorno en el que está, hasta el momento en que entra a un área con puntos de referencia conocidos, puediendo corregir su estimación de posición. Esto es lo denominado cierre de ciclo (o \textit{loop closure}), como puede verse en la Figura \ref{fig:poseloopclcorr}.

\begin{figure}
    \centering
    \includegraphics[width=\textwidth]{Img/Pose_LoopClosureCorr.png}
    \caption{Estimación de pose y su correción mediante el loop closure}
    \label{fig:poseloopclcorr}
\end{figure}

Volviendo al ejemplo del robot de limpieza, si el mismo conociera el mapa en el que se encuentra, podría entonces trazar la ruta mas óptima para realizar la limpieza del lugar. Como no todos los ambientes son iguales y cada uno tiene, por ejemplo, muebles ubicados en distintos lugares, dicho robot podría hacer primero un reconocimiento del lugar previo a la limpieza para tener una noción de todo el espacio, o bien obtener el mapa y la ubicación en la que se encuentra en base al movimiento que realiza, pasando por los lugares que le faltaron aspirar. Para cualquiera de las dos opciones, por lo tanto, necesitará tanto de tomar los datos del lugar como de controlar al vehículo para que vaya circulando por el ambiente.

La estimación de la pose, entonces, es una parte no separable de aplicaciones como el control del vehículo y el mapeo. Varios sensores se utilizan comúnmente para estimar la pose de un robot. Las unidades de medición inercial (IMU), la cámara, la odometría de las ruedas\footnote{Se mide la rotación de las ruedas para estimar cambios de la posición del vehículo a lo largo del tiempo} para el caso de ciertos robots móviles  y el LiDAR se encuentran entre los sensores más populares para la localización en interiores \cite{delrosario2016}, mientras que para exteriores suele sumarse el GPS a estos sensores, corrigiendo al mismo mediante acción de la IMU \cite{caron2006}. Entre estos sensores, el LiDAR ha recibido menos atención para la estimación de la pose. Esto se debe a que los LiDAR bidimensionales no pueden ofrecer una estimación completa de la pose\footnote{De lo que refiere a la orientación y posicion en tres dimensiones} y los LiDAR tridimensionales son voluminosos y costosos.

En la localización al aire libre, la señal del sistema de posicionamiento global (GPS) suele estar disponible y, dado que la posición se puede obtener con precisión mediante GPS, es la opción común para la fusión con los datos obtenidos de la IMU \cite{engel2014}. Sin embargo, en un entorno denegado por GPS, como dentro de un edificio, se deben usar otros sensores para corregir las estimaciones de IMU. Un enfoque común para abordar este problema es la fusión IMU-cámara \cite{mirzaei2008}, \cite{hesch2009}, \cite{chambers2014}, \cite{hesch2013}, y en ciertos casos combinadas con LiDAR \cite{lee2016}.

Un algoritmo SLAM robusto es esencial para que cualquier robot móvil navegue de manera segura a través de un entorno no estructurado. El algoritmo SLAM con frecuencia define un marco de coordenadas global para que el robot opere; uno que generalmente es utilizado por todas las funcionalidades de alto nivel, como navegación, planificación de rutas, exploración, identificación de objetos, seguimiento de objetos y manipulación de los mismos. Estas dependencias hacen que el algoritmo SLAM sea una parte central de cualquier arquitectura de robot móvil, y las fallas irrecuperables son altamente indeseables. Es probable que cualquier robot desorientado por una falla de SLAM no pueda realizar su tarea, o peor, puede poner en peligro a los humanos, a sí mismo o al medio ambiente.

Si bien se han demostrado varios algoritmos SLAM en laboratorios, es más difícil producir soluciones robustas en entornos no estructurados del mundo real. El deslizamiento de las ruedas, por ejemplo, puede producir mediciones de odometría ruidosas y sesgadas, mientras que los sensores del sistema de posicionamiento global (GPS) a menudo producen una localización global muy ruidosa y sesgada. El SLAM robusto del mundo real se vuelve aún más difícil cuando estos errores de detección aleatorios y sistemáticos ocurren en entornos que tienen geometrías redundantes u objetos en movimiento.
        \ifimagenes
\section{Marco teórico}
\else
\section{Reconstrucción del entorno}
\fi
\label{sec:marcoteorico}
En esta sección se presentan distintos métodos para poder realizar una reconstrucción del entorno en base a las nubes de puntos, además de dar una introducción a la librería de nube de puntos (PCL) y su vínculo con ROS, utilizados para el desarrollo del trabajo.

\ifimagenes
\subsection{Reconstrucción del entorno}
\else
\fi
En base a los datos provistos por los sensores exteroceptivos (LIDAR, cámaras, entre otros), el objetivo principal del trabajo consiste en recolectar dicha información para poder reconstruir el entorno en el cual se encuentra sumergido el robot, además de poder estimar su posición. Sin embargo, estos tipos de sensores proveen un campo de visión limitado, por lo que no es posible describir el mundo real como un todo en base a una sola medición de estos sensores, sino que solo puede mencionarse a una pequeña porción del mismo, denominada \textit{escena}.

A su vez, el tipo de dato que se tome de la escena dependerá del tipo de sensor extereoceptivo que los provea. Por ejemplo, una cámara monocular es capaz de otorgar imágenes 2D, mientras que con una cámara estéreo se consigue una nube de puntos 
\ifimagenes
    \ifimagenespaper
tridimensional.
    \else
tridimensional, tal como se describe en el apéndice.
    \fi
\fi

\subsection{Point cloud}
Una \textit{nube de puntos}, o de su terminología en inglés, \textit{point cloud}, se utiliza para describir a un conjunto de puntos de datos en un espacio dado. Las nubes de puntos 3D, por ejemplo, se tratan de un conjunto de puntos tridimensionales que se caracterizan por tener principalmente coordenadas espaciales XYZ, y opcionalmente pueden dar información de intensidad, color, entre otras.

Como se mencionaba anteriormente, la nube de puntos de la escena adquirida dependerá del principio de medición del sensor utilizado. En concreto, puede categorizarse en: \cite{weinmann2016}
\begin{itemize}
    \item \textit{Técnicas pasivas}, donde la luz ambiental se encuentra presente, permitiendo utilizar sensores como cámaras estéreo para obtener las imágenes propias del entorno.
    \item \textit{Técnicas activas} donde los sensores emiten radiaciones electromagnéticas, tal como es el caso de los LIDAR 3D o las cámaras infrarrojas.
\end{itemize}

Dependiendo de la técnica de adquisición involucrada y el dispositivo utilizado, los datos de la nube de puntos adquiridos pueden corromperse con más o menos ruido y, además de la información espacial en forma de coordenadas XYZ, los atributos de los puntos respectivos, como la información de color o intensidad también puede adquirirse, como se mencionaba anteriormente.

\subsection{Point Cloud Library (PCL)}
La \textit{Point Cloud Library (PCL)} se trata de un proyecto abierto desarrollado en \textit{C++} que ofrece herramientas para procesar imágenes o nubes de puntos tanto 2D como 3D. El framework PCL contiene numerosos algoritmos que realizan filtrado, estimación de características, reconstrucción de superficies, registro, ajuste de modelos y segmentación. Con estos métodos, es posible procesar la nube de puntos, extraer keypoints para reconocer objetos en el mundo en función de su apariencia geométrica, crear superficies a partir de las nubes de puntos y visualizarlas.

En general, PCL contiene una estructura de datos muy importante, que es $pcl::PointCloud$. Esta estructura de datos está diseñada como una clase de template que toma el tipo de punto que se utilizará como parámetro. Como resultado de esto, la clase de nube de puntos no es mucho más que un contenedor de puntos que incluye toda la información común requerida por todas las nubes de puntos independientemente de su tipo de punto. Algunos de los tipos de puntos más utilizados son
\begin{itemize}
    \item $pcl::PointXYZ$, el cual es el más simple que posee la librería, almacenando la información en XYZ únicamente
    \item $pcl::PointXYZRGB$, almacena además de la posición XYZ, el color en formato RGB de cada punto.
    \item $pcl::PointNormal$, representa la superficie normal en un punto dado y la medida de su curvatura, además de las coordenadas XYZ.
    \item $pcl::PointXYZI$, que a las coordenadas XYZ les asocia también información de intensidad del punto.
\end{itemize}

Debido al gran número de tipos de puntos que existen en la librería, cada algoritmo implementado en la misma refiere a una clase perteneciente a una jerarquía de clases con puntos en común específicos. Gracias a ello, dichos algoritmos pueden ser parametrizados en base a lo que necesite el usuario. 

\ifimagenes
    \ifimagenespaper
    \else
    \subparagraph{Interfaz con ROS}
    \fi
\fi
La interfaz PCL para ROS proporciona los medios necesarios para comunicar las estructuras de datos PCL a través del sistema de comunicación basado en mensajes proporcionado por ROS. Para hacerlo, hay varios tipos de mensajes definidos para contener nubes de puntos, así como otros productos de datos de los algoritmos PCL. En combinación con estos tipos de mensajes, también se proporciona un conjunto de funciones de conversión para convertir de tipos de datos PCL nativos en mensajes. Uno de los más importantes tipos de mensajes de ROS es el $sensor_msgs::PointCloud2$, el cual contiene una colección de puntos N-dimensionales, que pueden contener información adicional como normales, intensidad, entre otros.

Para relacionar ambos mundos, ROS cuenta con el paquete \textit{pcl\_conversions}\footnote{http://wiki.ros.org/pcl\_conversions}. Por ejemplo, en el Codigo \ref{lst:pc2pcl} se convierte un mensaje del tipo \lstinline{sensor_msgs::PointCloud2} a \lstinline{pcl::PointCloud<pcl::PointXYZRGB>}.

\begin{lstlisting}[caption=Conversión de mensaje ROS a PCL, label=lst:pc2pcl]
  // ROS PointCloud2
  sensor_msgs::PointCloud2 &pc2;
  
  // Output Cloud
  pcl::PointCloud<pcl::PointXYZRGB>::Ptr cloud_out(new pcl::PointCloud<pcl::PointXYZRGB>);
  // PCL PointCloud2
  pcl::PCLPointCloud2 pcl_pc2;   

  // Convert from ROS message to PCL data
  pcl_conversions::toPCL(pc2, pcl_pc2);
  pcl::fromPCLPointCloud2(pcl_pc2, *cloud_out);
\end{lstlisting}


\subsection{Adquisición de la nube de puntos}
Como se menciona en el documento, dependiendo del sensor utilizado para obtener los datos, se conseguirán las nubes de puntos asociadas dependiendo del principio de 
\ifimagenes
medición. A continuación se menciona en primera instancia una librería usada comúnmente en este tipo de aplicaciones (OpenCV), además de los sensores utilizados generalmente en este tipo de aplicaciones.

\subsubsection{OpenCV}
OpenCV\footnote{http://opencv.org} es una librería open source multiplataforma que busca brindar una infraestructura de visión artificial fácil de usar y que pueda utilizarse para aplicaciones de tiempo real, por lo que hace hincapié en la optimización. Para ello, OpenCV cuenta con un set de más de 500 funciones \cite{kaehler2017} que abarcan muchas áreas en visión, tales como calibración de cámaras, visión estéreo y robótica.

\subsubsection{Cámara estéreo}
Con el fin de poder simular el efecto de la percepción humana, las cámaras estéreo se ubican separadas una distancia fija y, por lo general, alineadas horizontalmente, como se observa la Bumblebee 2 en la Figura \ref{fig:stereoandrgbdcameras}.a. Al tener dos fuentes de imágenes distintas y ubicadas en una posición relativa conocida, es posible a partir de las mismas determinar la profundidad en la que se encuentran los objetos que ven entre ambas. 

Si bien se puede realizar con una cámara monocular, la técnica cuenta con mayor complejidad y con ciertas restricciones\footnote{Por ejemplo, sería necesario conocer una cierta cantidad de puntos claves del objeto para que, estando el mismo en una pose distinta a la analizada a priori, sea posible determinar la pose del objeto mediante estos puntos claves.}, aunque puede solventarse mediante el agregado de sensores que utilicen otro principio de medición\footnote{Las cámaras RGB-D, por ejemplo, cuentan con una cámara y sensores de distancia para determinar la ubicación de determinados píxeles.}.

Para poder conseguir una nube de puntos mediante el uso de dos cámaras, deben seguirse una serie de pasos \cite{kaehler2017}
\begin{enumerate}
    \item \textit{Remover las distorsiones} de la lente.
    \item Ajustar las distancias y ángulos entre las cámaras, conocido como \textit{rectificación}.
    \item Encontrar las mismas características en ambas cámaras, proceso conocido como \textit{correspondencia}. A partir de este se consigue un \textit{mapa de disparidad} (o \textit{disparity map}), donde las disparidades son las diferencias en la coordenada x de los planos de la imagen de la misma característica vista en las cámaras izquierda y dercha.
    \item Si se conoce el arreglo geométrico de las cámaras, puede cambiarse el mapa de disparidad en distancias por \textit{triangulación}. Este paso es denominado \textit{proyección}, y el resultado es un mapa de profundidad.
\end{enumerate}

Para obtener mejores resultados, es necesario que las cámaras estén sincronizadas entre sí, caso contrario los objetos en movimiento serán un problema.

\begin{figure}
    \centering
    \includegraphics[width=.9\linewidth]{Img/bumblekinect}
    \caption{Cámaras: (a) Bumblebee 2, (b) Microsoft Kinect}
    \label{fig:stereoandrgbdcameras}
\end{figure}

\paragraph{Calibración}
La calibración estéreo depende de hallar la matriz de rotación y la traslación entre las dos cámaras previamente calibradas \cite{kaehler2017}, lo que puede realizarse mediante la función de OpenCV \texttt{cv::stereoCalibrate()}. En el mismo, se busca una sola matriz de rotación y un vector de traslación que relacione la cámara derecha con la cámara izquierda

Para la calibración en ROS se puede utilizar, al igual que en cámaras monoculares, el paquete \texttt{camera\_calibration}, el cual en este caso recibirá parámetros distintos. La interfaz gráfica para la calibración será similar al de cámaras monoculares, con la diferencia que en este caso aparecen ambas cámaras, aunque el principio de calibración del lado usuario es prácticamente el mismo, teniendo en cuenta que ahora en ambas cámaras debe visualizarse el patrón buscado, tal como se observa en la Figura \ref{fig:stereocalibrationchessboardrosexample}.
\begin{figure}
    \centering
    \includegraphics[width=\linewidth]{Img/StereoCalibrationChessboard.png}
    \caption{Calibración estéreo mediante ROS}
    \label{fig:stereocalibrationchessboardrosexample}
\end{figure}


\paragraph{Rectificación estéreo}
Si los planos de ambas cámaras no se encuentran perfectamente alineados, la disparidad estéreo aumenta su complejidad. Para poder mitigar este problema, lo que se busca es reproyectar los planos de la imagen de ambas cámaras para que residan en el mismo plano, conocido como \textit{rectificación estéreo}. Dentro de OpenCV, existen numerosas formas de obtenerlo, como pueden ser los algoritmos de Hartley o de Bouguet. El algoritmo de Hartley permite evitar la calibración de la cámara, en cambio para el segundo es necesaria.

Una vez que se obtienen los términos de la calibración estéreo, se pueden calcular los mapas de rectificación izquierda y derecha por separado utilizando para cada uno \texttt{cv::initUndistortRectifyMap()}.

En ROS, el paquete utilizado para dicha tarea es el denominado \texttt{stereo\_image\_proc}\footnote{\url{http://wiki.ros.org/stereo_image_proc}}, que permite obtener un mapa de profundidades. 

\paragraph{Correspondencia estéreo}
La correspondencia estéreo, esto es, coincidir un punto tridimensional en las vistas de ambas cámaras, requiere que estos puntos de ambas cámaras se solapen. Para ello, OpenCV implementa dos algoritmos distintos para correspondencias, convirtiendo dos imágenes (izquierda y derecha) en una sola imagen de profundidad
\begin{itemize}
    \item \textit{Block matching (BM) algorithm}, implementada mediante \texttt{cv::StereoBM()}, el cual es rápido y efectivo, aunque en escenas de baja textura presenta problemas. 
    \item \textit{Semi-global block matching (SGBM)}, implementada mediante \texttt{cv::StereoSGBM()}, el cual presenta una mayor exactitud que el BM, pero es computacionalmente más demandante.
\end{itemize}

En base a este proceso es que puede obtenerse una \textbf{nube de puntos} del entorno.

Para el caso de ROS, dentro del paquete \texttt{stereo\_image\_proc} se encuentra el ejecutable \texttt{dynamic\_reconfigure}\footnote{\url{http://wiki.ros.org/stereo_image_proc/Tutorials/ChoosingGoodStereoParameters}}, que permite modificar dinámicamente los parámetros de correspondencia, además de elegir el algoritmo a utilizar mediante una interfaz gráfica.

\subsubsection{Cámara RGB-D}
Las cámaras RGB-D\cite{hogman2011} en lugar de poseer dos cámaras monoculares presentan una cámara monocular con la ayuda de sensores infrarrojos para determinar la ubicación de determinados píxeles, siendo un ejemplo conocido de las mismas es la Microsoft Kinect, la cual se observa en la Figura \ref{fig:stereoandrgbdcameras}.b. Para que el mismo pueda obtener la nube de puntos, primero el sensor de distancia proyecta un patrón de moteado infrarrojo. El patrón proyectado es luego capturado por una cámara de infrarrojos en el sensor y comparado parte por parte con patrones de referencia almacenados en el dispositivo. Estos patrones fueron capturados previamente a profundidades conocidas en el proceso de calibración de los mismos. A continuación, el sensor estima la profundidad por píxel en función de los patrones de referencia con los que el patrón proyectado coincide mejor. Los datos de profundidad proporcionados por el sensor de infrarrojos se correlacionan luego con una cámara RGB calibrada. Esto produce una imagen RGB con una profundidad asociada a cada píxel. Una representación unificada popular de estos datos es una nube de puntos: una colección de puntos en un espacio tridimensional, donde cada punto puede tener características adicionales asociadas. Con un sensor RGB-D, el color puede ser una de esas características. Además, las normales de superficie aproximadas también se almacenan a menudo con cada punto en la nube de puntos resultante.
\else
medición, tal como se desarrolló en la Sección \ref{sec:sensors}.
\fi
\subsection{Point cloud registration}
Si bien un primer paso es obtener la nube de puntos en base al sensor utilizado, uno de los principales enfoques utilizados en robots móviles con los mismos es el llamado \textit{registro de nube de puntos} o, de su terminología en inglés, \textit{point cloud registration}. Dicho proceso responde a encontrar una transformación espacial que alinee dos nubes de puntos generalmente contiguas en el tiempo. Los sensores que utilizan normalmente este enfoque son las cámaras estéreo y RGB-D.

En concreto, dada una nube de puntos fuente (tambíen llamada \textit{input} o \textit{source}) $P$ con puntos $p \in P$, y una nube de puntos objetivo $Q$ (llamada \textit{target}) con puntos $q \in Q$, el problema del registro se basa en encontrar correspondencias entre $P$ y $Q$, y estimar una transformación $T$ que, cuando se aplica a $P$, se alinea todos los pares de puntos correspondientes ($p_i \in P$, $q_j \in Q$). Un problema fundamental del registro es que estas correspondencias generalmente no se conocen y deben ser determinadas por el algoritmo de registro. Dadas las correspondencias correctas, hay diferentes formas de calcular la transformación óptima con respecto a la métrica de error utilizada, como se detalla a continuación.

El registro de dos nubes de puntos puede dividirse en una serie de pasos, los cuales se observan en la Figura \ref{fig:registrationpipeline} y conforman el denominado \textit{registration pipeline}. El mismo si bien puede extenderse para un caso más general \cite{garcia2017}, en este caso se da una estructura representativa del trabajo presente, siendo dichos pasos la \textit{selección}, \textit{matcheo}, \textit{rejection} y \textit{alineación}.

\begin{figure}[!ht]
    \centering
    \includegraphics[width=0.45\linewidth]{Img/3DRegistrationPipeline.png}
    \caption{Registration pipeline}
    \label{fig:registrationpipeline}
\end{figure}

En primera instancia, debido a la gran densidad de puntos que presentan las nubes de puntos, además del ruido inherente de los sensores utilizados, es necesario realizar un proceso de \textit{selección} de los datos de interés, no solo para que el \textit{proceso de registro} pueda converger al valor óptimo, sino también para reducir el tiempo que tarda el algoritmo en completar el procesamiento.

Una vez que se tienen los datos seleccionados de cada una de las nubes de puntos a alinear, es necesario encontrar las correspondencias entre las mismas, esto es, encontrar un conjunto de puntos (los cuales suelen ser los \textit{keypoints}) en la nube de puntos fuente que se pueden \textit{identificar como los mismos puntos} en la nube de puntos objetivo.

En el proceso descripto anteriormente, normalmente una gran proporción de las correspondencias consideradas como tales en realidad son desajustes debidos al cambio de punto de vista, oclusión \cite{barazzetti2018}, entre otros. Estos desajustes suelen ser suficientes para arruinar los métodos de estimación tradicionales. Por lo tanto, es necesario eliminar o reducir la influencia indebida de los desajustes. Por ello, luego del pareo de los puntos característicos es necesario \textit{rechazar} las correspondencias para reducir el número de valores atípicos.

Finalmente, con las correspondencias filtradas, se busca la transformación que se ajusta mejor a ambas nubes de puntos mediante un proceso conocido como \textit{alineación}.

% \ifimagenes
% \else
% %% VA O NO VA?????
% A partir de la clasificación anterior, pueden diferenciarse dos tipos de algoritmos de registro, el \textit{registro basado en características (features)}, y los \textit{algoritmos de registro iterativos}, que se observan en la Figura \ref{fig:registrationprocess} y se detallan a continuación.
% \begin{itemize}
%     \item \textit{Registro basado en características (features)}, para calcular alineaciones iniciales, y
%     \item \textit{Algoritmos de registro iterativos}, siguiendo el principio del algoritmo ICP para iterativamente registrar nube de puntos (que ya se encuentran relativamente alineadas).
% \end{itemize}

% \begin{figure}[!ht]
%     \centering
%     \includegraphics[width=\linewidth]{Img/RegistrationProcess.png}
%     \caption{Registration process}
%     \label{fig:registrationprocess}
% \end{figure}

% Para el registro basado en características, los descriptores de características geométricas se calculan y combinan en algún espacio de alta dimensión. Cuanto más descriptivos, únicos y persistentes sean estos descriptores, mayor será la probabilidad de que todas las \textit{coincidencias} (o \textit{correspondencias}) encontradas sean pares de puntos que realmente se correspondan entre sí.

% En el algoritmo ICP de Besl y McKay \cite{besl1992} no se calculan descriptores de características, sino que se considera que los puntos más cercanos en el espacio cartesiano se corresponden entre sí. Se estima una transformación que minimiza las distancias euclidianas entre pares encontrados de puntos más cercanos en el sentido de mínimos cuadrados. El proceso de determinar los puntos correspondientes en los dos conjuntos de datos y calcular la transformación que los alinea se repite iterativamente. Se espera que el conjunto de puntos de origen converja hacia el conjunto de destino a medida que las correspondencias sean cada vez mejores. Simultáneamente, Chen y Medioni \cite{chen1992} formularon un algoritmo similar, pero en lugar de minimizar las distancias euclidianas al cuadrado entre los puntos correspondientes, aplicaron una métrica de error de punto a plano.
% %%%%%
% \fi
A continuación, se desarrollan los pasos nombrados anteriormente.

\subsubsection{Selección}
Como la información suele ser redundate y lo suficientemente densa como para necesitar un gran costo computacional a la hora de procesar los datos, es necesario filtrar las nubes de puntos quitando la información irrelevante, reduciendo así el tiempo de ejecución de los algoritmos. Si bien existen en la literatura muchos criterios para decidir cuales puntos tomar y cuales no, en principio se pueden distinguir dos métodos de reducción de datos, siendo el primero el de extraer automáticamente un pequeño conjunto de keypoints únicos y repetibles, mientras que el otro se basa en muestrear los datos originales con respecto a una distribución objetivo deseada. Mientras que el primero está destinado a la alineación inicial basada en características \cite{merino2016}, el segundo se puede utilizar para reducir de manera eficiente la cantidad de datos para los algoritmos de registro iterativos. PCL implementa varios de estos métodos de muestreo, en particular, el muestreo en el espacio índice (simplemente tomando cada n-ésimo punto), submuestreo uniforme en el espacio 3D de entrada (para capturar mejor las estructuras ambientales detectadas) y muestreo en el espacio de normales de superficie (para muestrear puntos en todas las orientaciones de la superficie). A continuación se presentan dos métodos que suelen emplearse en este tipo de tareas
\begin{itemize}
    \item \textit{Normal space sampling}: Al tratar generalmente con modelos suaves con pequeñas irregularidades (por ejemplo, un plano), el proceso puede resultar en el muestreo de muchos puntos que contienen esencialmente la misma información en términos de vectores normales. Por ello, la estrategia de \textit{normal space sampling} pretende dar uso a esta información \cite{rusinkiewicz2001}, y la misma consiste en, en primera instancia, agrupar puntos en ''cubos'' de acuerdo con los ángulos entre sus vectores normales (considerados en la esfera unitaria) y los ejes de coordenadas, y luego muestrear uniformemente sobre los cubos resultantes, proporcionando un submuestreo de los puntos con más vectores normales "frecuentes". Dentro de PCL, este algoritmo se encuentra implementado en el método \lstinline{pcl::NormalSpaceSampling}.
    \item \textit{Harris 3D}: El operador de Harris \cite{harris1988} se trata de un detector de puntos de interés para imágenes. El método es una técnica popular debido a su fuerte invariancia a la rotación, escala, variación de iluminación, y ruido de imagen \cite{schmid2000}. El detector de Harris se basa en la función de autocorrelación local de una señal, que mide los cambios locales de la señal con parches desplazados una pequeña cantidad en diferentes direcciones. El mismo se ha utilizado en muchas aplicaciones en procesamiento de imágenes y visión artificial por su sencillez y eficiencia. Sin embargo, el problema con los datos 3D es que la topología es arbitraria y no está claro cómo calcular las derivadas. Para solucionar este problema, en \cite{sipiran2011} proponen transformar a los puntos de la nube de la siguiente manera
    \begin{enumerate}
        \item Por cada punto de la nube, se define un \textit{punto vecino} del mismo, y se calcula el centroide de este último
        \item Todos los puntos de la nube se trasladan para que el centroide coincida con el origen de coordenadas
        \item Luego, se calcula un plano de ajuste a los puntos traslada
        \item Se aplica el \textit{análisis de componentes principales} (PCA de sus siglas en inglés) \cite{jollife2016} al conjunto de puntos y se elige el \textit{eigenvector} con el \textit{eigenvalue} asociado más bajo como la normal del plano de ajuste.
        \item Se rotan los puntos hasta que la normal al plano coincide con el eje z.
        \item El plano XY resultante se utiliza para calcular las derivadas. Estas derivadas se calculan utilizando una superficie cuadrática de seis términos (paraboloide) ajustada al conjunto de puntos transformados.
    \end{enumerate}
    Dentro de PCL, en el método \lstinline{pcl::HarrisKeypoint3D} se encuentra una implementación de dicho algoritmo.
\end{itemize}

\subsubsection{Estimación}
La estimación de correspondencias es el proceso de emparejar los puntos $p_i$ desde la nube de puntos fuente $P$ con sus vecinos más cercanos $q_j$ en la nube objetivo $Q$. 

En PCL, con la función \lstinline{pcl::registration::DetermineCorrespondences} se obtiene el conjunto de pares de correspondencia encontrados entre el la nubes de puntos fuente y objetivo. Cada par consta del índice del punto en la nube de origen y el índice de la coincidencia encontrada en la nube de puntos de destino. 

En el caso de datos de entrada provenientes de sensores que cumplen con el modelo de cámara estenopeica \cite{kaehler2017}, el procedimiento de estimación de correspondencia puede acelerarse significativamente con la contraparte de perder algo de precisión. Estos sensores incluyen cámaras RGB-D populares como Microsoft Kinect, y recopilan información del entorno en forma de imágenes de profundidad y color. En árboles PCL para búsquedas de vecinos más cercanos en el espacio 3D, es posible utilizar la naturaleza proyectiva de las imágenes de profundidad para obtener una aproximación razonable. Cada punto de la nube corresponde a un píxel en la imagen de profundidad, lo que permite proyecciones desde puntos de origen en coordenadas mundiales hasta el plano de la cámara del marco de destino mediante el uso de parámetros de cámara intrínsecos y extrínsecos. Este enfoque es rápido, pero impreciso para nubes de puntos con grandes discontinuidades de profundidad o para cuadros que están muy alejados entre sí. Es por eso que se recomienda usar este método solo después de que las dos nubes de puntos se hayan acercado, lo que lo hace bueno para alinear nubes de puntos consecutivas en una secuencia grabada a alta velocidad de cuadros.

%% QUE ONDA CON ESTO???
% Los métodos que extraen pocos keypoints pueden utilizar la fuerza bruta para encontrar correspondencias. Sin embargo, este proceso es computacionalmente costoso. Algunos métodos reducen el tiempo de cálculo minimizando el espacio de búsqueda, como 3D Shape Context [69] que preselecciona los posibles candidatos que satisfacen ciertos criterios y luego aplica fuerza bruta con estos candidatos. No obstante, en la mayoría de las situaciones, se necesitan algoritmos más elaborados para informar los resultados en un período de tiempo razonable.

% Como se necesitan al menos tres puntos no coplanares en cada conjunto para determinar una transformación rígida entre dos conjuntos de puntos 3D sin ambigüedad, el costo asintótico de tales enfoques está en O (n6). En consecuencia, el espacio para navegar en la búsqueda de correspondencias es enorme. Diseñar una estrategia de búsqueda sofisticada que sea capaz de aprovechar la información de detección y descripción tiene el potencial de reducir en gran medida los costos de cálculo y, por lo tanto, aumentar el rango de aplicación de tales algoritmos de registro. Los métodos existentes que implementan estrategias de búsqueda ya logran muy buenos resultados en comparación con los métodos típicos de fuerza bruta.

% A diferencia de los algoritmos de alineación, la finalidad de estas estrategias de coincidencia es lograr solo una alineación aproximada. La idea es identificar la posición arbitraria de las formas de entrada y encontrar las transformaciones entre ellas, lo más rápido posible. La precisión no es, por tanto, el factor más importante. En cambio, la robustez es clave proporcionando garantías para el posterior ajuste fino.
% A continuación se presentan algunos métodos conocidos de la literatura.

% \paragraph{Métodos basados en RANSAC}
% RANdom SAmple and Consesus (RANSAC) es un método iterativo diseñado para encontrar los parámetros de un modelo a partir de un conjunto de datos que contiene valores atípicos (\textit{outliers}). Dada una entrada de datos ruidosos, RANSAC encuentra los parámetros que ajustan los datos de entrada a un modelo dado, descartando los valores atípicos. Este enfoque es la base de una amplia variedad de métodos. Uno de ellos es el enfoque presentado por [34] que se basa en el hecho de que podemos determinar una transformación rígida con solo tres puntos (una base B). La idea es encontrar una base en una de las formas y encontrar la base correspondiente en la otra forma. El algoritmo funciona de la siguiente manera: 
% \begin{itemize}
%     \item primero, determina tres puntos diferentes al azar en la primera superficie: primario (ap), secundario (as) y auxiliar (aa). Considere que las distancias entre estos tres puntos son dps, dpa y dsa. Cada punto en la segunda superficie se considera como el punto correspondiente bp del punto primario ap en la primera superficie.
%     \item Luego, se busca la correspondencia del punto secundario en la segunda superficie a una distancia dps de bp. Si no existe ningún punto alrededor de bp a la distancia dps, descarte bp y comience de nuevo con otro punto primario en la segunda superficie. Sin embargo, si hay un punto secundario bs busque el punto auxiliar ba que satisface las distancias. La transformación entre ambas superficies se puede determinar cuando se identifica la base BB en la segunda superficie. Esta búsqueda se repite para todas las bases encontradas. La mejor transformación es la que tiene el mayor número de puntos correspondientes.
% \end{itemize}

% Aunque este método es robusto incluso con valores atípicos, el principal inconveniente es su tiempo de cálculo. De hecho, este método solo se puede utilizar con una pequeña cantidad de datos de entrada.

% Dentro de PCL, el método \lstinline{pcl::RandomSampleConsensus} representa una implementación del algoritmo RANSAC, tal como se describe en \textbf{[Fischler,1981]}.

% \paragraph{}
%%

%% VER SI PONER O QUE HACER







% \subsubsection{Decimación y filtrado}
% Uno de los grandes problemas que presentan las nubes de puntos al querer hacer un procesamiento de las mismas refieren a la gran densidad de puntos y al\usepackage{comment} ruido en si. Para el primer caso, por ejemplo, si se tiene una nube de 640x480 (estándar), se tendrían que procesar entonces 307200 puntos, generando que se requiera mucho tiempo para completar el procesamiento del algoritmo. El segundo problema, en cambio, genera que se malinterprete la información, provocando resultados incorrectos. 

\subsubsection{Rechazo}
Dado que las correspondencias no válidas pueden afectar negativamente los resultados del registro, la mayoría de los procesos de registro presentan un paso de rechazo. El mismo consiste en filtrar los pares de puntos emparejados en la etapa anterior para facilitar el algoritmo de estimación de la transformación hacia la convergencia al mínimo global. Este paso puede aprovechar la información auxiliar disponible de las nubes de puntos de entrada, como las normales de superficie locales o las estadísticas sobre las correspondencias. A continuación se detallan algunos de los algoritmos más utilizados
\begin{itemize}
    \item \textit{Rechazo de correspondencias basado en la distancia}: en base a un umbral, se filtran las correspondencias que estén a una distancia mayor que dicho límite. En PCL, se implementa mediante \lstinline{pcl::registration::CorrespondenceRejectorDistance}.
    \item \textit{Rechazo en base a la distancia mediana}: a diferencia del método anterior, en este caso el umbral se calcula en base a la mediana de todas las distancias punto a punto en las correspondencias estimadas inicialmente. Se utiliza la mediana ya que suele ser más efectiva en reducir la influencia de valores atípicos. El método \lstinline{pcl::registration::CorrespondenceRejectorMedianDistance} implementa dicha operación.
    \item \textit{Rechazo basado en RANSAC}: Este método aplica el RANdom SAmple Consensus \cite{fischler1981} para estimar una transformación para subconjuntos del conjunto dado de correspondencias y elimina las correspondencias atípicas basadas en la distancia euclidiana entre los puntos después de que la transformación calculada se aplica a la nube de puntos fuente. Es muy eficaz para evitar que el algoritmo ICP converja en mínimos locales, ya que siempre produce correspondencias ligeramente diferentes y es bueno para filtrar valores atípicos. Además, proporciona buenos parámetros iniciales para la estimación de la transformación con todas las correspondencias internas que siguen. El mismo cuenta con el método de PCL \lstinline{pcl::registration::CorrespondenceRejectorSampleConsensus}.
    \item \textit{Rechazo basado en la compatibilidad normal}: este filtro usa la información normal sobre los puntos y rechaza aquellos pares que tienen normales inconsistentes, es decir, el ángulo entre sus normales es mayor que un umbral dado. Puede rechazar pares erróneos que parecen correctos cuando se juzgan solo por la distancia entre los puntos. El mismo es implementado por \lstinline{pcl::registration::CorrespondenceRejectorSurfaceNormal}.
\end{itemize}

\subsubsection{Alineación}
A lo largo de los años, ha habido numerosos enfoques matemáticos para resolver la transformación rígida $T$ que minimiza el error de los pares de puntos. $T$ se compone de una rotación $R$ y una traslación $t$. Tenga en cuenta que, a continuación, cuando se hace referencia a una transformada $T$ y un punto $p$, se utilizarán coordenadas homogéneas. Hay dos métricas de error principales que se deben minimizar y que se han considerado en la literatura: \textit{punto a punto} (Ec. 2) y \textit{punto a plano} (Ec. 3), donde ($p_k$, $q_k$) es el \textit{k-ésimo} de el par $N$ corresponde desde la nube de origen a la nube de destino.
\begin{itemize}
    \item Métrica de error estándar de punto a punto: La métrica de error estándar utilizada en el algoritmo ICP es la métrica de error de punto a punto (Ec. 2). Fue mencionado por primera vez por Arun \cite{arun1987}; los investigadores propusieron varias formas de minimizarlo, seguidas de la introducción del algoritmo ICP \cite{besl1992}. Eggert y col. \cite{eggert1997} evaluó cada uno de estos métodos en términos de estabilidad numérica y precisión, llegando a la conclusión de que tienen un desempeño cercano. PCL ofrece una implementación (\lstinline{pcl::registration::TransformationEstimationSVD}) utilizando \textit{descomposición en valores singulares} (SVD), propuesto en primera instancia por \cite{horn1987}.
    \item \textit{Métrica de error de punto a plano}: Chen y Medioni \cite{chen1992} introdujeron la métrica de punto a plano (Ec. 3) y demostraron que es más estable y converge más rápido que los enfoques anteriores. Utiliza la distancia entre el punto de origen $\bm{p}_k$ y el plano descrito por el punto de destino $\bm{q}_k$ y su normal de superficie local $n_{\bm{q}_k}$. A diferencia de la métrica punto a punto, no tiene una solución de forma cerrada, por lo que la minimización se realiza con solucionadores no lineales (como Levenberg-Marquadt), o linealizándolo \cite{low2004} (bajo el supuesto de ángulos de rotación pequeños, es decir, $sin \theta \sim \theta$ y $cos \theta \sim 1$). Dependiendo de la superficie subyacente y la distribución de puntos, el uso de la métrica de error de punto a plano puede ser considerablemente más robusto. Un procedimiento estándar para minimizarlo se basa en el optimizador no lineal de Levenberg-Marquardt \cite{fitzgibbon2001}. Dicha funcionalidad se emplea en \lstinline{pcl::registration::TransformationEstimationPointToPlane}.
    \item \textit{Métrica de error punto-a-plano ponderada}: Asignar un peso diferente a cada correspondencia puede mejorar la convergencia. La ponderación de los pares de puntos puede verse como un rechazo de correspondencia suave, ajustando la influencia de los puntos correspondientes ruidosos en el proceso de minimización. La ponderación puede ser una función de la distancia punto a punto o punto a plano entre los puntos, una función del ángulo entre las normales correspondientes a los puntos, o una función del modelo de ruido del sensor que se ha usado. Un ejemplo de PCL puede ser \lstinline{pcl::registration::TransformationEstimationPointToPlaneWeighted}.
\end{itemize}

\subsection{Point Cloud 2D}
Si bien las nubes de puntos tridimensionales son una generalización de aquellas que refieren a dos dimensiones, el hecho de tener información sólo de un plano reduce drásticamente la cantidad de puntos a analizar. Este caso sucede en sensores como los LIDAR 2D, donde midiendo el tiempo de vuelo de la señal láser se logran conocer la distancia de distintos puntos de los objetos respecto al sensor. Es por esto que el enfoque de reconstrucción del entorno no suele ser el mismo que el descripto anteriormente.

\subsubsection{Algoritmo para LIDAR 2D}
Al igual que con el método descripto anteriormente, en la reconstrucción de mapas 2D se pretende conseguir la transformación que mejor responda al movimiento del robot entre el instante de tiempo actual y el anterior. Un algoritmo común de coincidencia de escaneo láser encuentra la transformación óptima de cuerpo rígido $T$ que alinea el escaneo láser actual $S_t$ en el tiempo $t$ con el anterior $S_{t-1}$ en el tiempo $t-1$. Este método solo considera dos escaneos láser secuenciales, y cuando se aplican iterativamente para todos los escaneos láser uno por uno, el problema de deriva de pose se deterioraría debido a los errores de coincidencia acumulados, lo que afectará la precisión de la siguiente coincidencia. 

Para abordar este inconveniente, en lugar de comparar los datos en el instante $t$ con el anterior, $t - 1$, se puede comparar dicha información actual con el mapa generado anteriormente, $M_{t-1}$, el cual es generado por todos los escaneos anteriores, de $1$ a $t-1$, y, en caso de ser un occupancy grid map, almacena el valor de probabilidad de cada celda de la cuadrícula en la región del espacio 2D. De acuerdo con las Reglas de Bayes, asumiendo la independencia de cada punto de $S_t$, el valor de probabilidad de la suma de $S_t$ respecto al mapa $M_{t-1}$ se calcula como
\begin{equation}
    p(S_t|M_{t-1}) = \sum_{x\epsilon S_t} p(x|M_{t-1})
\end{equation}
siendo $p(x|M_{t-1})$ la probabilidad de que un punto de escaneo $x \epsilon S_t$ coincida con uno perteneciente a $M_{t-1}$ en esa misma ubicación. Entonces, para buscar la transformación de $S_t$ que mejor se adapte al movimiento del robot respecto al mapa generado anteriormente $M_{t-1}$, llámese $T^*$, puede aplicarse el método de máxima verosimilitud
\begin{equation}
    T^* = argmax(p(T\propto S_t|M_{t-1})
\end{equation}
donde $T\propto S_t$ refiere a los datos del sensor en el instante actual $S_t$ transformados por $T$. A continuación, se presentan una serie de pasos a seguir para lograr el objetivo.

\paragraph{Filtrado de los datos}
Un ejemplo de los datos provenientes de un LIDAR 2D pueden verse en la Figura \ref{fig:lidar2dpoints}. Tal como en el caso de las nubes de puntos tridimensionales, para el caso del LIDAR 2D es necesario quedarse sólo con los datos de mayor relevancia. Por ejemplo, un punto en el espacio sin otros cercanos no aportará mucha información, mientras que una conglomeración de puntos puede resultar en una pared, por ejemplo.

\begin{figure}[!ht]
    \centering
    \includegraphics[width=0.75\linewidth]{Img/LIDAR2DPoints.png}
    \caption{Puntos dados por un LIDAR 2D}
    \label{fig:lidar2dpoints}
\end{figure}

Existen numerosas técnicas para obtener los distintos puntos de interés, ya sea filtrando en base a la distancia entre los puntos, como también la extracción de líneas en base a dichos puntos \textbf{(Gao, 2018)}.

\paragraph{Estimación de la transformación óptima}
En concreto, hay dos formas principales de encontrar la transformación de cuerpo rígido óptima: una es mediante métodos de búsqueda bruta y la otra es la que se basa en el ascenso en gradiente. El método de ascenso en gradiente puede atascarse en el mínimo local, mientras que el método de búsqueda bruta es una búsqueda global y es más robusto. Además, un mapa de múltiples resoluciones y una ventana de búsqueda estrecha pueden mejorar en gran medida la eficiencia de búsqueda del método de búsqueda bruta en una aplicación en tiempo real costosa en tiempo \textbf{(Olson, 2009)(Olson, 2015)}.

\subsubsection{Generación de mapas con LIDAR 2D}
Como los LIDAR en si no aportan información de colores, sino que para el caso de los bidimensionales la misma se trata de puntos en un plano, como se observa en la Figura \ref{fig:lidar2dpoints}, los mapas generados por dichos sensores suelen ser son los \textit{occupancy grid maps} vistos en la Sección \ref{sec:slam}, donde a partir de la ubicación de los puntos provistos por el sensor se determinan los lugares ocupados y libres, siendo las casillas libres las que se encuentran entre el sensor y cada punto y las ocupadas las correspondientes a la posición de los puntos en si, tal como se observa en la Figura \ref{fig:lidar2dmap}. Si bien existen los mapas binarios, en los que se tiene \textit{ocupado, vacio} o \textit{desconocido} solamente, al existir un margen de error en las mediciones, dichos mapas suelen ser probabilísticos, siendo 0.5 el valor por defecto de todas las celdas desconocidas.

\begin{figure}[!ht]
    \centering{{\includegraphics[width=0.75\textwidth]{Img/LIDAR2DMap.png}}}%
    \caption{Mapa generado a partir de los datos del LIDAR 2D}
    \label{fig:lidar2dmap}
\end{figure}

Para poder conseguir el mapa mencionado anteriormente, se tienen que seguir una serie de pasos y tener ciertas consideraciones, las cuales se desarrollan a continuación.

\paragraph{Obtención de las distintas celdas}
Como se trata de un mapa basado en grillas, se deben determinar no solo las celdas que se corresponden con cada punto, sino también aquellas que corresponden a las celdas libres entre los puntos y el sensor.

En el primer caso, al tener la resolución del mapa y la cantidad de cuadros en el mismo, pueden determinarse sin problemas las posiciones de los puntos dados por los datos del LIDAR. 

Para el segundo caso, en cambio, como se dispone a priori de la coordenada del punto y la ubicación del sensor, se podría trazar una recta entre dichos puntos. A partir de la misma, en base a la grilla utilizada, se necesita conocer cuáles celdas atraviesa dicha línea y cuáles no. Un método comúnmente utilizado en este tipo de problemas es el algoritmo de Bresenham \textbf{(Bresenham, 1965)}, permitiendo conseguir los resultados observados en la Figura \ref{fig:bresenhamlinealgorithm}.

\begin{figure}[!ht]
    \centering
    \includegraphics[width=0.9\linewidth]{Img/BresenhamLineAlgorithm.png}
    \caption{Ejemplo del algoritmo de Bresenham en base a una grilla dada. En negro se observa la línea de la cual se parte, y las celdas en gris son las que se obtienen a partir del algoritmo.}
    \label{fig:bresenhamlinealgorithm}
\end{figure}

\paragraph{Ponderación de las celdas}
Como se mencionaba anteriormente, los mapas suelen ser probabilísticos debido al ruido inherente de los sensores. Para poder determinar los valores que debe tomar cada celda (entre 0 y 1), suele ser recurrente el \textit{desenfoque Gaussiano} (del inglés \textit{Gaussian blur}), el cual se basa en desenfocar una imagen mediante una función Gaussiana. Si bien el mismo suele conocerse para una dimensión, puede extenderse a dos dimensiones sin mucho esfuerzo \textbf{(Haddad, 1991)}, ya que se trata del producto de dos Gaussianas, una en cada dimensión.

\paragraph{Actualización del mapa}
Una vez obtenido el mapa basado en la medición actual y, por ende, la transformación que describe el movimiento del robot, se procede a la actualización del mapa generado en los instantes de tiempo anteriores. Para poder realizar esto, una forma sería mediante la actualización de Bayes, vista en la Sección \ref{sec:regressionanalysis} y representada por la Expresión (\ref{eq:posteriorfull}). El problema de la misma es que, si se multiplican valores muy pequeños, el costo computacional aumenta. Una forma de mitigar este inconveniente es mediante el uso de la función \textit{logit}, la cual se basa en un número $p$ entre $0$ y $1$ que responde a la forma

\begin{equation}
    logit(p) = \log\left(\frac{p}{1-p}\right)
\end{equation}
obteniendo entonces valores entre $-\infty$ y $+\infty$, tal como se observa en la Figura .a. Para volver a la forma de probabilidad comúnmente utilizada, observada en la Figura .b, se utiliza la siguiente Expresión, la cual se consigue despejando la anterior
\begin{equation}
    p = \frac{e^{logit(p)}}{1+e^{logit(p)}}
\end{equation}

Partiendo entonces de la Expresión (\ref{eq:posteriorfull}), separando la medición actual de las anteriores y utilizando la suposición de Markov, la cual define que la medición actual es independiente de las mediciones anteriores si el estado actual es conocido, se llega a
\begin{equation}
    p(m_i|y_{1:t}) = \frac{p(y_t|m_i)p(m_i|y_{1:t-1})}{p(y_t|y_{1:t-1})}
\end{equation}
siendo 
\begin{itemize}
    \item $m_i$ la celda actual del mapa,
    \item $y_{1:t}$ las mediciones del sensor en esa celda desde el tiempo $1$ al tiempo $t$, 
    \item $p(y_t|m_i)$ la probabilidad de la medición actual dado el estado de la celda,
    \item $p(m_i|y_{1:t-1})$ la probabilidad de que una celda esté ocupada dadas todas las mediciones anteriores, y
    \item $p(y_t|y_{1:t-1})$ la probabilidad de tener una medición $y_t$ dadas todas las mediciones anteriores $y_{1:t-1}$.
\end{itemize}


Si se expande $p(y_t|m_i)$ mediante la regla de Bayes
\begin{equation}
    p(y_t|m_i) = \frac{p(m_i|y_t)p(y_t)}{p(m_i)}
\end{equation}
y se aplica a la Expresión anterior, se obtiene
\begin{equation}
    p(m_i|y_{1:t}) = \frac{p(m_i|y_t)p(y_t)p(m_i|y_{1:t-1})}{p(m_i)p(y_t|y_{1:t-1})}
\end{equation}

Luego, aplicando la fracción presente en la función \textit{logit}, se puede llegar a
\begin{equation}
    \frac{p(m_i|y_{1:t})}{1-p(m_i|y_{1:t})} = \frac{p(m_i|y_t)(1-p(m_i))p(m_i|y_{1:t-1})}{(1-p(m_i|y_t))p(m_i)(1-p(m_i|y_{1:t-1})}
\end{equation}

Finalmente, aplicando el logaritmo,
\begin{equation}
    logit(p(m_i|y_{1:t})) = logit(p(m_i|y_t)) + logit(p(m_i|y_{1:t-1})) - logit(p(m_i))
    \label{eq:logitupdatemap}
\end{equation}
resultando entonces la actualización del mapa en la adición de tres funciones \textit{logit}, siendo el primer término el referido al valor de la celda en base a la medición actual, el segundo correspondiente al valor de la celda respecto a las mediciones anteriores, y el tercero corresponde al valor de la celda en el instante inicial (siendo desconocido, $m_i = 0.5$ generalmente).

\subsection{Resumen}
En esta Sección, se presentó el concepto de nube de puntos, a su vez de mencionar las características de la \textit{Point Cloud Library} en este tipo de aplicaciones, siendo una herramienta esencial para el desarrollo del trabajo presentado.

A su vez, se presentaron los distintos métodos para la generación de mapas 2D, además de mencionar las técnicas utilizadas para la estimación del movimiento según los datos acutales y los anteriores.
        \section{Resolución}
\subsection{Calibración de la unidad incercial}
\subsubsection{Calibración del giróscopo}
\textbf{ALLAN}
Para poder conocer el tamaño adecuado que permita obtener el \textit{bias} del giróscopo en el instante inicial, se utiliza la \textit{varianza de Allan}[20][8] $\sigma_{Allan}$, la cual mide la varianza de la diferencia entre promedios de intervalos consecutivos, siendo entonces
\begin{align}
    \sigma_{Allan} &= \frac{1}{2} E[(x(\tilde{t},k) - x(\tilde{t},k-1))^2] \\
    &= \frac{1}{2K}\sum_{k=1}^K(x(\tilde{t},k) - x(\tilde{t},k-1))^2
\end{align}
donde $x(\tilde{t},k)$ es el \textit{k}-ésimo intervalo promedio que abarca $\tilde{t}$ segundos, y K es el número de intervalos en que se segmenta el tiempo total considerado. Se computa la varianza de Allan para los tres ejes, y en el intervalo de tiempo en el que los tres convergen a un valor pequeño representa una buena elección para elegir el período de inicialización, $T_{init}$.

\textbf{INTEGRACION}
Resolver esta ecuación diferencial implica poder integrarla. Si bien existen varios métodos para hacerlo, ...


Esta función, en concreto, requiere de una integración de la velocidad angular en un tiempo discreto. Si bien existen diferentes métodos de integración numérica, es necesario que el mismo sea robusto y estable para mejorar la exactitud de la calibración. Por eso, el \textit{Runge-Kutta} $4^th$ \textit{order normalized method} (RK4n)[19] es el elegido.

Si la ecuación diferencial que describe a la cinemática del cuaternión se define como
\begin{equation}
    \bm{f}(\bm{q},t) = \dot{\bm{q}} = \frac{1}{2}\bm{\Omega}(\bm{\omega}(t))\bm{q}
\end{equation}
donde $\bm{\Omega}(\bm{\omega}(t))$ es el operador que convierte la velocidad angular tridimensional considerada en la representación de la matriz simétrica oblicua real, esto es,
\begin{equation}
    \bm{\Omega}(\bm{w}) = 
    \begin{bmatrix}
        0 & -w_x & -w_y & -w_z \\
        w_x & 0 & w_z & -w_y \\
        w_y & -w_z & 0 & w_x \\
        w_z & w_y & -w_x & 0
    \end{bmatrix}
\end{equation}
El algoritmo de integración RK4n es
\begin{align}
    \bm{q}_{k+1} &= \bm{q}_k + \Delta t\frac{1}{6}(\bm{k}_1 +\bm{k}_2 + \bm{k}_3 + \bm{k}_4) \\
    \bm{k}_i &= \bm{f}(\bm{q}^{(i)},t_k+c_i\Delta t) \\
    \bm{q}^{(i)} &= \bm{q}_k &&\text{para} &&&i=1 \\
    \bm{q}^{(i)} &= \bm{q}_k + \Delta t\sum_{j=1}^{i-1}a_{ij}\bm{k}_j &&\text{para} &&&i>1
\end{align}
donde todos los coeficientes necesarios, $c_i$ y $a_{ij}$ son
\begin{align*}
    c_1 &= 0,\ c_2 = \frac{1}{2},\ c_3 = \frac{1}{2},\ c_4 = 1 \\
    &a_{21} = \frac{1}{2},\ a_{31} = 0,\ a_{41} = 0, \\
    &a_{32} = \frac{1}{2},\ a_{42} = 0,\ a_{43} = 1
\end{align*}

Finalmente, en cada paso, es necesario normalizar el cuaternión $(k+1)$-ésimo, ya que puede derivar de la longitud de la unidad
\begin{equation}
    \bm{q}_{k+1} \rightarrow \frac{\bm{q}_{k+1}}{||\bm{q}_{k+1}||}
\end{equation}

        \section{Conclusiones}
\label{sec:5_concl}
\ifimagenes
En el trabajo presentado se logró en primera instancia realizar las comprobaciones prácticas en una cámara estereo para luego realizar un algoritmo de SLAM en base a los datos de una cámara RGB-D, con la capacidad de poder reconstruir un mapa símil al real, y obteniendo las poses relativas de cada iteración. Si bien es cierto que se realizaron las mediciones en base a simulaciones, el hecho de haber incorporado ROS permite que el código pueda adaptarse fácilmente no solo a un entorno real, sino también a cualquier robot, permitiendo entonces una gran flexibilidad a la hora de utilizarlo. 

Respecto a los tiempos de procesamiento del algoritmo, los mismos son suficientes para el movimiento de robots en velocidades moderadas, en cambio si se pretende utilizar al mismo a una gran velocidad es posible que no pueda obtener los resultados esperados, debido a que el algoritmo espera que la nube de puntos anterior no esté muy alejada de la nueva nube de puntos. 

\section{Trabajos futuros}
Este trabajo forma parte de un trabajo final de carrera en desarrollo, el cual pretende realizar un SLAM 2D y 3D aprovechando otros sensores para obtener el mapa final, en particular, una IMU y un LIDAR 2D. Por dicha razón es que el documente presentado es adecuado para dicha tarea ya que, por ejemplo, con los datos de un LIDAR 2D a la hora de correr el algoritmo en tiempo real, la pose anterior será la estimada por el LIDAR, haciendo que el punto de partida no esté muy lejos del resultado final, evitando así errores por no encontrar el mínimo global.

Como este proyecto se engloba dentro de un problema de SLAM con fusión sensorial entre la cámara, LIDAR 2D e IMU, se pretende que el algoritmo mejore considerablemente a la hora de realizar dicha fusión.
\else
En el trabajo presentado se logró, en primera instancia, diseñar una plataforma versátil para aplicaciones robóticas en general, ya sea para robots terrestres como para robots aéreos en primera instancia, gracias a los sensores que presenta la misma por defecto. A su vez, si se pretende extender a la misma a otras aplicaciones (incluso si se quiere agregar redundancia de sensores, por ejemplo), la misma cuenta con el factor de forma PC104, permitiendo así extender sus funcionalidades en caso de que sea necesario bajo un estándar conocido en la industria.

Luego, en base a los sensores seleccionados anteriormente, se realizaron distintas calibraciones de los mismos. Cabe destacar que, si bien los datos provistos de la calibración de la IMU no parece ser en primera instancia numéricamente significativa, el hecho de que se integre dos veces el valor de la aceleración por cada muestra del sensor hace que, si el dato no es lo mas próximo a la realidad, hará que aumente el error de dicha estimación de posición y, en consecuente, diverja al poco tiempo. Si bien el hecho de realizar la estimación de la posición en base a los datos de la IMU únicamente es una tarea muy compleja, la correcta calibración de la misma permitiría una mejor estimación de la pose en una futura fusión sensorial.

A continuación, se realizaron las comprobaciones prácticas en una cámara estéreo, desde la adquisición de los datos hasta la obtención de la nube de puntos arrojada por la misma.

Como no se dispuso de una forma de corroborar las poses obtenidas en cada instante mediante los datos reales, se optó por el uso de simulaciones mediante Gazebo, utilizando el modelo del robot ROSbot 2.0, ya que cuenta con una cámara RGB-D, un LIDAR 2D y una IMU, entre otros sensores.

Luego, se realizó un algoritmo de SLAM 3D en base a los datos de la cámara RGB-D y, utilizando los datos obtenidos de las simulaciones, se pudo reconstruir un mapa símil al real, obteniendo las poses relativas de cada iteración.

A su vez, con el uso de un LIDAR 2D ayudado por los datos de una IMU se consiguió resolver el problema de SLAM 2D, con tiempos considerables como para que el mismo pueda implementarse en tiempo real.

Si bien es cierto que se realizaron las mediciones en base a simulaciones, el hecho de haber incorporado ROS permite que el código pueda adaptarse fácilmente no solo a un entorno real, sino también a cualquier robot, permitiendo entonces una gran flexibilidad a la hora de utilizarlo.

Si bien este trabajo pretende realizar mejoras, las cuales se tratarán en la Sección \ref{sec:futureworks}, el código fuente realizado se encuentra disponible en el repositorio \textit{ramon\_slam}\footnote{\url{https://github.com/frand08/ramon_slam}} a todo aquel que pretenda incursionar ya sea con SLAM, calibraciones, simulaciones o la implementación de ROS en microcontroladores, entre otras características realizadas. En el mismo también se encuentra una guía de instalación de los distintos módulos a todo aquel que quiera utilizar el código.

%Respecto a los tiempos de procesamiento del algoritmo, los mismos son suficientes para el movimiento de robots en velocidades moderadas, en cambio si se pretende utilizar al mismo a una gran velocidad es posible que no pueda obtener los resultados esperados, debido a que el algoritmo espera que la nube de puntos anterior no esté muy alejada de la nueva nube de puntos. 

\fi
        \bibliographystyle{unsrt}
        \bibliography{thebibliography}
    \else
        \centering
        \begin{figure}[t]
        	\centering
        	\includegraphics[scale=0.15]{utn.jpg}
            \vspace{0.5cm}
        \end{figure}%
    \fi
\fi

\ifimagenes
	{\LARGE Procesamiento digital de imágenes\par}
    {\LARGE 2020\par}
	\vspace{1cm}
	{\huge\bfseries SLAM 3D basado en cámara RGB-D y PCL\par}
\else
	{\LARGE Proyecto Final\par}
    {\LARGE 2020\par}
	\vspace{1cm}
	{\huge\bfseries Robots con detección de entorno\par}
	\vspace{1cm}
    {\LARGE Trabajo final de carrera\par}
\fi
    \vspace{1cm}
	{\Large\itshape Domínguez, Francisco\par}
	\vfill
\end{titlepage}

\tableofcontents

\newpage
\listoffigures
\ifimagenes
\else
    \newpage
    \listoftables
    \newpage
    \section*{Objeto del Documento}
\label{sec:0_objeto}
El objeto del presente es hacer un análisis exhaustivo del estado de las plataformas de Hardware presentes en el mercado actual, de manera de diseñar a partir de dicho análisis una plataforma versátil para el usuario que permita extender el rango de aplicaciones de las plataformas ya existentes, permitiendo así, por ejemplo, el montaje de sensores adicionales a la placa. En base a dicho diseño, se implementa una solución al problema de la localización y mapeo simultáneos (SLAM).
\fi

\newpage
\ifimagenes
\ifimagenespaper
\begin{abstract}
\else
\section*{Abstract}
\fi
Para un robot móvil que explora un entorno estático desconocido, localizarse y construir un mapa al mismo tiempo es el problema conocido como localización y mapeo simultáneos (SLAM). Con el fin de poder resolver este problema, en el presente trabajo se desarrolla un enfoque de la odometría visual a partir de imágenes en colores con mapa de profundidad (RGB-D) de una cámara Microsoft Kinect bajo el entorno de simulación Gazebo. Con este fin, se propone un \textit{pipeline de registro} que tiene como objetivo encontrar la mejor estimación de movimiento de cuerpo rígido para mapear una imagen de profundidad en otra, asumiendo una escena estática tomada por una cámara en movimiento. 

El pipeline propuesto se basa en nubes de puntos organizadas, esto es, que dichas nubes se presenten como matrices 2D, tal como la estructura que presentan las cámara monoculares. Aprovechando esto, se emplea una técnica la cual asegura que la nube de puntos resultante tenga un número reducido de muestras extraídas en base a las distintas orientaciones superficiales que presenta la nube, llamada \textit{normal space sampling}, aumentando la probabilidad de que el registro converja al mínimo global. Los resultados obtenidos se asemejan a la trayectoria real simulada por el robot.
\ifimagenespaper
\end{abstract}

\begin{IEEEkeywords}
RGB-D, PCL, SLAM3D, SLAM, Gazebo
\end{IEEEkeywords}
\else
\fi

\else
\section*{Abstract}
\label{sec:1_abstract}

En los últimos años, los robots han recibido una mayor atención de la comunidad de investigación, en especial debido a la aparición de los vehículos aéreos no tripulados.

El presente trabajo describe el desarrollo de una plataforma de control de robots. Dicha plataforma cuenta con la posibilidad de expandir sus funcionalidades gracias al estándar de hardware PC/104 presente en la industria, logrando así poder desarrollar a partir de la misma tanto vehículos aéreos no tripulados (UAV) como robots terrestres, entre otros. En base a la misma, se desarrollará un robot capaz de estimar tanto su posición como el contorno del mapa en el cual se encuentra, problema conocido como el de localización y mapeo simultáneos. 

Como se utilizan en el mismo distintos sensores, en primera instancia se realiza la calibración de gran parte de los mismos, para luego utilizarlos en la generación de dos mapas: uno en dos dimensiones, basado en la IMU y el LIDAR; y el otro tridimensional, basado en la cámara RGB-D.
\fi



\newpage
\section{Introducción}
\label{sec:2_marcoteorico}
Si bien el uso de robots tiene sus orígenes principalmente en fábricas para el ensamblaje de automóviles, en los últimos años la electrónica moderna ha permitido introducirlos en otras áreas, ya sea desde electrodomésticos y juguetes de la vida cotidiana, hasta viajes espaciales con el fin de explorar planetas desconocidos. Esta expansión es debido a que los mismos permiten reducir la interacción humana no solo en tareas que presentan un riesgo a la integridad de la persona, sino también en aquellas que tienen cierto grado de repetitividad.

Yendo a un caso más concreto, en el último tiempo muchas personas han adquirido los llamados robots aspiradoras, los cuales consiguen limpiar la superficie de las casas en un tiempo medianamente razonable, aunque este tiempo no suele ser una preocupación ya que al ser el mismo completamente autónomo, uno puede seguir con sus actividades cotidianas. Ahora bien, si se analiza el recorrido que realizan la mayoría de estos robots, se puede apreciar que el mismo es completamente aleatorio, y por ende se tendería a creer que no podrá pasar por toda la superficie y limpiarla completamente. La realidad es que, como están mucho tiempo circulando, logran pasar por todos lados, pero esto genera que circulen más veces por unos lugares que por otros, haciendo ineficiente el trabajo, tal como se observa en la Figura \ref{fig:vaccumrobot}.

\begin{figure}%
    \centering
    \subfloat{{\includegraphics[width=0.47\textwidth]{Img/Roomba}}}%
    \qquad
    \subfloat{{\includegraphics[width=0.47\textwidth]{Img/VaccumRobot.png}}}%
    \caption{Ejemplo del recorrido de un robot aspiradora comercial}
    \label{fig:vaccumrobot}
\end{figure}

A pesar de que se busque reducir la interacción humana bajo estas circunstancias, no todos los robots hoy en día son autónomos, principalmente debido a la falta de robustez del algoritmo de control involucrado en el proceso. Tomando por caso el ejemplo anterior, si bien el robot aspiradora es autónomo, su eficiencia respecto a la forma óptima de realizar la tarea no es una garantía en todos los casos.

Es útil distinguir entre robots que están inmóviles, como un brazo robótico en una fábrica, y robots que son móviles, como un auto sin conductor. En este trabajo se hará hincapié en los robots móviles. Usamos este término para describir un robot impulsado por sus propios medios que puede moverse cinemáticamente entre ubicaciones en su entorno. Cuando se habla de la posición y orientación combinadas del robot, esto se define como su \textit{pose}.

Los robots móviles pueden referirse a robots que se mueven sobre el suelo, bajo el agua, a través del aire y en entornos de microgravedad, tales como los que pueden observarse en la Figura \ref{fig:mobilerobots}. Si bien este trabajo busca aplicar en parte a cualquiera de estos entornos, el enfoque del mismo se refiere principalmente a los robots móviles que permanecen en contacto con el suelo. El término vehículos terrestres no tripulados (con sus siglas en inglés, UGV) a menudo se usa más específicamente para describir robots móviles terrestres, mientras que el término vehículos aéreos no tripulados (del inglés, UAV) refieren principalmente a los drones.

Los robots móviles se mueven a través de entornos grandes y potencialmente dinámicos, lo que hace que la percepción sea mucho más difícil que los robots industriales con entornos de trabajo limitados y parámetros operativos rígidos. Los robots móviles requieren sensores adicionales, una mejor percepción y mayores grados de autonomía para operar en entornos del mundo real que cambian con frecuencia.


\begin{figure}%
    \centering
    \subfloat[Wheeled robot]{{\includegraphics[width=0.25\textwidth]{Img/WheeledRobot.jpeg}}}%
    \qquad
    \subfloat[Legged robot]{{\includegraphics[width=0.25\textwidth]{Img/LeggedRobot.jpeg}}}%
    \qquad
    \subfloat[Tracked robot]{{\includegraphics[width=0.25\textwidth]{Img/TrackedRobot.jpeg}}}%
    \qquad
    \subfloat[Drone]{{\includegraphics[width=0.3\textwidth]{Img/Drone.jpeg}}}%
    \qquad
    \subfloat[Water-based robot]{{\includegraphics[width=0.5\textwidth]{Img/WaterBasedRobot.jpg}}}%
    \caption{Tipos de robot móviles}
    \label{fig:mobilerobots}
\end{figure}

Dentro de lo que respecta a robótica móvil, uno de sus desafíos actuales es lograr la completa autonomía del vehículo. Una gran parte de las empresas de la industria automotriz han dedicado mucho tiempo y dinero para lograr el objetivo. La investigación ha avanzado hasta el punto de que en la actualidad hay autos semi-autónomos disponibles en el mercado, y se cree ampliamente que en el futuro cercano casi todos los vehículos serán completamente autónomos. Para lograr esto, obviamente hay una necesidad de sensores y funciones avanzadas. Uno de los principales problemas es el hecho de que el vehículo debe ser consciente de su posición actual dentro de un entorno desconocido. Esto se conoce como la localización y mapeo simultáneos, comúnmente abreviado como SLAM.

El SLAM es un problema difícil de solucionar, así como lo fue para los humanos en el pasado. Tiempo atrás, los marineros usaban los llamados registros de fichas para estimar su velocidad y extrapolaban esta velocidad con el tiempo para calcular su posición u orientación. Este proceso de navegación por estima (en inglés, \textit{dead reckoning}) conduce inevitablemente a errores graves de posicionamiento con el tiempo y, por lo tanto, siempre se ha respaldado mediante el uso de puntos de referencia (o \textit{landmarks}) para la navegación. Quedándonos en este caso particular, con el uso de las estrellas los marineros lograban saber donde estaba el Norte, permitiendo corregir su dirección. Sin embargo, no siempre se podía observar este patrón por días debido al cielo nublado, entonces tenían que confiar solamente en su navegación por estima, haciendo que la incertidumbre crezca día tras día. Para el caso de los robots, cuando el mismo recorre un  mapa completamente desconocido, acumulará error debido a las predicciones que debe tomar respecto a su pose y entorno en el que está, hasta el momento en que entra a un área con puntos de referencia conocidos, puediendo corregir su estimación de posición. Esto es lo denominado cierre de ciclo (o \textit{loop closure}), como puede verse en la Figura \ref{fig:poseloopclcorr}.

\begin{figure}
    \centering
    \includegraphics[width=\textwidth]{Img/Pose_LoopClosureCorr.png}
    \caption{Estimación de pose y su correción mediante el loop closure}
    \label{fig:poseloopclcorr}
\end{figure}

Volviendo al ejemplo del robot de limpieza, si el mismo conociera el mapa en el que se encuentra, podría entonces trazar la ruta mas óptima para realizar la limpieza del lugar. Como no todos los ambientes son iguales y cada uno tiene, por ejemplo, muebles ubicados en distintos lugares, dicho robot podría hacer primero un reconocimiento del lugar previo a la limpieza para tener una noción de todo el espacio, o bien obtener el mapa y la ubicación en la que se encuentra en base al movimiento que realiza, pasando por los lugares que le faltaron aspirar. Para cualquiera de las dos opciones, por lo tanto, necesitará tanto de tomar los datos del lugar como de controlar al vehículo para que vaya circulando por el ambiente.

La estimación de la pose, entonces, es una parte no separable de aplicaciones como el control del vehículo y el mapeo. Varios sensores se utilizan comúnmente para estimar la pose de un robot. Las unidades de medición inercial (IMU), la cámara, la odometría de las ruedas\footnote{Se mide la rotación de las ruedas para estimar cambios de la posición del vehículo a lo largo del tiempo} para el caso de ciertos robots móviles  y el LiDAR se encuentran entre los sensores más populares para la localización en interiores \cite{delrosario2016}, mientras que para exteriores suele sumarse el GPS a estos sensores, corrigiendo al mismo mediante acción de la IMU \cite{caron2006}. Entre estos sensores, el LiDAR ha recibido menos atención para la estimación de la pose. Esto se debe a que los LiDAR bidimensionales no pueden ofrecer una estimación completa de la pose\footnote{De lo que refiere a la orientación y posicion en tres dimensiones} y los LiDAR tridimensionales son voluminosos y costosos.

En la localización al aire libre, la señal del sistema de posicionamiento global (GPS) suele estar disponible y, dado que la posición se puede obtener con precisión mediante GPS, es la opción común para la fusión con los datos obtenidos de la IMU \cite{engel2014}. Sin embargo, en un entorno denegado por GPS, como dentro de un edificio, se deben usar otros sensores para corregir las estimaciones de IMU. Un enfoque común para abordar este problema es la fusión IMU-cámara \cite{mirzaei2008}, \cite{hesch2009}, \cite{chambers2014}, \cite{hesch2013}, y en ciertos casos combinadas con LiDAR \cite{lee2016}.

Un algoritmo SLAM robusto es esencial para que cualquier robot móvil navegue de manera segura a través de un entorno no estructurado. El algoritmo SLAM con frecuencia define un marco de coordenadas global para que el robot opere; uno que generalmente es utilizado por todas las funcionalidades de alto nivel, como navegación, planificación de rutas, exploración, identificación de objetos, seguimiento de objetos y manipulación de los mismos. Estas dependencias hacen que el algoritmo SLAM sea una parte central de cualquier arquitectura de robot móvil, y las fallas irrecuperables son altamente indeseables. Es probable que cualquier robot desorientado por una falla de SLAM no pueda realizar su tarea, o peor, puede poner en peligro a los humanos, a sí mismo o al medio ambiente.

Si bien se han demostrado varios algoritmos SLAM en laboratorios, es más difícil producir soluciones robustas en entornos no estructurados del mundo real. El deslizamiento de las ruedas, por ejemplo, puede producir mediciones de odometría ruidosas y sesgadas, mientras que los sensores del sistema de posicionamiento global (GPS) a menudo producen una localización global muy ruidosa y sesgada. El SLAM robusto del mundo real se vuelve aún más difícil cuando estos errores de detección aleatorios y sistemáticos ocurren en entornos que tienen geometrías redundantes u objetos en movimiento.

\newpage
\section{ROS}

Robot Operating System (ROS) es un \textit{framework} open source pensado para escribir software orientado a robots. Este software está estructurado como un gran número de pequeños programas que se pasan rápidamente los mensajes entre sí. Este paradigma fue elegido para fomentar la reutilización de software de robótica fuera del robot y del entorno en particular que impulsó su creación. Es por ello que cuenta con librerías, herramientas y convenciones que buscan facilitar la tarea de realizar comportamientos robustos y complejos de casi cualquier tipo de robot existente.

\ifimagenes
\else
    \newpage
    \section{Geometría tridimensional}
En la presente sección se presentará la geometría tridimensional y específicamente los conceptos de rotación, traslación y algunas de sus representaciones. Presta especial atención al establecimiento de marcos de referencia. {\big Sastry (1999)} es una referencia integral sobre el control de la robótica que incluye un trasfondo sobre geometría tridimensional. {\big Hughes (1986)} también proporciona una buena base de primeros principios.

Los robots móviles generalmente son libres de trasladar y rotar. Matemáticamente, tienen seis grados de libertad: tres en traslación y tres en rotación. Esta configuración geométrica de seis grados de libertad se conoce como la \textit{pose} (posición y orientación) del robot. Algunos robots pueden tener múltiples cuerpos conectados entre sí; en este caso cada cuerpo tiene su propia pose. Consideraremos solo el caso de un solo cuerpo aquí.

\subsection{Marco de referencia}


    \newpage
    % Regression analysis o state estimation?
\section{Análisis de regresión}
\label{sec:regressionanalysis}
El análisis de regresión es un análisis estadístico para predecir el valor de una variable cuantitativa continua. Basándose en un conjunto de variables independientes, se busca estimar la relación de las mismas con una variable dependiente, mediante la obtención de una curva que mejor se ajuste a los datos disponibles, sin que necesariamente pase por todos ellos. En concreto, el modelo de regresión puede representarse como
\begin{equation}
    y_i = f(\bm{x}_i; \bm{\beta}) + v_i,
    \label{eq:regressionmodel}
\end{equation}
donde la variable $y_i$ corresponde a la respuesta o medición para el caso $i$, con $i = 1, 2, ..., m$, $\bm{x}_i = (x_{i1}, x_{i2}, ..., x_{in})$ corresponde al conjunto de valores, fijos o aleatorios, utilizados para explicar o predecir el comportamiento de $y_i$, conocidas como las \textit{variables regresivas} o \textit{independientes}, $\bm{\beta} = (\beta_1, \beta_2, ..., \beta_n)$ a los parámetros desconocidos\footnote{A diferencia de un estado, el que se define como una magnitud física que varía a través del tiempo, el parámetro es constante a través del tiempo.} a estimar, y $v_i$ a la componente de ruido propia para ese caso.

En base a esto, lo que se busca es estimar la función $f$ que mejor se ajuste a los datos, conocida como \textit{función de expectativa} para el modelo de regresión. Para ello, primero lo que debe hacerse es determinar la forma de dicha función que, en base a esta, se puede clasificar en \textit{regresión lineal} y \textit{regresión no lineal}.

Existen diferentes métodos dentro del análisis de regresión para poder estimar los parámetros desconocidos $\beta$, por lo que en las siguientes subsecciones se presentarán algunos de ellos, los cuales son de relevancia para el seguimiento de la tesis.

% En cambio, cuando la función que modela al sistema \textit{no puede expresarse como una combinación lineal de los parámetros desconocidos} $\bm{\beta}$, se trata entonces de una regresión no lineal, y debe realizarse un proceso iterativo para encontrar la curva que mejor se adapte al sistema.

% La información de tales relaciones se efectúa a partir de información muestral acerca de los valores tomados por $y$, $x_1$, $x_2$, $x_3$, ..., $x_n$, y trata de cuantificar la magnitud de la dependencia entre ellas.

\subsection{Regresión lineal}
En regresión lineal, la variable $y$ \textit{es una combinación lineal} de los parámetros, es decir,
\begin{align}
    y_i &= \beta_1 x_{i1} + \beta_2 x_{i2} + \beta_3 x_{i3} + ... + \beta_n x_{in} + v_i \\
      &= \bm{x}_i \bm{\beta} + v_i
\end{align}

\subsubsection{Método de cuadrados mínimos ordinario (OLS)}
Con el fin de hallar los parámetros desconocidos, uno de los métodos más utilizados corresponde al \textit{método de cuadrados mínimos}, el cual plantea que el valor más probable de dichos parámetros será aquel que minimiza la suma de los errores cuadráticos entre lo que se observa y lo que se espera. Este método es una variante especial del problema más general, en el que dada una función $f:\mathbb{R}^n\rightarrow\mathbb{R}$ se busca un argumento de $f$ que de el mínimo valor de la llamada \textit{función de coste}
\begin{equation}
    \hat{x} = argmin_x f(x)
    \label{eq:globalminimizer}
\end{equation}
llamado tambien como el \textit{minimzador global}.

Si se quiere, por ejemplo, medir el peso de una bolsa de naranjas, cuyo valor real es $x$, con el uso de una balanza de poca precisión, el valor medido $y$ puede modelarse como el valor real corrompido por ruido, $v$, linealmente mediante la ecuación
\begin{equation}
    y = x + v
    \label{eq:linearmeasmodel}
\end{equation}

Para cada una de las mediciones, se define un término escalar de ruido que es independiente de los otros términos de error ya que, para este caso, se asume que el ruido es independiente y de distribución uniforme. Ahora, si se define el error entre cada medición y el valor verdadero de la bolsa de naranjas, se obtiene el denominado \textit{criterio de error} de cada medición, esto es,
\begin{equation}
    e_i = y_i - x
\end{equation}

Con estos errores definidos, el método de cuadrados mínimos define que la mejor estimación del valor $x$ es la que minimiza el \textit{criterio de error cuadrático}
\begin{equation}
    \hat{x}_{OLS} = argmin_x(e_1^2+e_2^2+e_3^2+...+e_m^2) = \mathscr{L}_{OLS}(x),
    \label{eq:squarederrorcriterion}
\end{equation}

% Luego de un set de cinco mediciones realizadas por separado y en forma secuencial, se obtienen los valores que se observan en la Tabla \ref{tab:naranjasbalanza}, junto con sus modelos de medición y errores cuadráticos.

% \begin{table}[]
\centering
\begin{tabular}{c|c|c|c}
\textbf{Medición} & \textbf{Peso {[}g{]}} & \textbf{Modelo de medición} & \textbf{Error} \\ \hline
1                       & 1012                  & $y_1 = x + v_1$        & $e_1 = y_1 - x$      \\ \hline
2                       & 989                   & $y_2 = x + v_2$        & $e_2 = y_2 - x$      \\ \hline
3                       & 1008                  & $y_3 = x + v_3$        & $e_3 = y_3 - x$      \\ \hline
4                       & 1030                  & $y_4 = x + v_4$        & $e_4 = y_4 - x$      \\ \hline
5                       & 971                   & $y_5 = x + v_5$        & $e_5 = y_5 - x$           
\end{tabular}
\caption{Peso de una bolsa de naranjas para cada medición realizada}
\label{tab:naranjasbalanza}
\end{table}

Para poder minimizar la función de coste, suponiendo que fueron tomadas cinco mediciones realizadas por separado y en forma secuencial, primero hay que reescribir a la función de error en su forma matricial, siendo entonces
\begin{align}
    \bm{e} &= \bm{y} - \bm{H}\cdot x \\
    \begin{bmatrix}
        e_1\\ e_2\\ e_3\\ e_4\\ e_5
    \end{bmatrix}
    &= 
    \begin{bmatrix}
        y_1\\ y_2\\ y_3\\ y_4\\ y_5
    \end{bmatrix}
    -
    \begin{bmatrix}
        1\\ 1\\ 1\\ 1\\ 1
    \end{bmatrix}
    x
\end{align}
con $\bm{H}$ la \textit{matriz Jacobiana} que, para este caso particular, posee valores unitarios. Dicha matriz tiene las dimensiones de $m\times n$, donde \textit{m} es el número de mediciones y \textit{n} es el número de parámetros que se desean estimar. Por ello, \textit{x} si bien en este caso es un escalar, puede ser un vector en el caso que se tengan múltiples indeterminaciones. En base a esto, se puede redefinir a la función de coste como

\begin{align}
    \mathscr{L}_{OLS}(x) &= \bm{e}^T\bm{e} \\
                        &= (\bm{y} - \bm{H}x)^T(\bm{y} - \bm{H}x) \\
                        &= \bm{y}^T\bm{y} - x^T\bm{H}^T\bm{y} - \bm{y}^T\bm{H}x + x^T\bm{H}^T\bm{H}x
\end{align}

Para minimizar esta ecuación, se procede a computar la derivada parcial de la función de coste respecto a la incertidumbre $x$ para luego igualarla a cero.
\begin{align}
    \frac{\partial \mathscr{L}}{\partial x}\bigg\rvert_{x=\hat{x}} = -\bm{y}^T\bm{H} - \bm{y}^T\bm{H} + 2\hat{x}^T\bm{H}^T\bm{H} &= 0 \\
    -2\bm{y}^T\bm{H} + 2\hat{x}^T\bm{H}^T\bm{H} &= 0
\end{align}

Despejando, se llega al valor del peso de la bolsa de naranjas que minimiza el criterio de error cuadrático
\begin{equation}
    \hat{x}_{OLS} = (\bm{H}^T\bm{H})^{-1}\bm{H}^T\bm{y}
\end{equation}

Esta expresión tiene solución si y solo si existe $(\bm{H}^T\bm{H})^{-1}$, o sea, si la matriz tiene inversa. Si tenemos \textit{m} mediciones y \textit{n} parámetros,
\begin{align*}
    \bm{H} &\in \Re^{m\times n} \\
    \bm{H}^T\bm{H} &\in \Re^{n\times n}
\end{align*}

Por lo tanto, para que $(\bm{H}^T\bm{H})^{-1}$ exista es necesario que se dispongan mínimamente de la misma cantidad de mediciones que variables a estimar, esto es
\begin{equation*}
    m \geq n
\end{equation*}

\subsubsection{Método de cuadrados mínimos ponderado (WLS)}

Volviendo al ejemplo de la bolsa de naranjas, si se toman mediciones con distintas balanzas, es lógico pensar que mientras mayor sea la precisión de cada una, mayor importancia tendrá su valor indicado para determinar el peso de la bolsa. Si se considera al modelo de medición lineal con \textit{m} mediciones y \textit{n} incertidumbres,
\begin{align}
    \begin{bmatrix}
        y_1 \\ . \\ . \\ . \\ y_m
    \end{bmatrix}
    &=
    \bm{H}
    \begin{bmatrix}
        x_1 \\ . \\ . \\ . \\ x_n
    \end{bmatrix}
    +
    \begin{bmatrix}
        v_1 \\ .\\ . \\ . \\ v_m
    \end{bmatrix} \\
    \bm{y} &= \bm{H} \bm{x} + \bm{v}
\end{align}

En cuadrados mínimos ordinarios, se asume implícitamente que cada término de error, $v_i$, posee el mismo desvío estándar, $\sigma$. En cambio, si se toma que cada término de error es independiente pero con distinto desvío estándar,
\begin{equation}
    \mathbb{E}_{[v_i^2]} = \sigma_i^2 \hspace{0.5cm}(i=1,...,m)\hspace{1cm}\bm{R}=\mathbb{E}_{[\bm{v}\bm{v}^T]} = 
    \begin{bmatrix}
        \sigma_1^2  &        &     0      \\
                    & \ddots &            \\
            0       &        & \sigma_m^2
    \end{bmatrix}
\end{equation}
se puede definir a partir de esto el \textit{criterio de cuadrados mínimos ponderado} como
\begin{align}
    \mathscr{L}_{WLS}(x) &= \bm{e}^T\bm{R}^{-1}\bm{e} \\
                         &= \frac{e_1^2}{\sigma_1^2} + \frac{e_2^2}{\sigma_2^2} + ... + \frac{e_m^2}{\sigma_m^2}
\end{align}

Mientras mayor sea el ruido esperado, menor será el peso que tenga en la medición. En el caso que todos los desvíos sean iguales, \textit{no afecta el valor estimado final}, ya que pasa a ser una constante dividiendo a todos los errores.

Expandiendo el nuevo criterio,
\begin{align}
    \mathscr{L}_{WLS}(x) &= \bm{e}^T\bm{R}^{-1}\bm{e} \\
                         &= (\bm{y} - \bm{H}\bm{x})^T\bm{R}^{-1}(\bm{y}-\bm{H}\bm{x})
\end{align}

Como en el caso de los cuadrados mínimos ordinarios, la ecuación MONGO puede minimizarse realizando el gradiente en este caso al ser que se cuentan con \textit{n} incógnitas.
\begin{equation}
    \frac{\partial \mathscr{L}}{\partial \bm{x}}\bigg\rvert_{\bm{x}=\hat{\bm{x}}} = -\bm{y}^T\bm{R}^{-1}\bm{H} + \hat{\bm{x}}^T\bm{H}^T\bm{R}^{-1}\bm{H} = 0
\end{equation}

obteniendo entonces las ecuaciones normales ponderadas
\begin{equation}
    \hat{\bm{x}}_{WLS} = (\bm{H}^T\bm{R}^{-1}\bm{H})^{-1} \bm{H}^T\bm{R}^{-1}\bm{y}
\end{equation}

\subsubsection{Método de cuadrados mínimos recursivo (RLS)}

Hasta el momento, todos los datos de las mediciones se encontraban disponibles. Ahora, si lo que se tiene es, por ejemplo, un sensor que entrega datos cada cierto tiempo, con el razonamiento utilizado hasta el momento se tendería a pensar que es necesario correr alguno de los métodos vistos cada vez que llega un dato nuevo con todo el set completo. Se puede apreciar que claramente esto produciría un aumento de costo computacional a medida que transcurre el tiempo, haciendo que en un punto llegue a ser un problema.

Para evitar este problema, lo que se busca es un método recursivo que mantenga un estimador actual del parámetro óptimo para todas las mediciones que se han recolectado hasta ese momento, y luego actualizar dicho estimador dada la medición en el intervalo de tiempo actual. Para lograr esto, se puede utilizar entonces un \textit{estimador lineal recursivo}.

Suponiendo que se tiene un estimador óptimo $\hat{\bm{x}}_{k-1}$ de los parámetros desconocidos en el tiempo $k-1$ con ruido blanco Gaussiano aditivo, al llegar la nueva medición en el tiempo $k$,
\begin{equation}
    \bm{y}_k = \bm{H}_k \bm{x} + \bm{v}_k
\end{equation}

Por lo tanto, el objetivo es poder computar $\hat{\bm{x}}_k$ como una función de $\bm{y}_k$ y $\hat{\bm{x}}_{k-1}$.

Un estimador lineal recursivo está dado por
\begin{equation}
    \hat{\bm{x}}_k = \hat{\bm{x}}_{k-1} + \bm{K}_k (\bm{y}_k - \bm{H}_k \hat{\bm{x}}_{k-1})
\end{equation}
donde $\bm{K}_k$ es una matriz de ganancia del estimador, el término entre paréntesis se denomina la innovación, y cuantifica que tan bien la medición actual se equipara con el mejor estimador previo. Aún sin conocer la matriz $\bm{K}_k$, en dicha ecuación puede observarse que el nuevo estimador es la suma del estimador anterior y el término correctivo basado en la diferencia entre lo esperado de la medición y lo que acutalmente se midió. Si la innovación fuese igual a cero, se mantendría el estimador anterior.

Finalmente, para el término $\bm{K}_k$, lo que se busca es minimizar el \textit{valor esperado de la suma de errores cuadráticos} del estimador actual en el tiempo \textit{k}. Para un solo parámetro escalar,
\begin{align}
    \mathscr{L}_{RLS}(x) &= \mathbb{E}_{[(x_k - \hat{x}_k)^2]} \\
                         &= \sigma_k^2
\end{align}

En cambio, si se tienen \textit{n} parámetros desconocidos en el tiempo \textit{k}
\begin{align}
    \mathscr{L}_{RLS}(x) &= \mathbb{E}_{[(x_{1k} - \hat{x}_{1k})^2 + ... + (x_{nk} - \hat{x}_{nk})^2]} \\
                         &= tr(\bm{P}_k)
\end{align}

siendo $tr()$ la \textit{traza} (o \textit{trace})\footnote{Corresponde a la suma de los elementos de la diagonal principal de la matriz} de la matriz de covarianza $\bm{P}_k$. En lugar de minimizar directamente el error, lo que se minimiza es su valor esperado, el cual es la varianza del estimador. A menor varianza, mayor será el grado de confianza del estimador.

Al igual que antes, es posible expresar a la covarianza en función de $\bm{K}_k$ utilizando la formulación lineal recursiva
\begin{equation}
    \bm{P}_k = (\bm{1} - \bm{K}_k\bm{H}_k)\bm{P}_{k-1}(\bm{1}-\bm{K}_k\bm{H}_k)^T + \bm{K}_k\bm{R}_k\bm{K}_k^T
\end{equation}

Mediante el uso de cálculo matricial y derivadas parciales, se puede llegar a que este término se minimiza cuando
\begin{equation}
    \bm{K}_k = \bm{P}_{k-1}\bm{H}_k^T(\bm{H}_k\bm{P}_{k-1}\bm{H}_k^T + \bm{R}_k)^{-1}
\end{equation}

En base a esto, es posible reescribir la ecuación de la matriz de covarianza como
\begin{align}
    \bm{P}_k &= \bm{P}_{k-1} - \bm{K}_k\bm{H}_k\bm{P}_{k-1} \\
                 &= (\bm{1} - \bm{K}_k\bm{H}_k)\bm{P}_{k-1}
\end{align}

De este último término se puede observar que cuanto más grande sea la matriz de ganancia $\bm{K}$, será más pequeña la nueva covarianza del estimador. Por lo tanto, se puede decir que la covarianza \textit{se encoge} con cada medición.

Resumiendo, el algoritmo de los cuadrados mínimos recursivos consta de tres pasos
\begin{enumerate}
    \item Inicializar los parámetros desconocidos y la matriz de covarianza. Esta predicción inicial, por ejemplo, puede obtenerse a partir de la primer medición que se toma y la covarianza puede venir de especificaciones técnicas.
        \begin{align}
            \hat{\bm{x}}_0 &= \mathbb{E}_{[\bm{x}]} \\
            \bm{P}_0 &= \mathbb{E}_{[(\bm{x} - \hat{\bm{x}}_0)(\bm{x} - \hat{\bm{x}}_0)^T]}
        \end{align}
    \item Definir el Jacobiano y la matriz de covarianza de la medición
        \begin{equation}
          \bm{y}_k = \bm{H}_k\bm{x}+\bm{v}_k
        \end{equation}
    \item Actualizar el valor estimado de $\hat{\bm{x}}_k$ y la covarianza $\bm{P}_k$
        \begin{align}
            \bm{K}_k &= \bm{P}_{k-1}\bm{H}_k^T(\bm{H}_k\bm{P}_{k-1}\bm{H}_k^T + \bm{R}_k)^{-1} \\
            \hat{\bm{x}}_k &= \hat{\bm{x}}_{k-1} + \bm{K}_k(\bm{y}_k - \bm{H}_k\hat{\bm{x}}_{k-1}) \\
            \bm{P}_k &= (\bm{1} - \bm{K}_k\bm{H}_k)\bm{P}_{k-1}
        \end{align}
\end{enumerate}

\subsubsection{Método de máxima verosimilitud (MLE)}

En lugar de escribir una pérdida tal como se realiza en cuadrados mínimos, puede aproximarse el problema de la estimación de parámetros óptimos al buscar qué parámetros hacen que las mediciones obtenidas sean las más probables. En otras palabras, cuál \textit{x} maximiza la probabilidad condicional de $\bm{y}$
\begin{equation}
    \hat{x} = argmax_x p(\bm{y}|x)
    \label{eq:maxlikelihood}
\end{equation}

Volviendo al ejemplo de la bolsa de naranjas, si se tienen cuatro posibles valores de dicho peso, como se observa en la Figura \ref{fig:mostlikelyproba}, el valor que maximiza la verosimilitud condicional dada la medición $y_{med}$ corresponde a $x_b$, ya que es la densidad de probabilidad mayor en la ubicación medida.
\begin{figure}[!h]
    \centering
    \includegraphics[width=\textwidth]{Img/MostLikelyProba.png}
    \caption{Posibles valores del peso de la bolsa de naranjas y su valor real $y_{med}$}
    \label{fig:mostlikelyproba}
\end{figure}

Tomando en cuenta el modelo de medición presente en la Ecuación (\ref{eq:linearmeasmodel}), el mismo puede ser expresado como una probabilidad condicional de la medición, asumiendo alguna densidad de probabilidad para $v$, por ejemplo, ruido blanco Gaussiano aditivo
\begin{equation}
    v \sim \mathcal{N}(0,\sigma^2)
\end{equation}
entonces, el parámetro desconocido, $x$, resulta ser la media de esta densidad, y la varianza corresponde a la varianza de ruido.
\begin{equation}
    p(y|x) = \mathcal{N}(x,\sigma^2)
\end{equation}

Teniendo en cuenta que la función densidad de probabilidad de una Gaussiana es
\begin{equation}
    \mathcal{N}(z;\mu,\sigma^2) = \frac{1}{\sigma \sqrt{2\pi}} e^{\frac{-(z-\mu)^2}{2\sigma^2}}
\end{equation}

puede expresarse la medición de verosimilitud para una de las mediciones como
\begin{align}
    p(y|x) &= \mathscr{N}(y;x,\sigma^2) \\
           &= \frac{1}{\sqrt{2\pi\sigma^2}} e^{\frac{-(y-x)^2}{2\sigma^2}}
\end{align}

Si se tienen múltiples mediciones independientes, entonces
\begin{align}
    p(\bm{y}|x) &\propto \mathscr{N}(y_1;x,\sigma^2)\cdot\mathscr{N}(y_2;x,\sigma^2)\cdot...\cdot\mathscr{N}(y_m;x,\sigma^2) \\
            &= \frac{1}{\sqrt{(2\pi)^m\sigma^{2m}}} \exp\left({\frac{-\sum_{i=1}^m(y_i-x)^2}{2\sigma^2}}\right)
\end{align}
y el estimador de máxima verosimilitud (MLE) será
\begin{equation}
    \hat{x}_{MLE} = argmax_x p(\bm{y}|x)
\end{equation}

En lugar de tratar de optimizar el verosímil directamente, puede aplicarse el logaritmo
\begin{align}
    \hat{x}_{MLE} = argmax_x \log p(\bm{y}|x)
    \label{eq:mlewithlog}
\end{align}
y, como la función logaritmo aumenta monotónicamente
\begin{equation}
    \log p(\bm{y}|x) = -\frac{1}{2\sigma^2}\left((y_1-x)^2+...+(y_m-x)^2\right)+C
\end{equation}

Esta expresión es similar a la suma de los errores cuadráticos, presentada en (\ref{eq:squarederrorcriterion}). La constante $C$ en esta expresión refiere a términos que no son funciones de $x$ y pueden ser ignorados.

Luego, como el argumento que maximiza la función $f$ es equivalente al negativo del argumento que minimiza dicha función, o sea,
\begin{equation}
    argmax_z f(z) = argmin_z \left(-f(z)\right)
\end{equation}
el problema de la máxima verosimilitud puede ser escrito como
\begin{align}
    \hat{x}_{MLE} &= argmin_x -\left(\log p(\bm{y}|x)\right) \\
                  &= argmin_x \frac{1}{2\sigma^2}\left((y_1-x)^2+...+(y_m-x)^2\right)
\end{align}

Por lo tanto, es posible realizar la maximización de la verosimilitud mediante una minimización de la suma de errores cuadráticos. Esto es válido ya que se asume que las mediciones se encuentran corrompidas por ruido blanco Gaussiano independiente aditivo de igual varianza.

Finalmente, si se asume que cada medición tiene una varianza distinta, se llega al mismo criterio que en cuadrados mínimos ponderados
\begin{equation}
    \hat{x}_{MLE} = argmin_x \frac{1}{2}\left(\frac{(y_1-x)^2}{\sigma_1^2}+...+\frac{(y_m-x)^2}{\sigma_m^2}\right) =\frac{1}{2} \bm{e}^T\bm{R}^{-1}\bm{e}
    \label{eq:mlewls}
\end{equation}

Resumiendo, en ambos casos, el estimador de máxima verosimilitud para ruido blanco Gaussiano es equivalente a las soluciones de los cuadrados mínimos ordinarios o ponderados.
\begin{equation}
    \hat{x}_{MLE} = \hat{x}_{OLS} = argmin_x\mathscr{L}_{OLS}(x) = argmin_x\mathscr{L}_{MLE}(x)
\end{equation}

Esto cobra importancia en sistemas realistas, los cuales presentan una gran cantidad de fuentes de ruido. Recordando el teorema central del límite, que establece que cuando se suman variables aleatorias independientes, su suma normalizada tenderá a una distribución normal, y teniendo en cuenta que el método de cuadrados mínimos es equivalente a calcular la máxima verosimilitud\footnote{Siempre asumiendo que se está en presencia de ruido blanco Gaussiano aditivo}, es posible calcular el mejor estimador de una forma computacionalmente sencilla.

Sin embargo, una consideración importante a tener en cuenta con el método de cuadrados mínimos es cuando se presenta un valor atípico, esto es, los que se encuentran en las ''colas'' de la Gaussiana, provocando que el estimador final se aleje del valor verdadero. Por ello es importante cuantificar la distribución de error antes de aplicar ciegamente máxima verosimilitud o cuadrados mínimos.

\subsubsection{Método de máximo a posteriori (MAP)}
En lugar de buscar las mediciones más probables como en el método de máxima verosimilitud, puede buscarse el \textit{posterior}, el cual está asociado a la probabilidad condicional que es asignada luego de tomar en cuenta la medición, esto es, $p(x|\bm{y})$. Mediante el uso de la regla de Bayes, el posterior puede expresarse como
\begin{align}
    p(x|\bm{y}) &= \frac{p(\bm{y}|x)p(x)}{p(\bm{y})} \\
    &\propto p(\bm{y}|x)p(x)
\end{align}

Reemplazando en la Expresión (\ref{eq:maxlikelihood}) por el posterior, entonces
\begin{align}
    \hat{x}_{MAP} &= argmax_x\ p(\bm{y}|x)p(x) \\
                  &= argmax_x\ \log p(\bm{y}|x) + \log p(x)
\end{align}
se obtiene la \textit{expresión del máximo a posteriori}. A diferencia del método de máxima verosimilitud, en el máximo a posteriori se incluye a la probabilidad previa (o \textit{prior}) $p(x)$. Lo que significa es que, la probabilidad ahora está ponderada con algo de peso proveniente del prior.

Puede entonces decirse que el método de máxima verosimilitud es un caso especial del máximo a posteriori, donde la probabilidad previa es uniforme.

\subsection{Regresión no lineal}

La regresión lineal es un método muy utilizado para analizar los datos descriptos por modelos que son lineales en los parámetros. Sin embargo, muchas veces las formas de las curvas que mejor ajustan a los datos obtenidos son no lineales en los parámetros. En estos casos, el modelo de regresión sigue siendo de la misma forma que el de la Expresión (\ref{eq:regressionmodel}), a diferencia de que al menos una de las derivadas de la función de expectativa con respecto a los parámetros depende de al menos uno de los parámetros\footnote{Por ejemplo, si derivo $\theta_1 \theta_2 x_1$ respecto a cualquiera de esos dos parámetros, dicha derivada parcial dependerá del parámetro no derivado}.

Para diferenciar entre modelos lineales y no lineales, los parámetros para este último caso se definen con $\bm{\theta}$,
\begin{equation}
    y_i = f(\bm{x}_i, \bm{\theta}) + v_i
    \label{eq:nonlinearregressionmodel}
\end{equation}

El criterio de error cuadrático para modelos no lineales es uno general, del cual puede derivarse la Expresión (\ref{eq:squarederrorcriterion})
\begin{align}
    \mathscr{S}(\bm{\theta}) &= \sum_{i=1}^n (y_i - f(\bm{x}_i;\bm{\theta}))^2 \\
                   &= (\bm{y} - \bm{f}(\bm{\theta}))^T\bm{R}^{-1}(\bm{y} - \bm{f}(\bm{\theta})) \\
                   &= \bm{y}^T \bm{R}^{-1}\bm{y} - 2\bm{y}^T\bm{R}^{-1}\bm{f}(\bm{\theta}) + \bm{f}(\bm{\theta})^T\bm{R}^{-1}\bm{f}(\bm{\theta})
    \label{eq:generalsquarederrorcriterion}
\end{align}
donde $\bm{f}(\bm{\theta}) = (f_1(\bm{\theta}), f_2(\bm{\theta}), ..., f_n(\bm{\theta}))$ y $f_i(\bm{\theta}) = f(\bm{x}_i;\bm{\theta})$.

A diferencia de la situación de cuadrados mínimos lineales, $\mathscr{S}(\bm{\theta})$ puede tener varios mínimos relativos además del mínimo absoluto $\hat{\bm{\theta}}$. Por ello, si bien el \textit{minimizador global} presentado en la Expresión (\ref{eq:globalminimizer}) es válido para este tipo de modelos, este problema es muy difícil de resolver en general, por lo que se presentan sólo métodos para resolver el problema más simple de encontrar un minimizador local para $f$, un vector de argumento que proporciona un valor mínimo de $f$ dentro de una determinada región cuyo tamaño está dado por $\gamma$, donde $\gamma$ es pequeño y positivo. Entonces, debe cumplirse que

\begin{equation}
    f(\theta^*) \leq f(\theta)\hspace{1cm}\text{para}\hspace{1cm}||\theta - \theta^*|| < \gamma
    \label{eq:localminimizer}
\end{equation}
conocido como el \textit{minimizador local}.

La minimización de $\mathscr{S}$ respecto a sus parámetros debe ser realizada iterativamente.
%El objetivo de cada iteración es encontrar una perturbación $\bm{\delta}$ a los parámetros $\bm{\theta}$ que reduzca $\mathscr{S}$.
Desde un punto inicial $\bm{\theta}_0$, el método produce una serie de vectores $\bm{\theta}_1$, $\bm{\theta}_2$, ..., los cuales se espera que converjan a $\bm{\theta}^*$, un minimizador local para la función dada. La mayoría de los métodos tienen medidas que imponen la \textit{condición descendente}
\begin{equation}
    \bm{f}_{i+1}(\bm{\theta}) < \bm{f}_{i}(\bm{\theta})
    \label{eq:descendingcondition}
\end{equation}

Esto evita la convergencia a un maximizador y también hace menos probable que se converja hacia un punto de silla\footnote{Punto sobre una superficie en el que la pendiente es cero pero no se trata de un extremo local, sino que la elevación es máxima en una dirección y mínima en la dirección perpendicular.}. Si la función dada tiene varios minimizadores, el resultado dependerá del punto de partida $\bm{\theta}_0$. No se sabe a priori cuál de los minimizadores se encontrarán, por lo que no es necesariamente el minimizador más cercano a $\bm{\theta}_0$.

En muchos casos, el método produce vectores que convergen hacia el minimizador en dos etapas claramente diferentes. Cuando $\bm{\theta}_0$ está lejos de la solución, se busca que el método produzca iteraciones que se muevan constantemente hacia $\bm{\theta}^*$. En esta ''etapa global'' de la iteración, es necesario que los errores no aumenten, excepto en los primeros pasos, es decir
\begin{equation}
    ||\bm{e}_{k+1}|| < ||\bm{e}_k||\hspace{1cm}\text{para}\hspace{1cm}k<K
\end{equation}
donde $\bm{e}_k$ responde a la función de error
\begin{equation}
    \bm{e}_k = \bm{\theta}_k - \bm{\theta}^*
\end{equation}

En la etapa final de la iteración, donde $\bm{\theta}_k$ es cercano a $\bm{\theta}^*$, se quiere una convergencia más rápida. A partir de esto pueden distinguirse tres casos
\begin{itemize}
    \item \textit{Convergencia lineal}
    \begin{equation}
            ||\bm{e}_{k+1}|| \leq a||\bm{e}_k||\hspace{1cm}\text{cuando $||e_k||$ es pequeño}\hspace{1cm}0<a<1
    \end{equation}
    \item \textit{Convergencia cuadrática}
    \begin{equation}
        ||\bm{e}_{k+1}||=O(||\bm{e}_k||^2)\hspace{1cm}\text{cuando $||\bm{e}_k||$ es pequeño}
    \end{equation}
    \item \textit{Convergencia superlineal}
    \begin{equation}
        \frac{||\bm{e}_{k+1}||}{||\bm{e}_k||}\rightarrow 0\hspace{1cm}\text{para $k\rightarrow 0$}
    \end{equation}
\end{itemize}

Estos métodos presentados son \textit{métodos de descenso} que satisfacen la condición descendiente de la Expresión (\ref{eq:descendingcondition}) en cada paso de la iteración. Un paso del iterador actual consiste en
\begin{itemize}
    \item Encontrar una \textit{dirección de descenso} $\bm{\delta}_d$, el cual lo será para $\bm{f}$ en $\bm{\theta}$ si $\bm{\delta}^T\bm{f}'(\bm{\theta}) < 0$. En caso de no existir $\bm{\delta}$, entonces $\bm{f}'(\bm{\theta}) = 0$, representando en este caso que $\bm{\theta}$ es estacionario.
    \item Si existe $\bm{\delta}$, hallar una longitud de paso que de una buena disminución en el valor $\bm{f}$ en la dirección dada por $\bm{\delta}_d$, con tal de obtener una disminución en el valor de la función objetivo.
\end{itemize}

Una forma de hacer esto mediante el proceso llamado \textit{búsqueda de línea}, el cual consiste en hallar una aproximación a
\begin{equation}
    \alpha_e = argmin_{\alpha > 0} \bm{f}(\bm{\theta}+\alpha \bm{\delta})
\end{equation}

\subsubsection{Método de descenso de gradiente}

El método de descenso de gradiente es un método de minimización general que actualiza los valores de los parámetros en la dirección de descenso más pronunciada, esto es, la dirección opuesta al gradiente de la función objetivo. La idea detrás de este método radica en moverse repetidamente en la dirección del gradiente negativo (llamado el \textit{descenso más pronunciado}) hasta que converja. El método de descenso de gradiente converge bien para problemas con funciones objetivas simples. Para problemas con muchos parámetros, este método es a veces la única opción viable.

Volviendo al caso de la Expresión (\ref{eq:generalsquarederrorcriterion}), el gradiente de dicha función con respecto a los parámetros es
\begin{align}
    \frac{\partial \mathscr{S}}{\partial \bm{\theta}}\bigg\rvert_{\bm{\theta}=\hat{\bm{\theta}}} &= 2(\bm{y} - \bm{f}(\hat{\bm{\theta}}))^T \bm{R}^{-1}\frac{\partial}{\partial \bm{\theta}}(\bm{y} - \bm{f}(\hat{\bm{\theta}}) \\
    &= -2(\bm{y} - \bm{f}(\hat{\bm{\theta}}))^T \bm{R}^{-1}\left[\frac{\partial \bm{f}(\hat{\bm{\theta}})}{\partial \hat{\bm{\theta}}}\right] \\
    &= -2(\bm{y} - \bm{f}(\hat{\bm{\theta}}))^T \bm{R}^{-1}\bm{J}
\end{align}
donde la \textit{matriz Jacobiana} $\bm{J}$ representa la sensibilidad local de la función $\bm{f}$ a variaciones de los parámetros $\hat{\bm{\theta}}$\footnote{Intuitivamente, la matriz Jacobiana indica qué tan rápido está cambiando cada salida de la función a lo largo de cada dimensión de entrada}. Tener en cuenta que en los modelos que son lineales en los parámetros, $\bm{f} = \bm{X}\bm{\theta}$, el Jacobiano es la matriz de los vectores base del modelo, $\bm{X}$. La actualización de parámetros $\bm{\delta}$ que mueve los parámetros en la dirección del descenso más pronunciado viene dada por
\begin{equation}
    \bm{\delta}_{gd} = \alpha\bm{J}^T\bm{R}^{-1}(\bm{y} - \bm{f}(\hat{\bm{\theta}}))
\end{equation}
donde el escalar positivo $\alpha$ determina la longitud del escalón en la dirección de descenso más pronunciada.

\subsubsection{Método de Gauss-Newton}
El método de Gauss-Newton es un método para minimizar una función objetivo de suma de cuadrados. Presume que la función objetivo es aproximadamente cuadrática en los parámetros cercanos a la solución óptima. Para problemas de tamaño moderado, el método de Gauss-Newton generalmente converge mucho más rápido que el método de descenso de gradiente.

La función evaluada con parámetros de modelo perturbados puede aproximarse localmente a través de una expansión de la serie Taylor de primer orden.
\begin{equation}
    \bm{f}(\hat{\bm{\theta}+\bm{\delta}}) \approx \bm{f}(\hat{\bm{\theta}}) + \left[\frac{\partial \bm{f}}{\partial \hat{\bm{\theta}}}\right]\bm{\delta} = \bm{f}(\hat{\bm{\theta}}) + \bm{J}\bm{\delta}
    \label{eq:gaussnewtontaylor}
\end{equation}

Substituyendo la Expresión (\ref{eq:gaussnewtontaylor}) en (\ref{eq:generalsquarederrorcriterion}) para $\mathscr{S}(\bm{\theta} + \bm{\delta})$
\begin{equation}
    \begin{aligned}
        \mathscr{S}(\bm{\theta}+\bm{\delta}) \approx{} &\bm{y}^T\bm{R}^{-1}\bm{y} + \bm{f}(\bm{\theta})^T\bm{R}^{-1}\bm{f}(\bm{\theta}) - 2\bm{y}^T\bm{R}^{-1}\bm{f}(\bm{\theta}) - \\
        & -2\left(\bm{y} - \bm{f}(\bm{\theta})\right)^T\bm{R}^{-1}\bm{J}\bm{\delta} + \bm{\delta}^T\bm{J}^T\bm{R}^{-1}\bm{J}\bm{\delta}
    \end{aligned}
\end{equation}
La aproximación de Taylor de primer orden  de la Expresión (\ref{eq:gaussnewtontaylor}) da como resultado una aproximación para $\mathscr{S}$ que es cuadrática en $\bm{\delta}$. Por lo tanto, el $\bm{\delta}$ que minimiza a $\mathscr{S}$ se encuentra cuando $\frac{\partial \mathscr{S}}{\partial \bm{\delta}} = 0$
\begin{equation}
    \frac{\partial}{\partial \bm{\delta}}\mathscr{S}(\bm{\theta} + \bm{\delta}) \approx -2\left(\bm{y} - \bm{f}(\bm{\theta})\right)^T\bm{R}^{-1}\bm{J} + 2\bm{\delta}^T\bm{J}^T\bm{R}^{-1}\bm{J}
\end{equation}
y las ecuaciones normales resultantes para la actualización de Gauss-Newton son
\begin{equation}
    \left[\bm{J}^T\bm{R}^{-1}\bm{J}\right]\bm{\delta}_{gn} = \bm{J}^T\bm{R}^{-1}(\bm{y} - \bm{f}(\bm{\theta})
\end{equation}

\subsubsection{Método de Levenberg-Marquardt}
Aunque el Método de Gauss-Newton converge muy rápido cuando ya está cerca de la solución, puede fallar en dar un paso de descenso válido cuando el optimizador aún está lejos del mínimo buscado.

El algoritmo de Levenberg-Marquardt aborda este problema variando adaptativamente las actualizaciones de parámetros entre la actualización de descenso de gradiente y la actualización de Gauss-Newton,
\begin{equation}
    \left[\bm{J}^T\bm{R}^{-1}\bm{J} + \lambda\bm{I}\right]\bm{\delta}_{lm} = \bm{J}^T\bm{R}^{-1}\left(\bm{y} - \bm{f}(\bm{\theta}\right)
\end{equation}{}
donde pequeños valores del \textit{parámetro de amortiguación} $\lambda$ resultan en una actualización de Gauss-Newton, mientras que valores grandes de $\lambda$ llevan a una actualización de descenso de gradiente. El parámetro de amortiguación $\lambda$ se inicializa para que sea grande, de modo que las primeras actualizaciones sean pequeños pasos en la dirección de descenso más pronunciada. Si alguna iteración resulta en una peor aproximación $(\mathscr{S}(\bm{\theta} + \bm{\delta}_{lm}) > \mathscr{S}({\bm{\theta}}))$, entonces $\lambda$ aumenta. De lo contrario, a medida que la solución mejora, $\lambda$ disminuye, haciendo que el método de Levenberg-Marquardt se aproxima al método de Gauss-Newton, y la solución generalmente se acelera al mínimo local.

En la relación de actualización de Marquardt
\begin{equation}
    \left[\bm{J}^T\bm{R}^{-1}\bm{J} + \lambda\;  diag(\bm{J}^T\bm{R}^{-1}\bm{J})\right]\bm{\delta}_{lm} = \bm{J}^T\bm{R}^{-1}\left(\bm{y} - \bm{f}(\bm{\theta})\right)
\end{equation}{}
los valores de $\lambda$ se normalizan a los valores de $\bm{J}^T\bm{R}^{-1}\bm{J}$.

En sistemas donde solo hay un mínimo, el método de Levenberg-Marquardt convergerá al mínimo global incluso si la suposición inicial es arbitraria. En cambio, en sistemas con mínimos múltiples, es más probable que el método encuentre el mínimo global si la suposición inicial se encuentra próxima a la solución. Sin embargo, el mismo permite que la selección de valores iniciales esté más lejos de la solución que el método de Gauss-Newton.

Si bien existen numerosas implementaciones de este método, se presentará a continuación la solución propuesta por la \textit{librería Eigen}.

    % LO PONGO O NO LO PONGO?!?!?!??!?!
% \ifslamext
    \newpage
    \section{Estimación de estado}
La estimación de estado consiste en determinar el estado no medible de un sistema dinámico a partir de las mediciones de entrada y salida de dicho sistema, esto es, dado un estado inicial $\bm{x}_0$ y las observaciones $\bm{y}_1, \bm{y}_2, ..., \bm{y}_k$ conocidas en el tiempo $k$, el problema de estimación de estado en el paso de tiempo $k$ se basa en construir un estimador $\hat{\bm{x}}_k$ de $\bm{x}_k$.

Lamentablemente, las mediciones no son perfectas, lo que conduce a una inexactitud inherente en el valor de la medición. Para tener en cuenta estos errores, la estimación de estado procesa todas las mediciones disponibles y utiliza un \textit{análisis de regresión} para identificar el estado real probable del sistema.

\subsection{Filtro de Kalman}
El filtro de Kalman es un filtro Gaussiano con transición de estado y función de medición \textit{lineales}. Se le atribuye a {\big[\textbf{Kalman, 1960}]} y {\big[\textbf{Swerling, 1958}]} y se ha aplicado por primera vez al seguimiento por radar de objetivos aéreos, pero se ha utilizado en una gran cantidad de otros problemas de estimación desde entonces. El filtro de Kalman y sus diversos derivados son casi ubicuos en las aplicaciones de fusión de sensores.

El objetivo de este filtro, entonces, es computar $\{\hat{\bm{x}}_k,\hat{\bm{P}_k}\}$ utilizando toda la información disponible, incluyendo la presente en el tiempo $k$. Mientras el método de cuadrados mínimos recursivo actualiza la estimación de un parámetro estático, el filtro de Kalman es capaz de actualizar la estimación de un estado en evolución. Como se deriva del filtro general de Bayes{\Big[\textbf{REF BAYES}]}, el objetivo de este filtro es tomar una estimación probabilística de este estado y actualizarla en tiempo real usando dos pasos, \textit{predicción} y \textit{corrección}.

Para poder entender mejor el filtro, se considera el problema de estimar la posición de un vehículo en una dimensión. Iniciando de una estimación probabilística en el tiempo $k-1$, el objetivo es predecir el nuevo estado mediante un modelo de moción, el cual puede ser derivado, por ejemplo, de la odometría de las ruedas o de las mediciones de una unidad inercial. Luego, con el modelo de observación derivado de, por ejemplo, los datos del GPS, se corrige esa predicción de la posición del vehículo en el tiempo $k$, tal como se observa en la Figura \ref{fig:kalmanfilter}. Cada una de estas componentes, la estimación inicial, el estado predicho, y el estado final corregido son todas variables aleatorias especificadas por sus valores medios y covarianzas. De este modo, puede pensarse al Filtro de Kalman como una técnica para fusionar información de diferentes sensores para producir una estimación final de un estado desconocido, tomando en cuenta incertidumbres en el moción y en las mediciones\footnote{Para identificar estados predichos de corregidos, por ejemplo en la matriz de covarianza $\bm{P}$, para el primer caso se utiliza la notación $\check{\bm{P}}$, mientras que para el segundo caso $\hat{\bm{P}}$.}.

\begin{figure}
    \centering
    \includegraphics[width=\textwidth]{Img/KalmanFilter.png}
    \caption{Filtro de Kalman en una dimensión}
    \label{fig:kalmanfilter}
\end{figure}

En concreto, el filtro de Kalman requiere de un modelo de moción
\begin{equation}
    \bm{x}_k = \bm{F}_{k-1}\bm{x}_{k-1} + \bm{G}_{k-1}\bm{u}_{k-1} + \bm{w}_{k-1},
\end{equation}
y un modelo de medición lineal
\begin{equation}
    \bm{y}_k = \bm{H}_k\bm{x}_k + \bm{v}_k,
\end{equation}
siendo $\bm{u}_{k-1}$ una \textit{entrada de control} externa que afecta la evolución del estado del sistema, $\bm{v}_{k-1}$ el \textit{ruido de la medición} y $\bm{w}_{k-1}$ el \textit{rudo del proceso} que gobierna la incertidumbre de las entradas de control
\begin{equation}
    \bm{v}_k \sim \mathcal{N}(\bm{0},\bm{R}_k)\hspace{2cm}\bm{w}_k \sim \mathcal{N}(\bm{0},\bm{Q}_k)
\end{equation}

Puede decirse que el filtro de Kalman \textit{es muy similar a un estimador de cuadrados mínimos recursivo que incluye un modelo de moción} el cual indica cómo el estado evoluciona a través del tiempo. El mismo consta de dos pasos
\begin{enumerate}
    \item \textbf{Predicción}. Primero, se utiliza el modelo del proceso o moción para predecir como los estados evolucionaron desde el último paso de tiempo, y se propaga la incertidumbre.
    \begin{align}
        \check{\bm{x}}_k &= \bm{F}_{k-1}\bm{x}_{k-1}+\bm{G}_{k-1}\bm{u}_{k-1} \\
        \check{\bm{P}}_k &= \bm{F}_{k-1}\hat{\bm{P}}_{k-1}\bm{F}_{k-1}^T + \bm{Q}_{k-1}
    \end{align}
    \item \textbf{Actualización}. En el mismo se utiliza el modelo de medición y puede dividirse en dos pasos
    \begin{enumerate}
        \item \textit{Ganancia óptima}. Se utiliza la medición para corregir la predicción basada en el residuo o innovación de la medición y la ganancia óptima
        \begin{equation}
            \bm{K}_k = \check{\bm{P}}_{k}\bm{H}_k^T\left(\bm{H}_k\check{\bm{P}}_k\bm{H}_k^T + \bm{R}_k\right)^{-1}
        \end{equation}
        \item \textit{Corrección}. Se utiliza la ganancia para propagar la covarianza de estado de la predicción a la estimación corregida.
        \begin{align}
            \hat{\bm{x}}_k &= \check{\bm{x}}_k + \bm{K}_k\left(\bm{y}_k - \bm{H}_k\check{\bm{x}}_k\right) \\
            \hat{\bm{P}}_k &= (\bm{1} - \bm{K}_k\bm{H}_k)\check{\bm{P}}_k
        \end{align}
        siendo $(\bm{y}_k - \bm{H}_k\check{\bm{x}}_k)$ la llamada \textit{innovación de la medición}.
    \end{enumerate}
\end{enumerate}

Se dice que un estimador o filtro es \textit{insesgado} (en inglés, \textit{unbiased}) si produce un error promedio de cero para todo paso de tiempo k
\begin{equation}
    E[\hat{p}_k-p_k] = 0
\end{equation}
donde $\hat{p}_k$ denota al valor estimado y $p_k$ al valor verdadero.

Considerando la dinámica de error, siendo el error de estado predicho
\begin{equation}
    \check{\bm{e}}_k = \check{\bm{x}}_k - \bm{x}_k
\end{equation}
y el error de estimación corregido
\begin{equation}
    \hat{\bm{e}}_k = \hat{\bm{x}}_k - \bm{x}_k
\end{equation}
se puede obtener mediante el uso de las ecuaciones del filtro de Kalman que
\begin{align}
    \check{\bm{e}}_k &= \bm{F}_{k-1}\check{e}_{k-1} - \bm{w}_k \\
    \check{\bm{e}}_k &= \left(\bm{1} - \bm{K}_k\bm{H}_k\right)\check{\bm{e}}_k + \bm{K}_k\bm{v}_k
\end{align}

Si se tiene ruido blanco no correlacionado con media cero y
\begin{equation*}
    E[\hat{\bm{e}}_0] = \bm{0}\hspace{0.5cm}\land\hspace{0.5cm}E[\bm{v}] = \bm{0}\hspace{0.5cm}\land\hspace{0.5cm}E[\bm{w}] = \bm{0}
\end{equation*}
se llega a que el valor esperado de estos errores es cero para todo $k$
\begin{align}
    E[\check{\bm{e}}_k] &= \bm{F}_{k-1}E[\check{\bm{e}}_{k-1}] - E[\bm{w}_k] = \bm{0} \\
    E[\hat{\bm{e}}_k] &= \left(\bm{1} - \bm{K}_k\bm{H}_k\right)E[\check{\bm{e}}_k] + \bm{K}_k E[\bm{v}_k] = \bm{0}
\end{align}

Por consistencia se dice a que, para todos los pasos de tiempo $k$, la covarianza del filtro, $P_k$, equivale al valor esperado del cuadrado del error
\begin{equation}
    E[\hat{e}_k^2] = E[(\hat{p}_k - p_k)^2] = \hat{P}_k
\end{equation}
Esto quiere decir que el filtro no es ni demasiado seguro (optimista) ni inseguro respecto a la estimación que ha producido.

Se puede demostrar que, si se tiene ruido blanco y
\begin{equation*}
    E[\hat{\bm{e}}_0\hat{\bm{e}}_0^T] = \check{\bm{P}}_0\hspace{0.5cm}\land\hspace{0.5cm}E[\bm{v}] = \bm{0}\hspace{0.5cm}\land\hspace{0.5cm}E[\bm{w}] = \bm{0}
\end{equation*}
las predicciones serán consistentes
\begin{equation*}
    E[\check{\bm{e}}_k\check{\bm{e}}_k^T] = \check{\bm{P}_k}\hspace{1cm}\land\hspace{1cm}E[\hat{\bm{e}}_k\hat{\bm{e}}_k^T] = \hat{\bm{P}}_k
\end{equation*}

En conclusión, si el ruido es blanco no correlacionado con media cero, el filtro de Kalman es \textit{no sesgado} y \textit{consistente}. Debido a estos dos hechos, se dice que el filtro de Kalman es el \textit{mejor estimador lineal no sesgado} (o \textit{BLUE} por sus siglas en inglés), ya que produce estimaciones no sesgadas con la menor varianza posible.

\subsection{Filtro de Kalman Extendido}
Si bien el filtro de Kalman es el mejor estimador lineal no sesgado, por lo general los sistemas reales no son lineales. El concepto principal del filtro de Kalman Extendido es el de linealizar un sistema no lineal, esto es, elegir un punto de operación $a$ y hallar una aproximación lineal a la función no lineal en la vecindad del punto. En dos dimensiones refiere a encontrar la recta tangente, por ejemplo. Se llega a esto matemáticamente realizando la serie de Taylor de la función y tomando solo los términos de primer orden, esto es
\begin{equation}
    f(x)\approx f(a) + \frac{\delta f(x)}{\delta x}\bigg\rvert_{x=a}(x-a)
\end{equation}

Para el caso del filtro de Kalman Extendido, se elige como punto de operación al estimador de estado más reciente, entonces, el modelo de moción linealizado
\begin{equation}
    \begin{aligned}
        \bm{x}_k &= \bm{f}_{k-1}(\bm{x}_{k-1},\bm{u}_{k-1},\bm{w}_{k-1})\\
        &\approx \bm{f}_{k-1}(\hat{\bm{x}}_{k-1},\bm{u}_{k-1},\bm{0}) + \bm{F}_{k-1}\left(\bm{x}_{k-1} - \hat{\bm{x}}_{k-1}\right) + \bm{L}_{k-1}\bm{w}_{k-1}
    \end{aligned}
\end{equation}
y el modelo de medición linealizado
\begin{equation}
    \bm{y}_k = \bm{h}_k(\bm{x}_k,\bm{v}_k)\approx \bm{h}_k(\check{\bm{x}}_k,\bm{0}) + \bm{H}_k(\bm{x}_k - \check{\bm{x}}_k) + \bm{M}_k \bm{v}_k
\end{equation}
siendo
\begin{align}
    \bm{F}_{k-1} &= \frac{\delta \bm{f}_{k-1}}{\delta \bm{x}_{k-1}}\bigg\rvert_{\hat{\bm{x}}_{k-1},\bm{u}_{k-1},\bm{0}} \\
    \bm{L}_{k-1} &= \frac{\delta \bm{f}_{k-1}}{\delta \bm{w}_{k-1}}\bigg\rvert_{\hat{\bm{x}}_{k-1},\bm{u}_{k-1},\bm{0}} \\
    \bm{H}_k &= \frac{\delta \bm{h}_k}{\delta \bm{x}_k}\bigg\rvert_{\check{\bm{x}}_{k},\bm{0}} \\
    \bm{M}_k &= \frac{\delta \bm{h}_k}{\delta \bm{v}_k}\bigg\rvert_{\check{\bm{x}}_{k}\bm{0}}
\end{align}
las llamadas matrices Jacobianas del sistema.

Finalmente, teniendo en cuenta los modelos linealizados y las matrices Jacobianas, se llega a los pasos del filtro de Kalman Extendido
\begin{enumerate}
    \item \textbf{Predicción}
        \begin{align}
            \check{\bm{x}}_k &= \bm{f}_{k-1}(\hat{\bm{x}}_{k-1},\bm{u}_{k-1},\bm{0})\\
            \check{\bm{P}}_k &= \bm{F}_{k-1}\hat{\bm{P}}_{k-1}\bm{F}_{k-1}^T + \bm{L}_{k-1}\bm{Q}_{k-1}\bm{L}_{k-1}^T
        \end{align}
    \item \textbf{Actualización}
    \begin{enumerate}
        \item \textit{Ganancia óptima}
            \begin{equation}
                \bm{K}_k = \check{\bm{P}}_k\bm{H}_k^T(\bm{H}_k\check{\bm{P}}_k\bm{H}_k^T + \bm{M}_k\bm{R}_k\bm{M}_k^T)^{-1}
            \end{equation}
        \item \textit{Correción}
            \begin{align}
                \hat{\bm{x}}_k &= \check{\bm{x}}_k + \bm{K}_k(\bm{y}_k - \bm{h}_k(\check{\bm{x}}_k,\bm{0}))\\
                \hat{\bm{P}}_k &= (\bm{1} - \bm{K}_k\bm{H}_k)\check{\bm{P}}_k
            \end{align}
    \end{enumerate}
\end{enumerate}
% \fi
    \newpage
    \section{Sensores}
Para navegar de manera robusta a través de entornos desconocidos y no estructurados, los robots deben poder percibir y modelar su entorno. Es por esto que lo que se busca es construir mapas precisos y ubicarse en los mismos utilizando solo sensores a bordo. Dentro de los mismos pueden diferenciarse dos grandes grupos
\begin{itemize}
    \item En primer lugar, los sensores necesarios para obtener información de los \textit{alrededores} del robot, como pueden ser el LIDAR y la cámara. Estos se los denominan \textit{sensores exteroceptivos}. Un sistema SLAM mínimo requiere al menos de uno de ellos para poder realizar dicha tarea.
    \item Opcionalmente, aquellos sensores que miden el movimiento propio del robot, como pueden ser los acelerómetros y encoders, los que se denominan \textit{sensores propioceptivos}.
\end{itemize}

\subsection{LiDAR}
Light Detection And Ranging, conocido como LiDAR, es una tecnología que tiene su origen en la fusión de la tecnología láser junto con la tecnología RADAR (Radio Detection And Ranging), lo cual ha permitido mejorar en gran medida la precisión de los sistemas de detección, dando lugar a nuevas aplicaciones. Los sensores LIDAR son actualmente una de las opciones más confiables para SLAM robótico tanto en ambientes interiores como exteriores. Los mismos exhibien fuentes y tipos de ruido similares que pueden modelarse libremente como Gaussianos [133].

Muchos investigadores tienen sensores LiDAR de exploración servo-montados en configuraciones de cabeceo [116, 117] o de barrido [118] (Figura \ref{fig:lidars}.a) para producir exploraciones tanto planares como 3D, aunque para este último caso el mismo entrega escaneos cada 1Hz o menos. Este campo de visión, sobre todo el 3D, tiene el costo de una mayor complejidad (sincronización servo temporal) y una menor cobertura en direcciones críticas.
\textbf{[pfingsthron2012] [newman2006] [bosse2009]}

Con el fin de aumentar la frecuencia de recolección de datos, existen sensores que integran múltiples LiDAR en una sola unidad de escaneo (Figura \ref{fig:lidars}.b), de modo que se pueden obtener escaneados en 3D completos a altas velocidades.

\begin{figure}[!b]
    \centering
    \subfloat[Scanse Sweep]{\includegraphics[width=.35\textwidth]{Img/scanse-sweep}}
    \qquad
    \subfloat[Velodyne VLP-16]{\includegraphics[width=.5\textwidth]{Img/vlp-16}}
    \caption{Algunos LIDAR comerciales}
    \label{fig:lidars}
\end{figure}

\subsubsection{Principio de funcionamiento}
El  fundamento de los dispositivos basados en la tecnología LIDAR es el cálculo del tiempo de vuelo (\textit{ToF - Time  Of Flight}) de los pulsos láser, de manera que, conociendo la velocidad del mismo, las características angulares con las que fue emitido, y la diferencia de tiempos entre el rayo emitido y el reflejado, se puede determinar de manera sencilla la distancia a la que se encuentra el obstáculo/objeto con el que el rayo impactó. Esto permite, con gran exactitud, conocer las coordenadas de la posición de objetos o superficies con respecto del sistema de coordenadas del propio dispositivo.

A medida que cada pulso láser diverge, traza un volumen que es aproximadamente cónico. Si alguna parte de este volumen cónico se cruza con un objeto, parte de la luz reflejada puede regresar a través de la lente del LIDAR. Una vez que la parte frontal del receptor del LIDAR ha recogido suficiente luz (es decir, con un umbral) registra el retorno y calcula la distancia. Esta señal de entrada introduce efectos de acortamiento y alargamiento de rango.

\subsubsection{Fuentes de ruido}
Algunas de las fuentes de ruido que afectan a la medición del LIDAR son
\begin{itemize}
    \item \textit{Ángulo de incidencia:} Si un pulso LIDAR golpea una superficie perpendicularmente, todo el frente de la onda láser se refleja al mismo tiempo. Sin embargo, a medida que aumenta el ángulo de incidencia, parte del frente de onda se refleja antes y la señal recibida activará el umbral, tarde o temprano, dependiendo del diseño del detector frontal. A medida que el ángulo de incidencia se acerca a los 90 grados, una configuración común para el lidar montado horizontalmente, los pulsos lidar viajan casi paralelos al suelo. Estos retornos de ''pastoreo en el suelo'' son extremadamente sensibles al tono del robot, las variaciones en la superficie y otros ruidos. En esta configuración, incluso un robot sin movimiento puede producir mediciones lidar con medidores de ruido de rango. Estos retornos de pastoreo pueden presentar un desafío para SLAM con un lidar montado horizontalmente.
    \item \textit{Efectos de contorno:} están estrechamente relacionados con los retornos de pastoreo, los retornos espurios pueden ocurrir en el límite de los objetos, donde el frente de onda elíptica puede intersecar parte de uno o más objetos a medida que viaja. La medición del rango resultante a menudo se promedia y se crea una medición espuria "colgando" en el espacio vacío entre los objetos. La figura 2.4 demuestra este efecto.
    \item \textit{Propiedades de la superficie del objeto:} causa que la cantidad de luz láser reflejada varíe mucho. Un objeto altamente especular, como un espejo o agua quieta, reflejará la mayor parte de la luz láser y, a menudo, devolverá las mediciones a objetos más distantes (en el rumbo incorrecto). Cuanto más difusamente un objeto refleje la luz, mejor se puede medir en un rango más amplio de ángulos y distancias. Los objetos que absorben la luz infrarroja (que generalmente se ve negra para los humanos) a menudo no pueden reflejar suficiente luz, lo que limita los rangos de medición. Además, en algunos sensores lidar, los objetos blancos pueden aparecer un poco más cerca que los objetos negros, ya que el umbral del receptor se activa un poco antes [133].
    \item \textit{Luz ambiental:} La mayoría de los sensores comerciales LIDAR utilizan filtros de muesca infrarrojos para aumentar la relación señal a ruido. Si bien esto les permite funcionar al aire libre, la luz ambiental intensa, como la luz solar directa, disminuirá su alcance. Cuando la luz reflejada de un pulso lidar no es lo suficientemente fuerte como para activar una medición, un modelo de sensor típico supone que hay espacio libre hasta una fracción del rango máximo del sensor. Esta suposición de espacio libre debe variar según los niveles de luz ambiental, sin embargo, sin sensores adicionales, no es posible determinar cuándo la medición de lidar faltante se debe al espacio libre real o a la luz ambiental excesiva.
    \item \textit{Sincronización temporal:} cuando se monta en un robot en movimiento, el origen de un lidar se moverá a medida que el sensor giratorio complete cada escaneo. Moviéndose a $1m/s$, por ejemplo, un lidar de 40 Hz producirá hasta 25 mm de sesgo si no se compensa. La compensación se puede realizar utilizando un modelo de movimiento continuo, como [134], sin embargo, esto requiere una sincronización rigurosa y una marca de tiempo de los datos del sensor. La figura 2.4 muestra una nube de puntos tridimensional coloreada generada a partir de un lidar montado en un servomotor de movimiento rápido, con una cámara de obturación global y sincronización de microsegundos.
\end{itemize}

\subsubsection{Calibración}
MIRAR

\subsection{Cámara}
Las cámaras en su versión más simple tienen la característica de ser más económicas y sencillas de montar respecto al resto de los sensores utilizados para el SLAM (tal como el LiDAR), además de no emitir señales al entorno para obtener las características del mismo. A la hora de realizar el SLAM. el modelo de sensado de las cámaras consiste básicamente en un mapeo entre el entorno tridimencional y el plano de la imagen bidimensional. Dependiendo del tipo de cámara (monocular, estéreo, omnidireccional, RGB-D, entre otras), es posible utilizar diferentes modelos matemáticos que permitan relacionar puntos del mundo con su respectiva representación en la imagen.

Para las cámaras monoculares, la línea de base entre las imágenes debe estimarse a partir de la odometría, lo que da lugar al problema de SLAM monocular bien estudiado [124, 125]. Otra forma de realizarlo es mediante la percepción de alto nivel, donde se pueden utilizar señales como el tamaño relativo de un objeto para estimar la escala de la imagen y, por lo tanto, la profundidad [126].

En cambio, para las cámaras RGB-D (\textit{D: Depth} - Profundidad) (Figura \ref{fig:camaras}.b), uno de los métodos presentados en la literatura para la realización del SLAM a partir de las mismas [ENDERS2012] consiste en primero extraer características visuales de las imágenes de color entrantes. Luego, se comparan estas características con las características de imágenes anteriores. Al evaluar las imágenes de profundidad en las ubicaciones de estos puntos de características, se obtienen un conjunto de correspondencias 3D puntuales entre dos cuadros cualquiera. Basándose en estas correspondencias, se estima la transformación relativa entre los marcos utilizando RANSAC [TRIVEDI2013].

Por otro lado, las técnicas de visión estéreo utilizan dos cámaras con una separación entre las mismas fija y conocida (Figura \ref{fig:camaras}.a). Las correspondencias de características de la imagen se identifican entre las imágenes mediante la geometría epipolar[REF DE CASTRO] [111], y los rangos se calculan cuando se han medido las disparidades de imagen. Los enfoques pasivos sufren desde muchos modos de falla, desde agujeros de profundidad en áreas de imagen sin características visuales, hasta escenas que crean ambigüedades [123].

\begin{figure}
    \centering
    \includegraphics[width=.9\textwidth]{Img/bumblekinect}
    \caption{Cámaras: (a) Bumblebee 2, (b) Kinect}
    \label{fig:camaras}
\end{figure}

\subsection{Sensores Inerciales}
El término sensor inercial se usa para denotar la combinación de un acelerómetro de tres ejes y un giroscopio de tres ejes. Los dispositivos que contienen estos sensores se denominan comúnmente unidades de medición inercial (IMU, por \textit{intertial measurement unit}), los cuales en muchos casos incluyen también un magnetómetro de tres ejes. Estas unidades sensoriales suelen ser usadas para determinar la orientación y posición de objetos a los cuales se encuentran integrados, permitiendo así la incorporación de los mismos en gran número de aplicaciones [7, 59, 109, 156] [DE KOK2017].

Hoy en día, muchos de estos sensores se basan en la tecnología de sistemas microelectromecánicos (MEMS). Los componentes de MEMS son pequeños, ligeros, económicos, tienen un bajo consumo de energía y tiempos de arranque cortos. Su precisión ha aumentado significativamente con los años.

Un giroscopio mide la velocidad angular del sensor, es decir, la velocidad de cambio de la orientación del sensor. En cambio, un acelerómetro mide la fuerza específica externa que actúa sobre el sensor. La fuerza específica consiste tanto en la aceleración del sensor como en la gravedad de la Tierra. Por otro lado, los magnetómetros son los encargados de medir el campo magnético en el que el objeto se encuentra sumergido, obteniendo así información absoluta del ambiente y no referida únicamente al objeto en si. Todos estos sensores tienen el inconveniente de presentar un \textit{bias} variable en el tiempo que afecta su funcionamiento, errores de \textit{scaling} utilizados para convertir las salidas digitales de los sensores en cantidades físicas reales, desalineaciones entre ellos mismos en caso de que se trate de un circuito integrado incluyendo a dos o más de ellos, entre otros.

\subsubsection{Calibración}
En una IMU ideal, los grupos triaxiales deben compartir los mismos ejes de sensibilidad ortogonal 3D que abarcan un espacio tridimensional, mientras que el factor de escala debe convertir la cantidad digital medida por cada sensor en la cantidad física real. Desafortunadamente, las IMU basadas en MEMS de bajo costo generalmente se ven afectadas por el escalado no preciso, las desalineaciones del eje del sensor, las sensibilidades de los ejes cruzados y los sesgos distintos de cero. La calibración de la IMU se refiere al proceso de identificación de estas cantidades.

Sin embargo, las IMUs comerciales de costo elevado presentan también estos problemas (aunque en menor escala), con la diferencia de que el fabricante brinda los datos de la calibración, el cual es único para cada una de ellas, minimizando así los términos de incertidumbre. El factor de calibración de estas IMUs se suele realizar con métodos estándar, comparando los datos arrojados por la IMU con referencias conocidas, haciendo que el proceso sea lento y costoso debido al equipamiento necesario. A continuación se presentarán propuestas para resolver la calibración de los diferentes sensores

\paragraph{Acelerómetro-Giróscopo}
Para evitar el uso de equipamiento específico, en \textbf{[TEDALDI2014]} se propone utilizar solo el movimiento propio de la IMU y sus estados intermedios estáticos para calibrar tanto el giróscopo como el acelerómetro. Este procedimiento explota la idea básica del método de múltiples posiciones, presentado primero en [7] para la calibración de acelerómetros: en una posición estática, las normas de las aceleraciones medidas son iguales a las magnitudes de la gravedad más un multi-factor de error de origen (es decir, incluye \textit{bias}, desalineación, ruido, entre otros). Todas estas cantidades pueden estimarse a través de la minimización sobre un conjunto de actitudes estáticas. 

Después de la calibración de la tríada del acelerómetro, es posible utilizar las posiciones del vector de gravedad medidas por los acelerómetros como referencia para calibrar la tríada del giroscopio. Es por esto que la exactitud de la calibración del giroscopo depende fuertemente de la exactitud de la calibración del acelerómetro. Al integrar las velocidades angulares entre dos posiciones estáticas consecutivas, se consigue estimar las posiciones de gravedad en la nueva orientación. La calibración de los giroscopios finalmente se obtiene minimizando los errores entre estas estimaciones y las referencias de gravedad dadas por los acelerómetros calibrados.

Para una IMU ideal, los 3 ejes de la tríada de acelerómetros y los 3 ejes de la tríada de giroscopios definen un marco tridimensional único, compartido y ortogonal. En la realidad, en base a lo explicado anteriormente, los marcos de los acelerómetros y de los giróscopos no suelen ser ortogonales. A su vez, si se tiene a ambos dentro de un mismo chip, puede definirse un marco del cuerpo (o body frame), el cual es un marco ortogonal que representa, por ejemplo, el chasis del integrado. Este marco suele ser distinto a los otros dos marcos, pero como esta diferencia está dada por ángulos muy pequeños, la medición $\bm{m}$ tanto del acelerómetro como del giróscopo en un marco no ortogonal (o sea, el marco del sensor) puede llevarse al marco del cuerpo mediante
\begin{equation}
    \bm{m}^b = \bm{C}\bm{m}^s
\end{equation}
siendo $\bm{C}$ la matriz de rotación de la Expresión (\ref{eq:rpyinfinitesimal}).

Si se asume que el marco del cuerpo coincide con el marco ortogonal del acelerómetro, entonces para el caso de la aceleración
\begin{equation}
    \bm{a}^b = \bm{C}_a\bm{a}^s
\end{equation}
siendo
\begin{equation}
    \bm{C}_a =
    \begin{bmatrix}
        1 & \alpha_{12} & \alpha_{13} \\
        0 & 1 & \alpha_{23} \\
        0 & 0 & 1
    \end{bmatrix}
\end{equation}
Como se mencionó anteriormente, tanto las mediciones del acelerómetro como las del giróscopo deberían referir al mismo marco de referencia, en este caso, el marco del cuerpo. Por lo tanto,
\begin{equation}
    \bm{\omega}^b = \bm{C}_g\bm{\omega}^s
\end{equation}
siendo
\begin{equation}
    \bm{C}_g =
    \begin{bmatrix}
        1 & \gamma_{12} & \gamma_{13} \\
        \gamma_{21} & 1 & \gamma_{23} \\
        \gamma_{31} & \gamma_{32} & 1
    \end{bmatrix}
\end{equation}

Debido al error de conversión, ambos sensores se encuentran afectados por errores de \textit{scaling}
\begin{align}
    \bm{K}_a &=
    \begin{bmatrix}
        sa_x & 0 & 0 \\
        0 & sa_y & 0 \\
        0 & 0 & sa_z
    \end{bmatrix}
    \\
    \bm{K}_g &=
    \begin{bmatrix}
        sg_x & 0 & 0 \\
        0 & sg_y & 0 \\
        0 & 0 & sg_z
    \end{bmatrix}
\end{align}
 y de \textit{bias}
 \begin{align}
    \bm{b}_a &=
    \begin{bmatrix}
        ba_x \\
        ba_y \\
        ba_z
    \end{bmatrix}
    \\
     \bm{b}_a &=
     \begin{bmatrix}
        bg_x \\
        bg_y \\
        bg_z
     \end{bmatrix}
\end{align}

En consecuente, los modelos completos de los sensores resultan
\begin{align}
    \bm{a}^b &= \bm{C}_a\bm{K}_a\left(\bm{a}^s + \bm{b}_a + \bm{v}_a\right) \\
    \bm{\omega}^b &= \bm{C}_g\bm{K}_g\left(\bm{\omega}^s + \bm{b}_g + \bm{v}_g\right)
\end{align}
con $\bm{v}_i$ correspondiendo a los ruidos de medición de cada uno.

En concreto, para cada uno de los sensores se tendrán parámetros a estimar. Para el caso del acelerómetro
\begin{equation}
    \bm{\epsilon}_{a} = [\alpha_{12},\alpha_{13},\alpha_{23},sa_x,sa_y,sa_z,ba_x,ba_y,ba_z]
\end{equation}

Al aplicarse un promediado a cada intervalo de la calibración del acelerómetro, esto permite olvidar el ruido de la medición, y definir entonces
\begin{equation}
    \bm{a}^b = h(\bm{a}^s,\bm{\epsilon}_{a}) = \bm{T}_a\bm{K}_a(\bm{a}^s+\bm{b}_a)    
\end{equation}

Como en [7], se mueve a la IMU un set de $M$ rotaciones independientes y temporalmente estables, obteniendo entonces $M$ aceleraciones $\bm{a}^s_k$, las cuales son promediadas en cada uno de estos intervalos estáticos. La función de coste para minimizar los parámetros este caso es
\begin{equation}
    \mathscr{S}(\bm{\epsilon}_{a}) = \sum_{k=1}^M(||\bm{g}||^2-||h(\bm{a}^s_k,\bm{\epsilon}_{a})||^2)^2
\end{equation}
donde $||g||^2$ es la magnitud actual del vector de gravedad local, obtenido de tablas específicas. Al tratarse de una regresión no lineal, la misma puede minimizarse utilizando el algoritmo de \textit{Levenberg-Marquardt}.

Para calibrar la triada del giróscopo, puede asumirse al mismo libre de \textit{bias} ya que este sesgo puede obtenerse promediando sobre una cantidad de datos consecutivos de tamaño conveniente en el instante inicial estacionario. Por ello, el vector de parámetro desconocidos resulta
\begin{equation}
    \bm{\epsilon}_{g} = \left[\gamma_{12},\ \gamma_{13},\ \gamma_{21},\ \gamma_{23},\ \gamma_{31},\ \gamma_{32},\ sg_x,\ sg_y,\ sg_z\right]
\end{equation}

Una vez definido el \textit{bias}, como para la calibración del giróscopo se toma al acelerómetro como referencia conocida, se requiere convertir los datos de dicho giróscopo en un vector de aceleraciones. Para ello, se define el operador $\psi$, 
\begin{equation}
    \bm{u}_{g,k} = \psi\left[\bm{\omega}_i^s,\bm{u}_{a,k-1}\right]
\end{equation}
que toma como entrada una secuencia de $n$ lecturas del giróscopo $\bm{\omega}_i^s$ y el versor de gravedad inicial $\bm{u}_{a,k-1}$ obtenido del acelerómetro ya calibrado, y retorna el versor de gravidad final $\bm{u}_{g,k}$. 

Una vez obtenido $\bm{u}_{g,k}$, para el caso del giróscopo, la función de coste será entonces
\begin{equation}
    \mathscr{S}(\bm{\epsilon}_{g}) = \sum_{k=2}^M ||\bm{u}_{a,k} - \bm{u}_{g,k}||^2
\end{equation}
siendo $M$ el número de los intervalos estáticos, $\bm{u}_{a,k}$ es el versor de la aceleración medido promediando en la ventana temporal obtenida en el $k$-ésimo intervalo estático, y $\bm{u}_{g,k}$ corresponde al versor de aceleración en base a los datos del giróscopo computada anteriormente. Se obtiene, entonces, $\bm{\epsilon}_{g}$ minimizando la Expresión anterior utilizando, por ejemplo, el algoritmo de Levenberg-Marquardt.

\subparagraph{Ecuación diferencial ordinaria de la velocidad angular}
Si se tiene un vector $\bm{r}$ de longitud constante, la velocidad angular del mismo en base a las leyes físicas puede obtenerse realizando la derivada del mismo
\begin{equation}
    \frac{d\bm{r}}{dt} = \bm{\omega}\times \bm{r}
\end{equation}
siendo $\times$ el producto externo. Como para el caso evaluado la velocidad angular es perpendicular al vector $\bm{r}$ (producto intero $\bm{\omega}\cdot\bm{r} = 0$), tal como puede verse en la Figura \ref{fig:angularvelocity}. Por lo tanto, puede escribirse a la misma en la forma de cuaterniones
\begin{equation}
    \frac{d\bm{r}}{dt} = \bm{\omega} \otimes \bm{r}
\end{equation}
teniendo en cuenta que tanto $\bm{\omega}$ como $\bm{r}$ se obtienen mediante la Expresión (\ref{eq:vectorquaternionform}).

\textbf{REVISAR SI ESTA BIEN EL GRAFICO, LA IDEA !!!}
\begin{figure}[!ht]
    \centering
    \includegraphics[width=0.5\textwidth]{Img/AngularVelocity.png}
    \caption{Velocidad angular respecto al vector}
    \label{fig:angularvelocity}
\end{figure}

Ahora, siendo que $\bm{r}$ puede definirse mediante la Expresión (\ref{eq:quaternionvectorrotation}), y teniendo en cuenta las propiedades de los cuaterniones
\begin{align}
    \bm{r} &= \bm{q}\otimes \bm{r}_0 \otimes \bm{q}^{-1} \\
    \frac{d\bm{r}}{dt} &= \frac{d}{dt}\left[\bm{q}\otimes\bm{r}_0\otimes\bm{q}^{-1}\right] \\
    &= \dot{\bm{q}}\otimes\bm{r}_0\otimes\bm{q}^{-1} + \bm{q}\otimes\bm{r}_0\otimes\dot{\bm{q}^{-1}} = \bm{\omega}\otimes\bm{r}
\end{align}

Como $\dot{\bm{q}^{-1}}$ puede generar problemas, es posible obtenerlo realizando la derivada del conjugado del cuaternión con su conjugado
\begin{align}
    \frac{d}{dt}\left(\bm{q}\otimes\bm{q}^{-1}\right) &= \frac{d}{dt}1 \\
    \dot{\bm{q}}\otimes\bm{q}^{-1} + \bm{q}\otimes\dot{\bm{q}^{-1}} &= 0
\end{align}
entonces
\begin{equation}
    \dot{\bm{q}^{-1}} = -\bm{q}^{-1}\otimes\dot{\bm{q}}\otimes\bm{q}^{-1}
\end{equation}
sustituyendo y poniéndolo en función de $\bm{r}$
\begin{equation}
        \frac{d}{dt}\left[\bm{q}\otimes\bm{r}_0\otimes\bm{q}^{-1}\right] = \dot{\bm{q}}\otimes\bm{q}^{-1}\otimes\bm{r} - \bm{r}\otimes\dot{\bm{q}}\otimes\bm{q}^{-1}
\end{equation}
donde dicha Expresión tiene la forma de la operación conmutador $\left[\bm{p},\bm{q}\right]$. Finalmente, puede llegarse a la ecuación diferencial
\begin{align}
    \dot{\bm{q}}\otimes\bm{q}^{-1}\otimes\bm{r} - \bm{r}\otimes\dot{\bm{q}}\otimes\bm{q}^{-1} &= \bm{\omega}\otimes\bm{r} \\
    \left[\dot{\bm{q}}\otimes\bm{q}^{-1},\bm{r}\right] &= \bm{\omega}\otimes\bm{r} \\
    2\dot{\bm{q}}\otimes\bm{q}^{-1}\otimes\bm{r} &= \bm{\omega}\otimes\bm{r} \\
    \dot{\bm{q}} &= \frac{1}{2}\bm{\omega}\otimes\bm{q}
\end{align}
donde $\bm{\omega}(t)$ es la velocidad angular en el marco fijo global. En muchos casos es útil expresar la Expresión () en base a la velocidad angular en el marco del sensor. La velocidad angular en este marco es la velocidad angular global rotada en el marco del cuerpo, dado por la Expresión (\ref{eq:quaternionvectorrotation}). En consecuencia,
\begin{equation}
    \dot{\bm{q}} = \frac{1}{2}\bm{q}\otimes\bm{\omega}^s\otimes\bm{q}^{-1}\otimes\bm{q}
\end{equation}
y recordando la Propiedad (\ref{eq:quaternioninverse})
\begin{equation}
    \dot{\bm{q}} = \frac{1}{2}\bm{q}\otimes\bm{\omega}^s 
\end{equation}
Finalmente, en base a la Expresión (\ref{eq:quaternionproductmatrixright}), se llega a
\begin{equation}
    \dot{\bm{q}} = \frac{1}{2}\bm{\Omega}(\bm{\omega}(t))\bm{q}
    \label{eq:edoquaternion}
\end{equation}
siendo $\bm{\Omega}(\bm{\omega}(t))$ la matriz de rotación en base a la velocidad angular obtenida del giróscopo
\begin{equation}
    \bm{\Omega}(\bm{\omega}(t)) =
    \begin{bmatrix}
        0 & -\bm{\omega} \\
        \bm{\omega}^T & -\left[\bm{\omega}\right]_\times
    \end{bmatrix}
    =
    \begin{bmatrix}
        0 & -\omega_x & -\omega_y & -\omega_z \\
        \omega_x & 0 & \omega_z & -\omega_y \\
        \omega_y & -\omega_z & 0 & \omega_x \\
        \omega_z & \omega_y & -\omega_x & 0
    \end{bmatrix}
\end{equation}

\paragraph{Magnetómetro}
Para la calibración del magnetómetro, en cambio, los enfoques tradicionales suponen que hay un sensor de referencia disponible que puede proporcionar información de rumbo precisa. Un ejemplo bien conocido de esto es el balanceo de la brújula [3]. Sin embargo, para permitir que cualquier usuario realice la calibración, se han desarrollado una gran cantidad de enfoques que eliminan la necesidad de una fuente de información de orientación. Una clase de estos algoritmos de calibración de magnetómetro se enfoca en minimizar la diferencia entre la magnitud del campo magnético medido y la del campo magnético local[4]. Este enfoque también se conoce como verificación escalar[5]. Otra clase formula el problema de calibración como un problema de ajuste de elipsoide, es decir, como el problema de mapear un elipsoide de datos a una esfera[6] - [8]. El beneficio de usar esta formulación es que existe una vasta literatura sobre cómo resolver problemas de ajuste de elipsoides,[9][10]. Fuera de estas dos clases, también está disponible un gran número de otros enfoques de calibración[11], donde se consideran diferentes formulaciones del problema de calibración en términos de un problema de máxima verosimilitud. La ventaja de estos métodos es que utilizan sólo la información provista por el magnetómetro.

Sin embargo, si lo que se busca es calibrar un magnetómetro para mejorar la estimación del rumbo en combinación con sensores inerciales, la alineación de los ejes sensoriales de los sensores de inercia y el magnetómetro es crucial en este caso. Se puede ver que esta alineación determina la orientación de la esfera azul de los datos del magnetómetro calibrado en la Fig. 1. Los algoritmos que solo usan datos del magnetómetro pueden asignar el elipsoide rojo de los datos a una esfera, pero sin información adicional, la rotación de esta esfera permanece desconocido.

Varios enfoques más modernos incluyen un segundo paso en el algoritmo de calibración para determinar la desalineación[6], [12] - [14] entre diferentes ejes del sensor. Una opción común para alinear los ejes del sensor de inercia y el magnetómetro es utilizar mediciones de acelerómetro a partir de períodos de aceleraciones bastante pequeñas [12], [13]. La desventaja de este enfoque es que se debe determinar un umbral para usar las mediciones del acelerómetro. Además, se omiten los datos del giroscopio. En [15], por otro lado, el problema se reformula en términos del cambio de orientación, lo que permite el uso directo de los datos del giroscopio. \textbf{[DE KOK 2014]}


\subsection{Encoders}
El objetivo de los encoders es medir las rotaciones relativas de la rueda a la que están asociados. Los encoders normalmente se fijan a la salida del motor o de la caja de cambios, con opciones de diseño que a menudo cambian la resolución angular con la máxima velocidad. Los efectos de resolución y muestreo a menudo producen mediciones ruidosas que requieren filtrado digital. La odometría basada en ruedas supone que no hay deslizamiento entre las ruedas y el terreno. En la práctica, los robots móviles con frecuencia superan la fricción estática y rodante, y cuando una o más ruedas giran o se deslizan, pueden ocurrir grandes errores de odometría. La pendiente del terreno, la fricción y otras propiedades pueden variar, incluso entre ruedas individuales. Los robots de accionamiento diferencial con cuatro ruedas a menudo experimentan grandes errores de odometría cuando giran, ya que se requiere un deslizamiento de las ruedas para girar [150].

\subsection{GPS}
El GPS es un sistema de navegación georreferenciado basado en satélites mediante los cuales estima la latitud, longitud y altitud del objeto en cuestión. Debido a esto, el mismo tiene su campo de aplicación principalmente en ambientes al aire libre, donde mediante la triangulación entre 4 o más satélites puede determinar la ubicación del objeto en cuestión. Debido a esto y a que, entre otros factores, su principio de funcionamiento se basa en medir el tiempo que tardó la señal disparada por un satélite en ser recibida por el propio sensor, el mismo cuenta con precisiones variables dependiendo de la posición de los satélites [162]. Estos sesgos y errores de grandes pasos hacen que la navegación robótica con GPS sea problemática, especialmente con grandes equipos de robots que operan en entornos urbanos y durante muchas horas. [CARLSON2010]
\fi

\ifimagenes
\else
    \newpage
    
\section{Simultaneous Localization And Mapping}
Esta sección describe los conocimientos básicos respecto a la Localización y Mapeo Simultáneos (SLAM), a la vez de introducir parte de los algoritmos y sensores mencionados en secciones anteriores.

\subsection{Introducción al SLAM}
El problema de localización y mapeo simultáneos (SLAM) se basa en el proceso de un robot que construye un mapa de su entorno desconocido (\textit{mapping}) mientras este lo explora conociendo su ubicación en el mismo (\textit{localization}). Dicho problema puede ser expresado como el dilema del huevo o la gallina, ya que para conocer la ubicación del robot, es necesario determinar el mapa que lo rodea, sin embargo, para que el mismo pueda estimar el mapa en el cual se encuentra, necesita primero conocer su ubicación dentro de ese entorno. 

A partir de la detección y seguimiento de marcas naturales del ambiente (\textit{landmarks}), las técnicas de SLAM  estiman  tanto  la  posición  del  robot  como  la  ubicación  de las marcas en el entorno. El mapa se construye con las estimaciones de las posiciones  de  dichas  marcas,  las  cuales  van  siendo ajustadas  a  medida  que  son observadas desde distintas posiciones. Es necesario que la localización sea precisa, ya que si la misma es inexacta generará problemas a la hora de reconstruir el mapa. Es por esto que el mapeo y la localización dependen uno del otro, y se ejecutan \textit{simultáneamente} de forma entrelazada.

El SLAM es un problema difícil ya que, debido al ruido de los sensores utilizados, ninguna de las mediciones es perfecta. Esto significa que ni el movimiento del robot ni la estructura del entorno se conocen de manera absolutamente precisa, sino solo hasta cierto grado de incertidumbre. Con el fin de hacer frente a estas incertidumbres, el SLAM generalmente se entiende y se aborda mediante técnicas y modelos probabilísticos. Las diferentes formas en que se representan las funciones de densidad probabilística constituyen las diferencias en cada enfoque. 
\textbf{MIRAR LO COMENTADO}
% Muchas de ellas utilizan filtros de Bayes, tal como lo es el Filtro de Kalman Extendido (EKF) 
% \begin{large}
% [DURRANT-WHYTE2006]
% [KALMAN1960]
% \end{large}, o en versiones más modernas, por ejemplo, el Fast-SLAM
% \begin{large}
% [PONER FASTSLAM BIEN]
% \end{large}. 
Sin embargo, el uso de enfoques de estimación de estado para describir el problema de SLAM implica resolver una serie de inconvenientes que se generan a partir del mismo, siendo de los más destacados la \textit{asociación de datos}\textbf{[ref 103 y 104 de reid2016]} y el \textit{cierre de ciclo}.

\subsubsection{Asociación de datos}
Uno de los problemas comunes dentro del SLAM corresponde a la asociación de datos (\textit{data asociation}), el cual es el proceso de asociar los datos de los sensores utilizados con las características del ambiente, generando de esta forma los landmarks. En otras palabras, se busca asociar la medición del sensor con alguno de los \textit{n} landmarks ya extraídos y, en caso de que no pertenezca a ninguno de ellos, se generará un landmark nuevo \textit{n+1}. Es importante poder discernir de cuál landmark se trata o si es en efecto uno nuevo, ya que debido al ruido inherente de los sensores puede que caiga entre dos de ellos, como puede observarse en la Figura \ref{fig:dataassociation}, haciendo que pueda tender equívocamente al incorrecto, o creer que no se tiene dicho landmark cuando en realidad era uno ya medido.

\begin{figure}[!ht]
    \centering
    \includegraphics[width=\textwidth]{Img/DataAssociation.png}
    \caption{El problema de la asociación de datos: dificultad de discernir si el landmark detectado (rojo) corresponde a alguno de los ya existentes, o si en realidad es uno del que no se tuvo registro anteriormente}
    \label{fig:dataassociation}
\end{figure}

\subsubsection{Online SLAM y full SLAM}
Los mapas SLAM se construyen a partir de millones de lecturas de sensores, que se comparan entre sí en un paso de asociación de datos que depende completamente de las estimaciones de pose actuales. Para este caso, en el que tanto la pose como el mapa se expresan sólo en base a las mediciones del paso de tiempo actual, el problema se lo conoce como \textit{online SLAM} y se expresa mediante el posterior
% \bm{p}_t es la pose y m el mapa. z es la medicion y u la variable de control
% seria el maximum likelihood?
\begin{equation}
    p(\bm{p},\bm{m}|\bm{y}_{0:t},\bm{u}_{0:t})
\end{equation}
siendo $\bm{p}$ la pose, $\bm{m}$ el mapa, $\bm{y}$ las mediciones y $\bm{u}$ las variables de control.

En cambio, la evaluación y la reevaluación de estas asociaciones de datos, al mismo tiempo que se calcula y actualiza \textit{todo el historial} de poses del robot, describe el problema de \textit{full SLAM} \textbf{[ref 6 de reid2016]}
\begin{equation}
    p(\bm{p}_{0:t},\bm{m}|\bm{y}_{0:t},\bm{u}_{0:t})
\end{equation}

Si bien este último no puede resolverse para entornos no triviales, los enfoques descritos en la literatura a menudo producen resultados útiles en entornos del mundo real. En concreto, con el fin de reducir la complejidad del algoritmo se toman como supuestos comunes que el \textit{entorno es estático}, es decir, que el mapa no varía con el tiempo, y por segundo que las trayectorias de los robots \textit{pueden predecirse}, logrando así que los modelos de movimiento puedan predecir dónde es probable que esté un robot, permitiendo que la búsqueda de asociaciones de datos comience cerca del óptimo global.

\subsubsection{Cierre de ciclo}
\textbf{[REFS DE REID Y VER CASTRO]}
Si bien lo que se busca es evitar errores principalmente de localización, debido al ruido inherente de los sensores utilizados como se mencionó anteriormente, los algoritmos de SLAM no pueden estimar exactamente tanto la ubicación del robot como el mapa que lo rodea, sino hasta con cierto grado de incertidumbre. Como se ejecuta el algoritmo de manera secuencial, el error se irá acumulando a través del tiempo, resultando en una pésima estimación luego de varias corridas. Una forma de poder mitigar en parte estos errores es mediante el reconocimiento de una posición en la que el robot ya estuvo anteriormente. Tomar conocimiento de que se ha efectuado un ciclo en la trayectoria permite calcular el error cometido en la estimación de la posición y da origen a una serie de procesos que permiten corregir tanto la localización actual del robot como el mapa hasta ese momento construido. Este reconocimiento se lo conoce como \textit{cierre de ciclo} (\textit{loop closure}), el cual puede verse en la Figura \ref{fig:poseloopclcorr}.


\begin{figure}[!ht]
    \centering
    \includegraphics[width=\textwidth]{Img/Pose_LoopClosureCorr.png}
    \caption{Estimación de pose y su correción mediante el loop closure}
    \label{fig:poseloopclcorr}
\end{figure}

\subsection{Mapas}
\textbf{DE Suenderhauf 2012}
En base a los sensores extereoceptivos utilizados además del entorno en el que se encuentran los robots y tareas que tengan que realizar, los mapas creados pueden variar considerablemente. \textbf{THRUN 2002, SICILIANO 2008, THRUN 2005}

\subsubsection{Occupancy Grid Maps}
El concepto de esta técnica es la de discretizar el ambiente a mapear, utilizando pequeñas celdas o regiones discretas $m_i$, como puede observarse en la Figura \ref{fig:occupancygridmaps}.a. A cada una de estas celdas se le asocia un valor que expresa la probabilidad
\begin{equation}
    p(m_i|\bm{y}_{0:t},\bm{x}_{0:t})
\end{equation}
de que dicha celda esté ocupada por un obstáculo, dados todos los datos de los sensores hasta el momento ($\bm{y}_{0:t}$) y todas las poses del robot ($\bm{x}_{0:t}$). En caso de que se desconozca el valor de dichas celdas, el valor por defecto de las mismas es de 0.5. 

\begin{figure}[!ht]
    \centering
    \subfloat[Representación de cada celda en base a si la misma se encuentra ocupada, libre o no se tiene información al respecto]{{\includegraphics[width=0.45\textwidth]{Img/OccupancyGrids.png}}}%
    \qquad
    \subfloat[Mapa generado por un LIDAR]{{\includegraphics[width=0.45\textwidth]{Img/LIDAROccupancyGridMap.png}}}%
    \caption{Occupancy Grid Maps}
    \label{fig:occupancygridmaps}
\end{figure}


Un ejemplo de un mapa generado por un LIDAR 2D puede observarse en la Figura \ref{fig:occupancygridmaps}.b, donde los cuadros blancos corresponden a celdas libres, los negros a las ocupadas y, las grises, a las que no se tiene información alguna. A su vez, estos mapas pueden llevarse a la espacio tridimensional para representar datos provenientes de, por ejemplo, LIDAR 3D o cámaras.

\subsubsection{Mapas de características}
A diferencia de los Occupancy Grid Maps, los cuales generan un mapa denso del ambiente, los \textit{mapas de características} contienen sólo la posición de distintos landmarks o características del ambiente, resultando en un mapa disperso. Un ejemplo del mismo puede verse en la Figura \ref{fig:dataassociation} \textbf{O PONGO OTRA IMG?}. Dependiendo del principio de medición del sensor que recolecta estos landmarks, existen diferentes tipos de ellos
\begin{itemize}
    \item \textit{El Rango y el rumbo} se conocen, tal como es el caso de las cámaras estéreo o las cámaras RGB-D, en el que la ubicación del landmark respecto al sensor se encuentra bien definida, resultando ser el más sencillo de los tres tipos. En la Figura \ref{fig:landmarktypes}.a puede observarse al mismo.
    \item \textit{Sólo el rango} es conocido, como se observa en la Figura \ref{fig:landmarktypes}.b, haciendo que sea necesario triangular, por ejemplo, para la señal de WiFi, entre la fuerza de la señal recibida de diferentes routers y así conseguir la ubicación del robot en base a estos.
    \item \textit{Sólo el rumbo} es conocido, observado en la Figura \ref{fig:landmarktypes}.c, como es el caso de las cámaras monoculares, haciendo difícil medir la profundidad de cada punto.
\end{itemize}

\begin{figure}[!ht]
    \centering
    \subfloat[Se tiene rango y rumbo]{{\includegraphics[width=0.3\textwidth]{Img/LandmarkRangeBearing.png}}}%
    \qquad
    \subfloat[Se tiene sólo el rango]{{\includegraphics[width=0.3\textwidth]{Img/LandmarkRange.png}}}%
    \subfloat[Se tiene sólo el rumbo]{{\includegraphics[width=0.3\textwidth]{Img/LandmarkBearing.png}}}%
    \caption{Tipos de landmarks}
    \label{fig:landmarktypes}
\end{figure}

\subsubsection{Grafos de pose}
Los grafos de pose representan la trayectoria del robot como una estructura grafos donde los nodos representan las poses del robot y los bordes entre los nodos representan las conexiones espaciales entre estas poses. Los bordes contienen, por ejemplo, información de odometría o expresan cierres de ciclo. En la Figura puede verse el esquema de un grafo de pose, donde la trayectoria del robot es representada por los grafos donde los nodos son las poses del robot en un punto concreto de tiempo.
\begin{figure}[!ht]
    \centering
    \includegraphics[width=\textwidth]{Img/PoseGraphStructure.png}
    \caption{Esquema de un grafo de pose}
    \label{fig:posegraphstructure}
\end{figure}

Si bien los landmarks explícitos u otra información sobre el medio ambiente no son parte del grafo de pose en sí, dicha información se puede adjuntar a los nodos en el grafo, lo que permite que el grafo de pose exprese occupancy grids en 2D o 3D o mapas de características y cualquier combinación de ellos.

\subsection{Algoritmos de SLAM}
Puede decirse que un algoritmo de SLAM consiste en tres operaciones básicas, las cuales se repiten en cada paso de tiempo
\begin{itemize}
    \item El robot se mueve por el entorno, hasta alcanzar un nuevo punto de vista de la escena, que debido al ruido inherente de los sensores ocasiona que aumente la incertidumbre en la localización del mismo.
    \item El robot descubre landmarks en el ambiente, los que requieren ser incorporados al mapa.
    \item El robot observa el mapa que fue mapeado anteriormente y lo utiliza para corregir tanto la localización de dicho landmark como la del propio robot.
\end{itemize}

Si bien existen numerosos algoritmos para realizar el problema de SLAM, a continuación se describen algunos de los enfoques principales que se encuentran en la literatura respecto al tema en cuestión.

\subsubsection{EKF-SLAM}
El enfoque del EKF para solucionar el problema de SLAM se trata de un problema orientado al online SLAM mediante el uso de datos de sensores recopilados a partir del movimiento y la rotación del robot. Esto se puede hacer, por ejemplo, utilizando encoders y una IMU. Además, también es necesario recopilar información del entorno mediante, por ejemplo, el uso de un LIDAR. Con estos datos, el algoritmo realiza un seguimiento del lugar donde probablemente se ubica el robot dentro de un mapa, así como un seguimiento de los puntos de referencia específicos observados. Es por esto que se utiliza un mapa de características para este caso.

Al considerar un mapa estático, el único factor variante en el tiempo es el movimiento del robot. Por lo tanto, se tienen diferentes comportamientos para cada parte del vector de estados. El mismo en este caso se define como
\begin{equation}
    \bm{x} = 
    \begin{bmatrix}
        \bm{p} \\
        \bm{\mathcal{M}}
    \end{bmatrix}
\end{equation}
donde $\bm{p}$ corresponde a la pose del robot\footnote{Por lo general se suele incorporar la velocidad, ya que se utilizan las ecuaciones de la física clásica. Para la orientación puede utilizarse la forma de cuaterniones mediante la Expresión (\ref{eq:edoquaternion}).} y $\bm{\mathcal{M}}=(\mathcal{L}_1,...,\mathcal{L}_n)$ corresponde a los estados de los landmarks, siendo $n$ la cantidad de ellos detectados. En base a este, el resto de los parámetros del filtro de Kalman Extendido se modificarán respecto al visto en la Sección \ref{sec:stateestimation}\textbf{[SOLA 2014]}. Puede verse que a medida que el mapa aumenta, el costo computacional se incrementará, ya que se tendrán cada vez más landmarks.

Además de la complejidad cuadrática, este filtro cuando se aplica al problema SLAM tiene como desventaja sustancial \textbf{[Csorba 1997 DE SU - FAST-SLAM]}
la sensibilidad a fallas que presenta en las asociaciones de datos. Este problema con el EKF se aplica en situaciones en las que se desconocen las asociaciones de datos. El EKF mantiene una única hipótesis de asociación de datos por observación, típicamente elegida usando una heurística de máxima verosimilitud. Si la asociación de datos elegida por esta heurística es incorrecta, el efecto de incorporar esta observación en el EKF nunca se puede eliminar.

\subsubsection{FastSLAM}


\subsubsection{GraphSLAM}
Utilizando un grafo de poses, el GraphSLAM se orienta a un problema de full SLAM que consta de dos restricciones. Primero, necesita de datos de odometría (que suelen ser las entradas de control $\bm{u}$) que conecten los estados de las poses $\bm{x}_k$ y $\bm{x}_{k+1}$ como puede ser mediante los datos de un LIDAR o la odometría propia de los encoders
\begin{equation}
    \bm{x}_{k+1} \sim \mathcal{N}(f(\bm{x}_{k},\bm{u}_k),\bm{R}_k)
\end{equation}
donde $\bm{\Sigma}_k$ corresponde a la matriz covarianza en el paso de tiempo $k$.

Segundo, para poder detectar un cierre de ciclo, dos poses, que pueden no ser contiguas, se conectan mediante una Gaussiana,
\begin{equation}
    \bm{x}_j \sim \mathcal{N}(f(\bm{x}_k,\bm{c}_{kj}), \bm{Q}_{kj})
\end{equation}
siendo $\bm{c}_{kj}$ el cierre de ciclo.

La solución para este tipo de SLAM se basa en el método del máximo a posteriori (MAP), siendo $\bm{x}^*$ el punto donde la distribución tiene su máximo
\begin{equation}
    \bm{x}^* = argmax_x\ p(\bm{x}|\bm{u})
\end{equation}

Siendo que tanto la odometría como el cierre de ciclo cuentan con distribuciones Gaussianas independientes, el posterior puede factorizarse como
\begin{equation}
    p(\bm{x}|\bm{u},\bm{c}) \propto \underbrace{\prod_k p(\bm{x}_{k+1}|\bm{x}_{k},\bm{u}_k)}_{\text{\parbox{10em}{\centering Restricciones de odometría}}}\ \underbrace{\prod_{kj}p(\bm{x}_j|\bm{x}_k,\bm{c}_{kj})}_{\text{\parbox{10em}{\centering Restricciones de cierres de ciclo}}}
\end{equation}
que, aplicando el logaritmo como en la Expresión (\ref{eq:mlewithlog}), y teniendo en cuenta la analogía con los cuadrados mínimos ponderados
\begin{align}
    \bm{x}^* &= argmax_x\ p(\bm{x}|\bm{u},\bm{c}) \\
    &= argmin_x\ -log(\bm{x}|\bm{u},\bm{c}) \\
     &= argmin_x\ \bm{e}_k^T\bm{R}_k^{-1}\bm{e}_k + \bm{e}_{kj}^T\bm{Q}_{kj}\bm{e}_{kj} \\
     &= argmin_x\ \underbrace{\sum_{i} ||f(\bm{x}_i,\bm{u}_i)-\bm{x}_{i+1})||^2_{\bm{R}_i}}_{\text{{\centering Restricciones de odometría}}} + \underbrace{\sum_{kj} ||f(\bm{x}_i,\bm{c}_{kj}) - \bm{x}_{j}||^2_{\bm{Q}_{kj}}}_{\text{{\centering Restricciones de cierres de ciclo}}}
\end{align}
siendo $||a-b||^2_{\bm{O}} = (a - b)^T\bm{O}^{-1}(a-b)$ la definición de la distancia Mahalanobis. Esta puede minimizarse con algunas de las regresiones vistas en la Sección \ref{sec:regressionanalysis}, dependiendo de la morfología de la señal $f$.


\fi
\newpage
\ifimagenes
\section{Marco teórico}
\else
\section{Reconstrucción del entorno}
\fi
\label{sec:marcoteorico}
En esta sección se presentan distintos métodos para poder realizar una reconstrucción del entorno en base a las nubes de puntos, además de dar una introducción a la librería de nube de puntos (PCL) y su vínculo con ROS, utilizados para el desarrollo del trabajo.

\ifimagenes
\subsection{Reconstrucción del entorno}
\else
\fi
En base a los datos provistos por los sensores exteroceptivos (LIDAR, cámaras, entre otros), el objetivo principal del trabajo consiste en recolectar dicha información para poder reconstruir el entorno en el cual se encuentra sumergido el robot, además de poder estimar su posición. Sin embargo, estos tipos de sensores proveen un campo de visión limitado, por lo que no es posible describir el mundo real como un todo en base a una sola medición de estos sensores, sino que solo puede mencionarse a una pequeña porción del mismo, denominada \textit{escena}.

A su vez, el tipo de dato que se tome de la escena dependerá del tipo de sensor extereoceptivo que los provea. Por ejemplo, una cámara monocular es capaz de otorgar imágenes 2D, mientras que con una cámara estéreo se consigue una nube de puntos 
\ifimagenes
    \ifimagenespaper
tridimensional.
    \else
tridimensional, tal como se describe en el apéndice.
    \fi
\fi

\subsection{Point cloud}
Una \textit{nube de puntos}, o de su terminología en inglés, \textit{point cloud}, se utiliza para describir a un conjunto de puntos de datos en un espacio dado. Las nubes de puntos 3D, por ejemplo, se tratan de un conjunto de puntos tridimensionales que se caracterizan por tener principalmente coordenadas espaciales XYZ, y opcionalmente pueden dar información de intensidad, color, entre otras.

Como se mencionaba anteriormente, la nube de puntos de la escena adquirida dependerá del principio de medición del sensor utilizado. En concreto, puede categorizarse en: \cite{weinmann2016}
\begin{itemize}
    \item \textit{Técnicas pasivas}, donde la luz ambiental se encuentra presente, permitiendo utilizar sensores como cámaras estéreo para obtener las imágenes propias del entorno.
    \item \textit{Técnicas activas} donde los sensores emiten radiaciones electromagnéticas, tal como es el caso de los LIDAR 3D o las cámaras infrarrojas.
\end{itemize}

Dependiendo de la técnica de adquisición involucrada y el dispositivo utilizado, los datos de la nube de puntos adquiridos pueden corromperse con más o menos ruido y, además de la información espacial en forma de coordenadas XYZ, los atributos de los puntos respectivos, como la información de color o intensidad también puede adquirirse, como se mencionaba anteriormente.

\subsection{Point Cloud Library (PCL)}
La \textit{Point Cloud Library (PCL)} se trata de un proyecto abierto desarrollado en \textit{C++} que ofrece herramientas para procesar imágenes o nubes de puntos tanto 2D como 3D. El framework PCL contiene numerosos algoritmos que realizan filtrado, estimación de características, reconstrucción de superficies, registro, ajuste de modelos y segmentación. Con estos métodos, es posible procesar la nube de puntos, extraer keypoints para reconocer objetos en el mundo en función de su apariencia geométrica, crear superficies a partir de las nubes de puntos y visualizarlas.

En general, PCL contiene una estructura de datos muy importante, que es $pcl::PointCloud$. Esta estructura de datos está diseñada como una clase de template que toma el tipo de punto que se utilizará como parámetro. Como resultado de esto, la clase de nube de puntos no es mucho más que un contenedor de puntos que incluye toda la información común requerida por todas las nubes de puntos independientemente de su tipo de punto. Algunos de los tipos de puntos más utilizados son
\begin{itemize}
    \item $pcl::PointXYZ$, el cual es el más simple que posee la librería, almacenando la información en XYZ únicamente
    \item $pcl::PointXYZRGB$, almacena además de la posición XYZ, el color en formato RGB de cada punto.
    \item $pcl::PointNormal$, representa la superficie normal en un punto dado y la medida de su curvatura, además de las coordenadas XYZ.
    \item $pcl::PointXYZI$, que a las coordenadas XYZ les asocia también información de intensidad del punto.
\end{itemize}

Debido al gran número de tipos de puntos que existen en la librería, cada algoritmo implementado en la misma refiere a una clase perteneciente a una jerarquía de clases con puntos en común específicos. Gracias a ello, dichos algoritmos pueden ser parametrizados en base a lo que necesite el usuario. 

\ifimagenes
    \ifimagenespaper
    \else
    \subparagraph{Interfaz con ROS}
    \fi
\fi
La interfaz PCL para ROS proporciona los medios necesarios para comunicar las estructuras de datos PCL a través del sistema de comunicación basado en mensajes proporcionado por ROS. Para hacerlo, hay varios tipos de mensajes definidos para contener nubes de puntos, así como otros productos de datos de los algoritmos PCL. En combinación con estos tipos de mensajes, también se proporciona un conjunto de funciones de conversión para convertir de tipos de datos PCL nativos en mensajes. Uno de los más importantes tipos de mensajes de ROS es el $sensor_msgs::PointCloud2$, el cual contiene una colección de puntos N-dimensionales, que pueden contener información adicional como normales, intensidad, entre otros.

Para relacionar ambos mundos, ROS cuenta con el paquete \textit{pcl\_conversions}\footnote{http://wiki.ros.org/pcl\_conversions}. Por ejemplo, en el Codigo \ref{lst:pc2pcl} se convierte un mensaje del tipo \lstinline{sensor_msgs::PointCloud2} a \lstinline{pcl::PointCloud<pcl::PointXYZRGB>}.

\begin{lstlisting}[caption=Conversión de mensaje ROS a PCL, label=lst:pc2pcl]
  // ROS PointCloud2
  sensor_msgs::PointCloud2 &pc2;
  
  // Output Cloud
  pcl::PointCloud<pcl::PointXYZRGB>::Ptr cloud_out(new pcl::PointCloud<pcl::PointXYZRGB>);
  // PCL PointCloud2
  pcl::PCLPointCloud2 pcl_pc2;   

  // Convert from ROS message to PCL data
  pcl_conversions::toPCL(pc2, pcl_pc2);
  pcl::fromPCLPointCloud2(pcl_pc2, *cloud_out);
\end{lstlisting}


\subsection{Adquisición de la nube de puntos}
Como se menciona en el documento, dependiendo del sensor utilizado para obtener los datos, se conseguirán las nubes de puntos asociadas dependiendo del principio de 
\ifimagenes
medición. A continuación se menciona en primera instancia una librería usada comúnmente en este tipo de aplicaciones (OpenCV), además de los sensores utilizados generalmente en este tipo de aplicaciones.

\subsubsection{OpenCV}
OpenCV\footnote{http://opencv.org} es una librería open source multiplataforma que busca brindar una infraestructura de visión artificial fácil de usar y que pueda utilizarse para aplicaciones de tiempo real, por lo que hace hincapié en la optimización. Para ello, OpenCV cuenta con un set de más de 500 funciones \cite{kaehler2017} que abarcan muchas áreas en visión, tales como calibración de cámaras, visión estéreo y robótica.

\subsubsection{Cámara estéreo}
Con el fin de poder simular el efecto de la percepción humana, las cámaras estéreo se ubican separadas una distancia fija y, por lo general, alineadas horizontalmente, como se observa la Bumblebee 2 en la Figura \ref{fig:stereoandrgbdcameras}.a. Al tener dos fuentes de imágenes distintas y ubicadas en una posición relativa conocida, es posible a partir de las mismas determinar la profundidad en la que se encuentran los objetos que ven entre ambas. 

Si bien se puede realizar con una cámara monocular, la técnica cuenta con mayor complejidad y con ciertas restricciones\footnote{Por ejemplo, sería necesario conocer una cierta cantidad de puntos claves del objeto para que, estando el mismo en una pose distinta a la analizada a priori, sea posible determinar la pose del objeto mediante estos puntos claves.}, aunque puede solventarse mediante el agregado de sensores que utilicen otro principio de medición\footnote{Las cámaras RGB-D, por ejemplo, cuentan con una cámara y sensores de distancia para determinar la ubicación de determinados píxeles.}.

Para poder conseguir una nube de puntos mediante el uso de dos cámaras, deben seguirse una serie de pasos \cite{kaehler2017}
\begin{enumerate}
    \item \textit{Remover las distorsiones} de la lente.
    \item Ajustar las distancias y ángulos entre las cámaras, conocido como \textit{rectificación}.
    \item Encontrar las mismas características en ambas cámaras, proceso conocido como \textit{correspondencia}. A partir de este se consigue un \textit{mapa de disparidad} (o \textit{disparity map}), donde las disparidades son las diferencias en la coordenada x de los planos de la imagen de la misma característica vista en las cámaras izquierda y dercha.
    \item Si se conoce el arreglo geométrico de las cámaras, puede cambiarse el mapa de disparidad en distancias por \textit{triangulación}. Este paso es denominado \textit{proyección}, y el resultado es un mapa de profundidad.
\end{enumerate}

Para obtener mejores resultados, es necesario que las cámaras estén sincronizadas entre sí, caso contrario los objetos en movimiento serán un problema.

\begin{figure}
    \centering
    \includegraphics[width=.9\linewidth]{Img/bumblekinect}
    \caption{Cámaras: (a) Bumblebee 2, (b) Microsoft Kinect}
    \label{fig:stereoandrgbdcameras}
\end{figure}

\paragraph{Calibración}
La calibración estéreo depende de hallar la matriz de rotación y la traslación entre las dos cámaras previamente calibradas \cite{kaehler2017}, lo que puede realizarse mediante la función de OpenCV \texttt{cv::stereoCalibrate()}. En el mismo, se busca una sola matriz de rotación y un vector de traslación que relacione la cámara derecha con la cámara izquierda

Para la calibración en ROS se puede utilizar, al igual que en cámaras monoculares, el paquete \texttt{camera\_calibration}, el cual en este caso recibirá parámetros distintos. La interfaz gráfica para la calibración será similar al de cámaras monoculares, con la diferencia que en este caso aparecen ambas cámaras, aunque el principio de calibración del lado usuario es prácticamente el mismo, teniendo en cuenta que ahora en ambas cámaras debe visualizarse el patrón buscado, tal como se observa en la Figura \ref{fig:stereocalibrationchessboardrosexample}.
\begin{figure}
    \centering
    \includegraphics[width=\linewidth]{Img/StereoCalibrationChessboard.png}
    \caption{Calibración estéreo mediante ROS}
    \label{fig:stereocalibrationchessboardrosexample}
\end{figure}


\paragraph{Rectificación estéreo}
Si los planos de ambas cámaras no se encuentran perfectamente alineados, la disparidad estéreo aumenta su complejidad. Para poder mitigar este problema, lo que se busca es reproyectar los planos de la imagen de ambas cámaras para que residan en el mismo plano, conocido como \textit{rectificación estéreo}. Dentro de OpenCV, existen numerosas formas de obtenerlo, como pueden ser los algoritmos de Hartley o de Bouguet. El algoritmo de Hartley permite evitar la calibración de la cámara, en cambio para el segundo es necesaria.

Una vez que se obtienen los términos de la calibración estéreo, se pueden calcular los mapas de rectificación izquierda y derecha por separado utilizando para cada uno \texttt{cv::initUndistortRectifyMap()}.

En ROS, el paquete utilizado para dicha tarea es el denominado \texttt{stereo\_image\_proc}\footnote{\url{http://wiki.ros.org/stereo_image_proc}}, que permite obtener un mapa de profundidades. 

\paragraph{Correspondencia estéreo}
La correspondencia estéreo, esto es, coincidir un punto tridimensional en las vistas de ambas cámaras, requiere que estos puntos de ambas cámaras se solapen. Para ello, OpenCV implementa dos algoritmos distintos para correspondencias, convirtiendo dos imágenes (izquierda y derecha) en una sola imagen de profundidad
\begin{itemize}
    \item \textit{Block matching (BM) algorithm}, implementada mediante \texttt{cv::StereoBM()}, el cual es rápido y efectivo, aunque en escenas de baja textura presenta problemas. 
    \item \textit{Semi-global block matching (SGBM)}, implementada mediante \texttt{cv::StereoSGBM()}, el cual presenta una mayor exactitud que el BM, pero es computacionalmente más demandante.
\end{itemize}

En base a este proceso es que puede obtenerse una \textbf{nube de puntos} del entorno.

Para el caso de ROS, dentro del paquete \texttt{stereo\_image\_proc} se encuentra el ejecutable \texttt{dynamic\_reconfigure}\footnote{\url{http://wiki.ros.org/stereo_image_proc/Tutorials/ChoosingGoodStereoParameters}}, que permite modificar dinámicamente los parámetros de correspondencia, además de elegir el algoritmo a utilizar mediante una interfaz gráfica.

\subsubsection{Cámara RGB-D}
Las cámaras RGB-D\cite{hogman2011} en lugar de poseer dos cámaras monoculares presentan una cámara monocular con la ayuda de sensores infrarrojos para determinar la ubicación de determinados píxeles, siendo un ejemplo conocido de las mismas es la Microsoft Kinect, la cual se observa en la Figura \ref{fig:stereoandrgbdcameras}.b. Para que el mismo pueda obtener la nube de puntos, primero el sensor de distancia proyecta un patrón de moteado infrarrojo. El patrón proyectado es luego capturado por una cámara de infrarrojos en el sensor y comparado parte por parte con patrones de referencia almacenados en el dispositivo. Estos patrones fueron capturados previamente a profundidades conocidas en el proceso de calibración de los mismos. A continuación, el sensor estima la profundidad por píxel en función de los patrones de referencia con los que el patrón proyectado coincide mejor. Los datos de profundidad proporcionados por el sensor de infrarrojos se correlacionan luego con una cámara RGB calibrada. Esto produce una imagen RGB con una profundidad asociada a cada píxel. Una representación unificada popular de estos datos es una nube de puntos: una colección de puntos en un espacio tridimensional, donde cada punto puede tener características adicionales asociadas. Con un sensor RGB-D, el color puede ser una de esas características. Además, las normales de superficie aproximadas también se almacenan a menudo con cada punto en la nube de puntos resultante.
\else
medición, tal como se desarrolló en la Sección \ref{sec:sensors}.
\fi
\subsection{Point cloud registration}
Si bien un primer paso es obtener la nube de puntos en base al sensor utilizado, uno de los principales enfoques utilizados en robots móviles con los mismos es el llamado \textit{registro de nube de puntos} o, de su terminología en inglés, \textit{point cloud registration}. Dicho proceso responde a encontrar una transformación espacial que alinee dos nubes de puntos generalmente contiguas en el tiempo. Los sensores que utilizan normalmente este enfoque son las cámaras estéreo y RGB-D.

En concreto, dada una nube de puntos fuente (tambíen llamada \textit{input} o \textit{source}) $P$ con puntos $p \in P$, y una nube de puntos objetivo $Q$ (llamada \textit{target}) con puntos $q \in Q$, el problema del registro se basa en encontrar correspondencias entre $P$ y $Q$, y estimar una transformación $T$ que, cuando se aplica a $P$, se alinea todos los pares de puntos correspondientes ($p_i \in P$, $q_j \in Q$). Un problema fundamental del registro es que estas correspondencias generalmente no se conocen y deben ser determinadas por el algoritmo de registro. Dadas las correspondencias correctas, hay diferentes formas de calcular la transformación óptima con respecto a la métrica de error utilizada, como se detalla a continuación.

El registro de dos nubes de puntos puede dividirse en una serie de pasos, los cuales se observan en la Figura \ref{fig:registrationpipeline} y conforman el denominado \textit{registration pipeline}. El mismo si bien puede extenderse para un caso más general \cite{garcia2017}, en este caso se da una estructura representativa del trabajo presente, siendo dichos pasos la \textit{selección}, \textit{matcheo}, \textit{rejection} y \textit{alineación}.

\begin{figure}[!ht]
    \centering
    \includegraphics[width=0.45\linewidth]{Img/3DRegistrationPipeline.png}
    \caption{Registration pipeline}
    \label{fig:registrationpipeline}
\end{figure}

En primera instancia, debido a la gran densidad de puntos que presentan las nubes de puntos, además del ruido inherente de los sensores utilizados, es necesario realizar un proceso de \textit{selección} de los datos de interés, no solo para que el \textit{proceso de registro} pueda converger al valor óptimo, sino también para reducir el tiempo que tarda el algoritmo en completar el procesamiento.

Una vez que se tienen los datos seleccionados de cada una de las nubes de puntos a alinear, es necesario encontrar las correspondencias entre las mismas, esto es, encontrar un conjunto de puntos (los cuales suelen ser los \textit{keypoints}) en la nube de puntos fuente que se pueden \textit{identificar como los mismos puntos} en la nube de puntos objetivo.

En el proceso descripto anteriormente, normalmente una gran proporción de las correspondencias consideradas como tales en realidad son desajustes debidos al cambio de punto de vista, oclusión \cite{barazzetti2018}, entre otros. Estos desajustes suelen ser suficientes para arruinar los métodos de estimación tradicionales. Por lo tanto, es necesario eliminar o reducir la influencia indebida de los desajustes. Por ello, luego del pareo de los puntos característicos es necesario \textit{rechazar} las correspondencias para reducir el número de valores atípicos.

Finalmente, con las correspondencias filtradas, se busca la transformación que se ajusta mejor a ambas nubes de puntos mediante un proceso conocido como \textit{alineación}.

% \ifimagenes
% \else
% %% VA O NO VA?????
% A partir de la clasificación anterior, pueden diferenciarse dos tipos de algoritmos de registro, el \textit{registro basado en características (features)}, y los \textit{algoritmos de registro iterativos}, que se observan en la Figura \ref{fig:registrationprocess} y se detallan a continuación.
% \begin{itemize}
%     \item \textit{Registro basado en características (features)}, para calcular alineaciones iniciales, y
%     \item \textit{Algoritmos de registro iterativos}, siguiendo el principio del algoritmo ICP para iterativamente registrar nube de puntos (que ya se encuentran relativamente alineadas).
% \end{itemize}

% \begin{figure}[!ht]
%     \centering
%     \includegraphics[width=\linewidth]{Img/RegistrationProcess.png}
%     \caption{Registration process}
%     \label{fig:registrationprocess}
% \end{figure}

% Para el registro basado en características, los descriptores de características geométricas se calculan y combinan en algún espacio de alta dimensión. Cuanto más descriptivos, únicos y persistentes sean estos descriptores, mayor será la probabilidad de que todas las \textit{coincidencias} (o \textit{correspondencias}) encontradas sean pares de puntos que realmente se correspondan entre sí.

% En el algoritmo ICP de Besl y McKay \cite{besl1992} no se calculan descriptores de características, sino que se considera que los puntos más cercanos en el espacio cartesiano se corresponden entre sí. Se estima una transformación que minimiza las distancias euclidianas entre pares encontrados de puntos más cercanos en el sentido de mínimos cuadrados. El proceso de determinar los puntos correspondientes en los dos conjuntos de datos y calcular la transformación que los alinea se repite iterativamente. Se espera que el conjunto de puntos de origen converja hacia el conjunto de destino a medida que las correspondencias sean cada vez mejores. Simultáneamente, Chen y Medioni \cite{chen1992} formularon un algoritmo similar, pero en lugar de minimizar las distancias euclidianas al cuadrado entre los puntos correspondientes, aplicaron una métrica de error de punto a plano.
% %%%%%
% \fi
A continuación, se desarrollan los pasos nombrados anteriormente.

\subsubsection{Selección}
Como la información suele ser redundate y lo suficientemente densa como para necesitar un gran costo computacional a la hora de procesar los datos, es necesario filtrar las nubes de puntos quitando la información irrelevante, reduciendo así el tiempo de ejecución de los algoritmos. Si bien existen en la literatura muchos criterios para decidir cuales puntos tomar y cuales no, en principio se pueden distinguir dos métodos de reducción de datos, siendo el primero el de extraer automáticamente un pequeño conjunto de keypoints únicos y repetibles, mientras que el otro se basa en muestrear los datos originales con respecto a una distribución objetivo deseada. Mientras que el primero está destinado a la alineación inicial basada en características \cite{merino2016}, el segundo se puede utilizar para reducir de manera eficiente la cantidad de datos para los algoritmos de registro iterativos. PCL implementa varios de estos métodos de muestreo, en particular, el muestreo en el espacio índice (simplemente tomando cada n-ésimo punto), submuestreo uniforme en el espacio 3D de entrada (para capturar mejor las estructuras ambientales detectadas) y muestreo en el espacio de normales de superficie (para muestrear puntos en todas las orientaciones de la superficie). A continuación se presentan dos métodos que suelen emplearse en este tipo de tareas
\begin{itemize}
    \item \textit{Normal space sampling}: Al tratar generalmente con modelos suaves con pequeñas irregularidades (por ejemplo, un plano), el proceso puede resultar en el muestreo de muchos puntos que contienen esencialmente la misma información en términos de vectores normales. Por ello, la estrategia de \textit{normal space sampling} pretende dar uso a esta información \cite{rusinkiewicz2001}, y la misma consiste en, en primera instancia, agrupar puntos en ''cubos'' de acuerdo con los ángulos entre sus vectores normales (considerados en la esfera unitaria) y los ejes de coordenadas, y luego muestrear uniformemente sobre los cubos resultantes, proporcionando un submuestreo de los puntos con más vectores normales "frecuentes". Dentro de PCL, este algoritmo se encuentra implementado en el método \lstinline{pcl::NormalSpaceSampling}.
    \item \textit{Harris 3D}: El operador de Harris \cite{harris1988} se trata de un detector de puntos de interés para imágenes. El método es una técnica popular debido a su fuerte invariancia a la rotación, escala, variación de iluminación, y ruido de imagen \cite{schmid2000}. El detector de Harris se basa en la función de autocorrelación local de una señal, que mide los cambios locales de la señal con parches desplazados una pequeña cantidad en diferentes direcciones. El mismo se ha utilizado en muchas aplicaciones en procesamiento de imágenes y visión artificial por su sencillez y eficiencia. Sin embargo, el problema con los datos 3D es que la topología es arbitraria y no está claro cómo calcular las derivadas. Para solucionar este problema, en \cite{sipiran2011} proponen transformar a los puntos de la nube de la siguiente manera
    \begin{enumerate}
        \item Por cada punto de la nube, se define un \textit{punto vecino} del mismo, y se calcula el centroide de este último
        \item Todos los puntos de la nube se trasladan para que el centroide coincida con el origen de coordenadas
        \item Luego, se calcula un plano de ajuste a los puntos traslada
        \item Se aplica el \textit{análisis de componentes principales} (PCA de sus siglas en inglés) \cite{jollife2016} al conjunto de puntos y se elige el \textit{eigenvector} con el \textit{eigenvalue} asociado más bajo como la normal del plano de ajuste.
        \item Se rotan los puntos hasta que la normal al plano coincide con el eje z.
        \item El plano XY resultante se utiliza para calcular las derivadas. Estas derivadas se calculan utilizando una superficie cuadrática de seis términos (paraboloide) ajustada al conjunto de puntos transformados.
    \end{enumerate}
    Dentro de PCL, en el método \lstinline{pcl::HarrisKeypoint3D} se encuentra una implementación de dicho algoritmo.
\end{itemize}

\subsubsection{Estimación}
La estimación de correspondencias es el proceso de emparejar los puntos $p_i$ desde la nube de puntos fuente $P$ con sus vecinos más cercanos $q_j$ en la nube objetivo $Q$. 

En PCL, con la función \lstinline{pcl::registration::DetermineCorrespondences} se obtiene el conjunto de pares de correspondencia encontrados entre el la nubes de puntos fuente y objetivo. Cada par consta del índice del punto en la nube de origen y el índice de la coincidencia encontrada en la nube de puntos de destino. 

En el caso de datos de entrada provenientes de sensores que cumplen con el modelo de cámara estenopeica \cite{kaehler2017}, el procedimiento de estimación de correspondencia puede acelerarse significativamente con la contraparte de perder algo de precisión. Estos sensores incluyen cámaras RGB-D populares como Microsoft Kinect, y recopilan información del entorno en forma de imágenes de profundidad y color. En árboles PCL para búsquedas de vecinos más cercanos en el espacio 3D, es posible utilizar la naturaleza proyectiva de las imágenes de profundidad para obtener una aproximación razonable. Cada punto de la nube corresponde a un píxel en la imagen de profundidad, lo que permite proyecciones desde puntos de origen en coordenadas mundiales hasta el plano de la cámara del marco de destino mediante el uso de parámetros de cámara intrínsecos y extrínsecos. Este enfoque es rápido, pero impreciso para nubes de puntos con grandes discontinuidades de profundidad o para cuadros que están muy alejados entre sí. Es por eso que se recomienda usar este método solo después de que las dos nubes de puntos se hayan acercado, lo que lo hace bueno para alinear nubes de puntos consecutivas en una secuencia grabada a alta velocidad de cuadros.

%% QUE ONDA CON ESTO???
% Los métodos que extraen pocos keypoints pueden utilizar la fuerza bruta para encontrar correspondencias. Sin embargo, este proceso es computacionalmente costoso. Algunos métodos reducen el tiempo de cálculo minimizando el espacio de búsqueda, como 3D Shape Context [69] que preselecciona los posibles candidatos que satisfacen ciertos criterios y luego aplica fuerza bruta con estos candidatos. No obstante, en la mayoría de las situaciones, se necesitan algoritmos más elaborados para informar los resultados en un período de tiempo razonable.

% Como se necesitan al menos tres puntos no coplanares en cada conjunto para determinar una transformación rígida entre dos conjuntos de puntos 3D sin ambigüedad, el costo asintótico de tales enfoques está en O (n6). En consecuencia, el espacio para navegar en la búsqueda de correspondencias es enorme. Diseñar una estrategia de búsqueda sofisticada que sea capaz de aprovechar la información de detección y descripción tiene el potencial de reducir en gran medida los costos de cálculo y, por lo tanto, aumentar el rango de aplicación de tales algoritmos de registro. Los métodos existentes que implementan estrategias de búsqueda ya logran muy buenos resultados en comparación con los métodos típicos de fuerza bruta.

% A diferencia de los algoritmos de alineación, la finalidad de estas estrategias de coincidencia es lograr solo una alineación aproximada. La idea es identificar la posición arbitraria de las formas de entrada y encontrar las transformaciones entre ellas, lo más rápido posible. La precisión no es, por tanto, el factor más importante. En cambio, la robustez es clave proporcionando garantías para el posterior ajuste fino.
% A continuación se presentan algunos métodos conocidos de la literatura.

% \paragraph{Métodos basados en RANSAC}
% RANdom SAmple and Consesus (RANSAC) es un método iterativo diseñado para encontrar los parámetros de un modelo a partir de un conjunto de datos que contiene valores atípicos (\textit{outliers}). Dada una entrada de datos ruidosos, RANSAC encuentra los parámetros que ajustan los datos de entrada a un modelo dado, descartando los valores atípicos. Este enfoque es la base de una amplia variedad de métodos. Uno de ellos es el enfoque presentado por [34] que se basa en el hecho de que podemos determinar una transformación rígida con solo tres puntos (una base B). La idea es encontrar una base en una de las formas y encontrar la base correspondiente en la otra forma. El algoritmo funciona de la siguiente manera: 
% \begin{itemize}
%     \item primero, determina tres puntos diferentes al azar en la primera superficie: primario (ap), secundario (as) y auxiliar (aa). Considere que las distancias entre estos tres puntos son dps, dpa y dsa. Cada punto en la segunda superficie se considera como el punto correspondiente bp del punto primario ap en la primera superficie.
%     \item Luego, se busca la correspondencia del punto secundario en la segunda superficie a una distancia dps de bp. Si no existe ningún punto alrededor de bp a la distancia dps, descarte bp y comience de nuevo con otro punto primario en la segunda superficie. Sin embargo, si hay un punto secundario bs busque el punto auxiliar ba que satisface las distancias. La transformación entre ambas superficies se puede determinar cuando se identifica la base BB en la segunda superficie. Esta búsqueda se repite para todas las bases encontradas. La mejor transformación es la que tiene el mayor número de puntos correspondientes.
% \end{itemize}

% Aunque este método es robusto incluso con valores atípicos, el principal inconveniente es su tiempo de cálculo. De hecho, este método solo se puede utilizar con una pequeña cantidad de datos de entrada.

% Dentro de PCL, el método \lstinline{pcl::RandomSampleConsensus} representa una implementación del algoritmo RANSAC, tal como se describe en \textbf{[Fischler,1981]}.

% \paragraph{}
%%

%% VER SI PONER O QUE HACER







% \subsubsection{Decimación y filtrado}
% Uno de los grandes problemas que presentan las nubes de puntos al querer hacer un procesamiento de las mismas refieren a la gran densidad de puntos y al\usepackage{comment} ruido en si. Para el primer caso, por ejemplo, si se tiene una nube de 640x480 (estándar), se tendrían que procesar entonces 307200 puntos, generando que se requiera mucho tiempo para completar el procesamiento del algoritmo. El segundo problema, en cambio, genera que se malinterprete la información, provocando resultados incorrectos. 

\subsubsection{Rechazo}
Dado que las correspondencias no válidas pueden afectar negativamente los resultados del registro, la mayoría de los procesos de registro presentan un paso de rechazo. El mismo consiste en filtrar los pares de puntos emparejados en la etapa anterior para facilitar el algoritmo de estimación de la transformación hacia la convergencia al mínimo global. Este paso puede aprovechar la información auxiliar disponible de las nubes de puntos de entrada, como las normales de superficie locales o las estadísticas sobre las correspondencias. A continuación se detallan algunos de los algoritmos más utilizados
\begin{itemize}
    \item \textit{Rechazo de correspondencias basado en la distancia}: en base a un umbral, se filtran las correspondencias que estén a una distancia mayor que dicho límite. En PCL, se implementa mediante \lstinline{pcl::registration::CorrespondenceRejectorDistance}.
    \item \textit{Rechazo en base a la distancia mediana}: a diferencia del método anterior, en este caso el umbral se calcula en base a la mediana de todas las distancias punto a punto en las correspondencias estimadas inicialmente. Se utiliza la mediana ya que suele ser más efectiva en reducir la influencia de valores atípicos. El método \lstinline{pcl::registration::CorrespondenceRejectorMedianDistance} implementa dicha operación.
    \item \textit{Rechazo basado en RANSAC}: Este método aplica el RANdom SAmple Consensus \cite{fischler1981} para estimar una transformación para subconjuntos del conjunto dado de correspondencias y elimina las correspondencias atípicas basadas en la distancia euclidiana entre los puntos después de que la transformación calculada se aplica a la nube de puntos fuente. Es muy eficaz para evitar que el algoritmo ICP converja en mínimos locales, ya que siempre produce correspondencias ligeramente diferentes y es bueno para filtrar valores atípicos. Además, proporciona buenos parámetros iniciales para la estimación de la transformación con todas las correspondencias internas que siguen. El mismo cuenta con el método de PCL \lstinline{pcl::registration::CorrespondenceRejectorSampleConsensus}.
    \item \textit{Rechazo basado en la compatibilidad normal}: este filtro usa la información normal sobre los puntos y rechaza aquellos pares que tienen normales inconsistentes, es decir, el ángulo entre sus normales es mayor que un umbral dado. Puede rechazar pares erróneos que parecen correctos cuando se juzgan solo por la distancia entre los puntos. El mismo es implementado por \lstinline{pcl::registration::CorrespondenceRejectorSurfaceNormal}.
\end{itemize}

\subsubsection{Alineación}
A lo largo de los años, ha habido numerosos enfoques matemáticos para resolver la transformación rígida $T$ que minimiza el error de los pares de puntos. $T$ se compone de una rotación $R$ y una traslación $t$. Tenga en cuenta que, a continuación, cuando se hace referencia a una transformada $T$ y un punto $p$, se utilizarán coordenadas homogéneas. Hay dos métricas de error principales que se deben minimizar y que se han considerado en la literatura: \textit{punto a punto} (Ec. 2) y \textit{punto a plano} (Ec. 3), donde ($p_k$, $q_k$) es el \textit{k-ésimo} de el par $N$ corresponde desde la nube de origen a la nube de destino.
\begin{itemize}
    \item Métrica de error estándar de punto a punto: La métrica de error estándar utilizada en el algoritmo ICP es la métrica de error de punto a punto (Ec. 2). Fue mencionado por primera vez por Arun \cite{arun1987}; los investigadores propusieron varias formas de minimizarlo, seguidas de la introducción del algoritmo ICP \cite{besl1992}. Eggert y col. \cite{eggert1997} evaluó cada uno de estos métodos en términos de estabilidad numérica y precisión, llegando a la conclusión de que tienen un desempeño cercano. PCL ofrece una implementación (\lstinline{pcl::registration::TransformationEstimationSVD}) utilizando \textit{descomposición en valores singulares} (SVD), propuesto en primera instancia por \cite{horn1987}.
    \item \textit{Métrica de error de punto a plano}: Chen y Medioni \cite{chen1992} introdujeron la métrica de punto a plano (Ec. 3) y demostraron que es más estable y converge más rápido que los enfoques anteriores. Utiliza la distancia entre el punto de origen $\bm{p}_k$ y el plano descrito por el punto de destino $\bm{q}_k$ y su normal de superficie local $n_{\bm{q}_k}$. A diferencia de la métrica punto a punto, no tiene una solución de forma cerrada, por lo que la minimización se realiza con solucionadores no lineales (como Levenberg-Marquadt), o linealizándolo \cite{low2004} (bajo el supuesto de ángulos de rotación pequeños, es decir, $sin \theta \sim \theta$ y $cos \theta \sim 1$). Dependiendo de la superficie subyacente y la distribución de puntos, el uso de la métrica de error de punto a plano puede ser considerablemente más robusto. Un procedimiento estándar para minimizarlo se basa en el optimizador no lineal de Levenberg-Marquardt \cite{fitzgibbon2001}. Dicha funcionalidad se emplea en \lstinline{pcl::registration::TransformationEstimationPointToPlane}.
    \item \textit{Métrica de error punto-a-plano ponderada}: Asignar un peso diferente a cada correspondencia puede mejorar la convergencia. La ponderación de los pares de puntos puede verse como un rechazo de correspondencia suave, ajustando la influencia de los puntos correspondientes ruidosos en el proceso de minimización. La ponderación puede ser una función de la distancia punto a punto o punto a plano entre los puntos, una función del ángulo entre las normales correspondientes a los puntos, o una función del modelo de ruido del sensor que se ha usado. Un ejemplo de PCL puede ser \lstinline{pcl::registration::TransformationEstimationPointToPlaneWeighted}.
\end{itemize}

\subsection{Point Cloud 2D}
Si bien las nubes de puntos tridimensionales son una generalización de aquellas que refieren a dos dimensiones, el hecho de tener información sólo de un plano reduce drásticamente la cantidad de puntos a analizar. Este caso sucede en sensores como los LIDAR 2D, donde midiendo el tiempo de vuelo de la señal láser se logran conocer la distancia de distintos puntos de los objetos respecto al sensor. Es por esto que el enfoque de reconstrucción del entorno no suele ser el mismo que el descripto anteriormente.

\subsubsection{Algoritmo para LIDAR 2D}
Al igual que con el método descripto anteriormente, en la reconstrucción de mapas 2D se pretende conseguir la transformación que mejor responda al movimiento del robot entre el instante de tiempo actual y el anterior. Un algoritmo común de coincidencia de escaneo láser encuentra la transformación óptima de cuerpo rígido $T$ que alinea el escaneo láser actual $S_t$ en el tiempo $t$ con el anterior $S_{t-1}$ en el tiempo $t-1$. Este método solo considera dos escaneos láser secuenciales, y cuando se aplican iterativamente para todos los escaneos láser uno por uno, el problema de deriva de pose se deterioraría debido a los errores de coincidencia acumulados, lo que afectará la precisión de la siguiente coincidencia. 

Para abordar este inconveniente, en lugar de comparar los datos en el instante $t$ con el anterior, $t - 1$, se puede comparar dicha información actual con el mapa generado anteriormente, $M_{t-1}$, el cual es generado por todos los escaneos anteriores, de $1$ a $t-1$, y, en caso de ser un occupancy grid map, almacena el valor de probabilidad de cada celda de la cuadrícula en la región del espacio 2D. De acuerdo con las Reglas de Bayes, asumiendo la independencia de cada punto de $S_t$, el valor de probabilidad de la suma de $S_t$ respecto al mapa $M_{t-1}$ se calcula como
\begin{equation}
    p(S_t|M_{t-1}) = \sum_{x\epsilon S_t} p(x|M_{t-1})
\end{equation}
siendo $p(x|M_{t-1})$ la probabilidad de que un punto de escaneo $x \epsilon S_t$ coincida con uno perteneciente a $M_{t-1}$ en esa misma ubicación. Entonces, para buscar la transformación de $S_t$ que mejor se adapte al movimiento del robot respecto al mapa generado anteriormente $M_{t-1}$, llámese $T^*$, puede aplicarse el método de máxima verosimilitud
\begin{equation}
    T^* = argmax(p(T\propto S_t|M_{t-1})
\end{equation}
donde $T\propto S_t$ refiere a los datos del sensor en el instante actual $S_t$ transformados por $T$. A continuación, se presentan una serie de pasos a seguir para lograr el objetivo.

\paragraph{Filtrado de los datos}
Un ejemplo de los datos provenientes de un LIDAR 2D pueden verse en la Figura \ref{fig:lidar2dpoints}. Tal como en el caso de las nubes de puntos tridimensionales, para el caso del LIDAR 2D es necesario quedarse sólo con los datos de mayor relevancia. Por ejemplo, un punto en el espacio sin otros cercanos no aportará mucha información, mientras que una conglomeración de puntos puede resultar en una pared, por ejemplo.

\begin{figure}[!ht]
    \centering
    \includegraphics[width=0.75\linewidth]{Img/LIDAR2DPoints.png}
    \caption{Puntos dados por un LIDAR 2D}
    \label{fig:lidar2dpoints}
\end{figure}

Existen numerosas técnicas para obtener los distintos puntos de interés, ya sea filtrando en base a la distancia entre los puntos, como también la extracción de líneas en base a dichos puntos \textbf{(Gao, 2018)}.

\paragraph{Estimación de la transformación óptima}
En concreto, hay dos formas principales de encontrar la transformación de cuerpo rígido óptima: una es mediante métodos de búsqueda bruta y la otra es la que se basa en el ascenso en gradiente. El método de ascenso en gradiente puede atascarse en el mínimo local, mientras que el método de búsqueda bruta es una búsqueda global y es más robusto. Además, un mapa de múltiples resoluciones y una ventana de búsqueda estrecha pueden mejorar en gran medida la eficiencia de búsqueda del método de búsqueda bruta en una aplicación en tiempo real costosa en tiempo \textbf{(Olson, 2009)(Olson, 2015)}.

\subsubsection{Generación de mapas con LIDAR 2D}
Como los LIDAR en si no aportan información de colores, sino que para el caso de los bidimensionales la misma se trata de puntos en un plano, como se observa en la Figura \ref{fig:lidar2dpoints}, los mapas generados por dichos sensores suelen ser son los \textit{occupancy grid maps} vistos en la Sección \ref{sec:slam}, donde a partir de la ubicación de los puntos provistos por el sensor se determinan los lugares ocupados y libres, siendo las casillas libres las que se encuentran entre el sensor y cada punto y las ocupadas las correspondientes a la posición de los puntos en si, tal como se observa en la Figura \ref{fig:lidar2dmap}. Si bien existen los mapas binarios, en los que se tiene \textit{ocupado, vacio} o \textit{desconocido} solamente, al existir un margen de error en las mediciones, dichos mapas suelen ser probabilísticos, siendo 0.5 el valor por defecto de todas las celdas desconocidas.

\begin{figure}[!ht]
    \centering{{\includegraphics[width=0.75\textwidth]{Img/LIDAR2DMap.png}}}%
    \caption{Mapa generado a partir de los datos del LIDAR 2D}
    \label{fig:lidar2dmap}
\end{figure}

Para poder conseguir el mapa mencionado anteriormente, se tienen que seguir una serie de pasos y tener ciertas consideraciones, las cuales se desarrollan a continuación.

\paragraph{Obtención de las distintas celdas}
Como se trata de un mapa basado en grillas, se deben determinar no solo las celdas que se corresponden con cada punto, sino también aquellas que corresponden a las celdas libres entre los puntos y el sensor.

En el primer caso, al tener la resolución del mapa y la cantidad de cuadros en el mismo, pueden determinarse sin problemas las posiciones de los puntos dados por los datos del LIDAR. 

Para el segundo caso, en cambio, como se dispone a priori de la coordenada del punto y la ubicación del sensor, se podría trazar una recta entre dichos puntos. A partir de la misma, en base a la grilla utilizada, se necesita conocer cuáles celdas atraviesa dicha línea y cuáles no. Un método comúnmente utilizado en este tipo de problemas es el algoritmo de Bresenham \textbf{(Bresenham, 1965)}, permitiendo conseguir los resultados observados en la Figura \ref{fig:bresenhamlinealgorithm}.

\begin{figure}[!ht]
    \centering
    \includegraphics[width=0.9\linewidth]{Img/BresenhamLineAlgorithm.png}
    \caption{Ejemplo del algoritmo de Bresenham en base a una grilla dada. En negro se observa la línea de la cual se parte, y las celdas en gris son las que se obtienen a partir del algoritmo.}
    \label{fig:bresenhamlinealgorithm}
\end{figure}

\paragraph{Ponderación de las celdas}
Como se mencionaba anteriormente, los mapas suelen ser probabilísticos debido al ruido inherente de los sensores. Para poder determinar los valores que debe tomar cada celda (entre 0 y 1), suele ser recurrente el \textit{desenfoque Gaussiano} (del inglés \textit{Gaussian blur}), el cual se basa en desenfocar una imagen mediante una función Gaussiana. Si bien el mismo suele conocerse para una dimensión, puede extenderse a dos dimensiones sin mucho esfuerzo \textbf{(Haddad, 1991)}, ya que se trata del producto de dos Gaussianas, una en cada dimensión.

\paragraph{Actualización del mapa}
Una vez obtenido el mapa basado en la medición actual y, por ende, la transformación que describe el movimiento del robot, se procede a la actualización del mapa generado en los instantes de tiempo anteriores. Para poder realizar esto, una forma sería mediante la actualización de Bayes, vista en la Sección \ref{sec:regressionanalysis} y representada por la Expresión (\ref{eq:posteriorfull}). El problema de la misma es que, si se multiplican valores muy pequeños, el costo computacional aumenta. Una forma de mitigar este inconveniente es mediante el uso de la función \textit{logit}, la cual se basa en un número $p$ entre $0$ y $1$ que responde a la forma

\begin{equation}
    logit(p) = \log\left(\frac{p}{1-p}\right)
\end{equation}
obteniendo entonces valores entre $-\infty$ y $+\infty$, tal como se observa en la Figura .a. Para volver a la forma de probabilidad comúnmente utilizada, observada en la Figura .b, se utiliza la siguiente Expresión, la cual se consigue despejando la anterior
\begin{equation}
    p = \frac{e^{logit(p)}}{1+e^{logit(p)}}
\end{equation}

Partiendo entonces de la Expresión (\ref{eq:posteriorfull}), separando la medición actual de las anteriores y utilizando la suposición de Markov, la cual define que la medición actual es independiente de las mediciones anteriores si el estado actual es conocido, se llega a
\begin{equation}
    p(m_i|y_{1:t}) = \frac{p(y_t|m_i)p(m_i|y_{1:t-1})}{p(y_t|y_{1:t-1})}
\end{equation}
siendo 
\begin{itemize}
    \item $m_i$ la celda actual del mapa,
    \item $y_{1:t}$ las mediciones del sensor en esa celda desde el tiempo $1$ al tiempo $t$, 
    \item $p(y_t|m_i)$ la probabilidad de la medición actual dado el estado de la celda,
    \item $p(m_i|y_{1:t-1})$ la probabilidad de que una celda esté ocupada dadas todas las mediciones anteriores, y
    \item $p(y_t|y_{1:t-1})$ la probabilidad de tener una medición $y_t$ dadas todas las mediciones anteriores $y_{1:t-1}$.
\end{itemize}


Si se expande $p(y_t|m_i)$ mediante la regla de Bayes
\begin{equation}
    p(y_t|m_i) = \frac{p(m_i|y_t)p(y_t)}{p(m_i)}
\end{equation}
y se aplica a la Expresión anterior, se obtiene
\begin{equation}
    p(m_i|y_{1:t}) = \frac{p(m_i|y_t)p(y_t)p(m_i|y_{1:t-1})}{p(m_i)p(y_t|y_{1:t-1})}
\end{equation}

Luego, aplicando la fracción presente en la función \textit{logit}, se puede llegar a
\begin{equation}
    \frac{p(m_i|y_{1:t})}{1-p(m_i|y_{1:t})} = \frac{p(m_i|y_t)(1-p(m_i))p(m_i|y_{1:t-1})}{(1-p(m_i|y_t))p(m_i)(1-p(m_i|y_{1:t-1})}
\end{equation}

Finalmente, aplicando el logaritmo,
\begin{equation}
    logit(p(m_i|y_{1:t})) = logit(p(m_i|y_t)) + logit(p(m_i|y_{1:t-1})) - logit(p(m_i))
    \label{eq:logitupdatemap}
\end{equation}
resultando entonces la actualización del mapa en la adición de tres funciones \textit{logit}, siendo el primer término el referido al valor de la celda en base a la medición actual, el segundo correspondiente al valor de la celda respecto a las mediciones anteriores, y el tercero corresponde al valor de la celda en el instante inicial (siendo desconocido, $m_i = 0.5$ generalmente).

\subsection{Resumen}
En esta Sección, se presentó el concepto de nube de puntos, a su vez de mencionar las características de la \textit{Point Cloud Library} en este tipo de aplicaciones, siendo una herramienta esencial para el desarrollo del trabajo presentado.

A su vez, se presentaron los distintos métodos para la generación de mapas 2D, además de mencionar las técnicas utilizadas para la estimación del movimiento según los datos acutales y los anteriores.


\ifimagenes
\else
    \newpage
    \section{Plataformas disponibles}
\label{sec:platforms}
En esta sección se evaluarán principalmente diferentes plataformas de hardware disponibles en el mercado\footnote{El relevamiento fue realizado en la fecha 7 de Julio de 2018}. Si bien muchos autores han desarrollado arquitecturas de robots capaces de localización y mapeo simultáneos (SLAM) \cite{engel2012}, \cite{engel2014}, \cite{hausman2016}, las plataformas de control de las mismas no cuentan con un factor de forma de hardware definido que, si bien la cantidad de trabajos que lo utilizan son acotados \cite{qi2009}, esto permite la interconexión de la placa base con nuevos módulos a elegir por el usuario de la plataforma, logrando así expandir sus funcionalidades.

\subsection{Plataformas basadas en FPGA}
\begin{itemize}
    \item \textbf{Phenix Pro DevKit:} está diseñado sobre un System on Chip (SoC) por RobSense Tech, es reconfigurable. El controlador de vuelo (Fig. \ref{fig:fpga_based}.a) incluye el sistema operativo en tiempo real basado en FreeRTOS (UOS) y un Robot Operating System (ROS) basado en Linux. Esta plataforma soporta mas de 20 interfaces incluyendo sensores on-board, mmWave radar, Lidar, entre otros. Permite visión artificial y aplicaciones de algoritmos de Deep Neural Networks \cite{shah2017}.
    
    \item \textbf{Octagonal Pilot on Chip (OcPoc):} desarrollado por Aerotenna Company (Fig. \ref{fig:fpga_based}.b), expande sus capacidades entradas y salidas al incluir pines completamente programables de PWM, PPM y GPIO para integrar con un gran número de diferentes sensores adicionales. Incluye además otros conectores estándar para periféricos tales como GPS y tarjeta SD. El mismo corre la plataforma de software Autopilot \cite{autopilot} e implementa un procesamiento simultáneo, en tiempo real, de los datos de los sensores.
    \begin{figure}[!ht]
        \centering
        \includegraphics[width=.95\textwidth]{Img/fpga_based}
        \caption{Controlador de vuelo: (a) Phoenix Pro DevKit, (b) OcPoc}
        \label{fig:fpga_based}
    \end{figure}
\end{itemize}

\subsection{Plataformas basadas en ARM MCUs}
\begin{itemize}
    \item \textbf{Pixhawk:} consiste en un controlador PX4-Flight Management Unit (FMU) y una PX4-IO integrada en una misma placa con IO adicionales, memoria y otras características (Fig. \ref{fig:pixhawks}.a). Se encuentra dentro del proyecto DroneCode [REF DRONECODE].
    \item \textbf{Pixhawk 2:} Es un cubo pequeño que cuenta con tres IMUs redundantes y hasta tres módulos GPS (Fig. \ref{fig:pixhawks}.b) (Ardupilot). Toda la conección IO del cubo se encuentra en un conector DF17. Su placa portadora posee una interfaz con Intel Edison.
    
    \begin{figure}[!ht]
        \centering
        \includegraphics[width=.95\textwidth]{Img/pixhawks}
        \caption{Controlador de vuelo: (a) Pixhawk, (b) Pixhawk 2}
        \label{fig:pixhawks}
    \end{figure}
    
    \item \textbf{FlightCtrl:} Desarrollada por MikroKopter, su última versíon introducida al mercado, la V3.0 (Fig. \ref{fig:flypaparazzi}.a), cuenta con un microcontrolador avanzado, sistema de telemetría, GPS, entre otros, además de ofrecer sistemas con placas de control de vuelo duplicadas que mantienen el helicóptero estable incluso si hay una falla en la placa primaria de control de vuelo.
    \item \textbf{Paparazzi:} Es el primer proyecto de software y hardware open-source para drones [Brisset]. En marzo de 2017 salió el nuevo autopilot del mismo llamado Chimera (Fig. \ref{fig:flypaparazzi}.b) el cual está basado en el microcontrolador STM32F7.
    \item \textbf{FlyMaple:} basada en el Maple Project, el cual es un procesador Arduino ARM, consiste en una placa controladora para cuadricópteros (Fig. \ref{fig:flypaparazzi}.c). La misma cuenta como aplicaciones típícas las de robots balancines, plataformas móbiles, helicópteros y cuadricópteros que requieren IMUs y controladores en tiempo real de alto rendimiento.
    
    \begin{figure}[!ht]
        \centering
        \includegraphics[width=.95\textwidth]{Img/flypaparazzi}
        \caption{Controlador de vuelo: (a) FlightCtrl v3.0, (b) Paparazzi Chimera, (c) FlyMaple}
        \label{fig:flypaparazzi}
    \end{figure}
    
    \item \textbf{CC3D y Atom:} Son dos controladoras de vuelo que poseen las mismas funcionalidades pero con tamaños diferentes. Los mismos (Figs. \ref{fig:apmatom}.a y \ref{fig:apmatom}.b) tienen todos los tipos de hardware de estabilización que corre el firmware OpenPilot/LibraPilot. Pueden ser configuradas para volar con cualquier frame.
    
    \item \textbf{Ardupilot Mega (APM):} Desarrollada por la comunidad DIY Drones como una mejora de la controladora de vuelo Autopilot, consiste en una plataforma basada en  el Arduino Mega (Fig. \ref{fig:apmatom}.c). Puede controlar multicópteros autónomos, helicópteros tradicionales, entre otros.

    \begin{figure}[!ht]
        \centering
        \includegraphics[width=.95\textwidth]{Img/flypaparazzi}
        \caption{Controlador de vuelo: (a) CC3D, (b) Atom, (c) Ardupilot Mega 2.8}
        \label{fig:apmatom}
    \end{figure}
\end{itemize}

\subsection{Plataformas basadas en Raspberry Pi}
\begin{itemize}
    \item \textbf{Erle-Brain 3:} Consiste en un open pilot para drones basado en Linux desarrollado por Erle Robotics (Fig. \ref{fig:raspi}.a) (Erle-Robotics). Combina una Raspberry Pi junto a PXFmini, contando esta última con sensores, IO y alimentación. Está diseñado sobre el proyecto DroneCode.
    \item \textbf{NAVIO2 Autopilot:} Es una integración de sensores, GPS y alimentación apareado con una Raspberry Pi. Cuenta con doble IMU para mejorar el desempeño y conseguir redundancia (Fig. \ref{fig:raspi}.b).
    
    \begin{figure}[!ht]
        \centering
        \includegraphics[width=.95\textwidth]{Img/raspi}
        \caption{Controlador de vuelo: (a) Erle-Brain 3 Autopilot, (b) NAVIO2 Autopilot}
        \label{fig:raspi}
    \end{figure}
\end{itemize}

\subsection{Plataformas basadas en Qualcomm}
\begin{itemize}
    \item \textbf{Snapdragon Autopilot:} la plataforma Snapdragon Flight (Fig. \ref{fig:snapdragon}) es un piloto automático de gama alta que puede ejecutar el plan de vuelo sobre un sistema operativo en tiempo real DSP utilizando la API DSPAL para la compatibilidad con POSIX. El mismo cuenta con un SoC Snapdragon 801. En comparación con la Pixhawk, sus características incluyen potencia de procesamiento avanzada, control de vuelo en tiempo real en Hexagon DSP, Wi-Fi, conectividad Bluetooth, GPS automotriz, una cámara de flujo óptico y una cámara de video de resolución 4K.    
    \begin{figure}[!ht]
        \centering
        \includegraphics[width=.55\textwidth]{Img/snapdragon}
        \caption{Snapdragon Flight}
        \label{fig:snapdragon}
    \end{figure}
\end{itemize}

\subsection{Comparación entre plataformas existentes}


En el Cuadro \ref{tab:plataformas} (Yang, 2016)(Raj, 2016) se resumen las características principales. En la misma, se observa que la principal diferencia entre ellas es la unidad de procesamiento, además de que no todas ellas presentan redundancia de sensores críticos.

% Please add the following required packages to your document preamble:
% \usepackage{booktabs}
% \usepackage{graphicx}
\begin{table}
\resizebox{\textwidth}{!}{%
\begin{tabular}{@{}|c|c|c|c|c|c|c|@{}}
\toprule
\textbf{Plataforma} & \textbf{Procesador} & \textbf{Sensores} & \textbf{\begin{tabular}[c]{@{}c@{}}Interfaces /\\ Conectividad\end{tabular}} & \textbf{Redundancia} & \textbf{\begin{tabular}[c]{@{}c@{}}Dimensiones \\ (mm)\end{tabular}} & \textbf{Peso (g)} \\ \midrule
Phoenix Pro & \begin{tabular}[c]{@{}c@{}}Xilinx Zynq\\ 7020\end{tabular} & \begin{tabular}[c]{@{}c@{}}Acelerometro,\\ giroscopo,\\ barómetro,\\ GPS\end{tabular} & \begin{tabular}[c]{@{}c@{}}USB, UART, I2C,\\ CAN, SPI, miniHDMI,\\ Camera Link, LVDS,\\ PWM, telemetria\end{tabular} & - & 73,8*55,8*18 & 64 \\ \midrule
OcPoc & \begin{tabular}[c]{@{}c@{}}Xilinx Zynq\\ Z-7010\end{tabular} & \begin{tabular}[c]{@{}c@{}}IMU 9 DOF\\ (MPU9250),\\ Barometro\\ (MS5611)\end{tabular} & \begin{tabular}[c]{@{}c@{}}I2C, USB-OTG,\\ USB-UART, SPI, \\ CSI, GSI, CAN, \\ ADC, PWM\end{tabular} & \begin{tabular}[c]{@{}c@{}}IMU\\ (MPU9250)\end{tabular} & 92*64*21 & 70 \\ \midrule
PIXHAWK & \begin{tabular}[c]{@{}c@{}}ARM\\ Cortex-M4F\end{tabular} & \begin{tabular}[c]{@{}c@{}}Giroscopo\\ (L3GD20H),\\ acelerometro /\\ magnetometro\\ (LSM303D),\\ barometro\\ (MS5611)\end{tabular} & \begin{tabular}[c]{@{}c@{}}UART, CAN, I2C,\\ SPI, ADC, PWM\end{tabular} & \begin{tabular}[c]{@{}c@{}}Giroscopo /\\ acelerometro \\ (MPU6000)\end{tabular} & 50*15,5*81,5 & 38 \\ \midrule
PIXHAWK2 & \begin{tabular}[c]{@{}c@{}}ARM\\ Cortex-M4F\end{tabular} & \begin{tabular}[c]{@{}c@{}}IMU 9 DOF\\ (MPU9250 /\\ ICM20948 /\\ ICM20648 /\\ L3GD20 +\\ LSM303D), \\ Barometro\\ (MS5611)\end{tabular} & \begin{tabular}[c]{@{}c@{}}S.Bus, I2C, SPI, \\ CAN, Carrier board \\ para Intel\\ Edison, ADC, \\ PWM\end{tabular} & \begin{tabular}[c]{@{}c@{}}2x IMU \\ (MPU9250),\\ Barometro\\ (MS5611)\end{tabular} & \begin{tabular}[c]{@{}c@{}}50*15,5*81,6\\ +35*35 (Cube)\end{tabular} & - \\ \midrule
FlightCtrl V3.0 & \begin{tabular}[c]{@{}c@{}}ARM\\ Cortex-M4F\end{tabular} & \begin{tabular}[c]{@{}c@{}}Giroscopo,\\ magnetometro,\\ acelerometro,\\ barometro, GPS\end{tabular} & \begin{tabular}[c]{@{}c@{}}UART, I2C, SPI, \\ ADC, PWM, S.Bus, \\ CAN, telemetria\end{tabular} & - & 67*67*- & 32 \\ \midrule
Paparazzi & STM32F767 & \begin{tabular}[c]{@{}c@{}}IMU 9 DOF\\ (MPU9250),\\ Barometro\\ (MS5611),\\ Sensor de \\ presion\\ (MS4525DO)\end{tabular} & \begin{tabular}[c]{@{}c@{}}UART, I2C, SPI, \\ CAN, USB, PWM,\\ PPM/S.Bus\end{tabular} & - & 89*60*- & - \\ \midrule
CC3D/Atom & STM32F & \begin{tabular}[c]{@{}c@{}}Giroscopio /\\ acelerometro \\ (MPU6000)\end{tabular} & \begin{tabular}[c]{@{}c@{}}UART, I2C,\\ S.Bus/PPM, PWM\end{tabular} & - & \begin{tabular}[c]{@{}c@{}}36*36*-(CC3D)\\ 15*7*- (Atom)\end{tabular} & \begin{tabular}[c]{@{}c@{}}8 (CC3D)\\ 4 (Atom)\end{tabular} \\ \midrule
APM & ATMEGA2560 & \begin{tabular}[c]{@{}c@{}}Giroscopo /\\ acelerometro \\ (MPU6000),\\ barometro(MS5611), \\ GPS\end{tabular} & \begin{tabular}[c]{@{}c@{}}I2C, PWM, UART, \\ OSD, telemetria\end{tabular} & - & 142*96*18 & 55 \\ \midrule
FlyMaple & STM32F103 & \begin{tabular}[c]{@{}c@{}}Giroscopo\\ (ITG-3200),\\ acelerometro\\ (ADXL345),\\ magnetometro\\ (HMC5883L),\\ barometro\\ (BMP085), GPS\end{tabular} & \begin{tabular}[c]{@{}c@{}}PWM, USB, UART, \\ I2C, ADC\end{tabular} & - & 50*50*12 & 15 \\ \midrule
Erle-Brain 3 & \begin{tabular}[c]{@{}c@{}}ARMv8\\ Quad-Core\\ (Raspberry PI 3)\end{tabular} & \begin{tabular}[c]{@{}c@{}}Giroscopo,\\ magnetometro,\\ acelerometro,\\ barometro, GPS,temperatura\end{tabular} & \begin{tabular}[c]{@{}c@{}}I2C, UART, USB, \\ HDMI, Ethernet, PWM, \\ PPM/S.Bus,ADC, \\ Bluetooth, Audio Jack\end{tabular} & - & 95*70*23,8 & 100 \\ \midrule
NAVIO2 & \begin{tabular}[c]{@{}c@{}}ARMv8\\ Quad-Core\\ (Raspberry PI 3)\end{tabular} & \begin{tabular}[c]{@{}c@{}}IMU 9 DOF \\ (MPU9250), \\ barometro, \\ GPS, RC I/O\end{tabular} & \begin{tabular}[c]{@{}c@{}}I2C, UART,\\ PWM, S.Bus\end{tabular} & \begin{tabular}[c]{@{}c@{}}IMU 9 DOF\\ (LSM9DS1)\end{tabular} & 55*65*- (shield) & 23 (shield) \\ \midrule
Snapdragon & \begin{tabular}[c]{@{}c@{}}Snapdragon\\ 801\end{tabular} & \begin{tabular}[c]{@{}c@{}}IMU 9 DOF\\ (MPU9250),\\ barometro\\ (BMP280),\\ optical flow\\ (OV7251), GPS\end{tabular} & \begin{tabular}[c]{@{}c@{}}Wifi, USB, PWM, \\ UART, I2C\end{tabular} & - & 68*52*- & - \\ \bottomrule
\end{tabular}%
}
\caption{Comparativa entre distintas plataformas disponibles}
\label{tab:plataformas}
\end{table}

En muchas de las plataformas analizadas, puede observase la presencia de sensores duplicados, tal como es el caso de giroscopos y acelerómetros. Dicha \textit{redundancia} permite al sistema aumentar su \textit{confiabilidad}, logrando así reducir las posibles fallas del sistema, sobre todo en los sensores más críticos.
\begin{large}
[PONER DE $https://www.ncbi.nlm.nih.gov/pmc/articles/PMC6165073/$]
[$https://link.springer.com/chapter/10.1007/978-1-4612-3148-6_8$]
\end{large}
\subsection{Estándar de Hardware}

A partir del relevamiento realizado, se puede establecer que los métodos de conexión que presentan las plataformas existentes en el mercado no cuentan con un estándar de hardware definido, sino que cada una tiene su interfaz de conexión propia. Esto trae como inconveniente que, si se requiere expandir su funcionalidad agregando, por ejemplo, más sensores, en base a la plataforma que se tenga dependerá el hardware asociado a realizarse.

\subsubsection{PC/104}
PC/104 es una familia de estándares de computadoras embebidas que definen tanto los factores de forma como los buses de sistema. El mismo está diseñado para entornos especializados donde se requiere un sistema informático pequeño y resistente. El estándar es modular, permitiendo \textit{stackear} placas de una gran variedad de fabricantes con tal de producir un sistema integrado personalizado \cite{pc104}.

Las placas PC/104 se apilan unas sobre otras como bloques de construcción. La especificación PC/104 define cuatro orificios de montaje en las esquinas de cada módulo, lo que permite que las tablas se sujeten entre sí mediante separadores. El tamaño de la placa compacta contribuye aún más a la robustez del factor de forma al reducir la posibilidad de flexión de PCB bajo impacto y vibración.

\begin{figure}[!ht]
    \centering
    \includegraphics[width=\textwidth]{Img/pci104threebank}
    \caption{Board Layouts de PCI/104-Express y PCIe/104}
    \label{fig:pci104threebank}
\end{figure}

Las estructuras de buses definidos son los que dan lugar a las familias de estándares conocidos del mismo, siendo:

\begin{itemize}
    \item \textbf{PC/104:} Este bus original deriva del bus ISA. Incluye todas las señales encontradas en el mismo, con pines de tierra adicionales agregados para garantizar la integridad del bus.
    \item \textbf{PC/104-Plus:} El mismo agrega soporte para el bus PCI, además del bus ISA del estándar PC/104.
    \item \textbf{PCI-104:} Incluye el conector PCI, pero no el conector PC/104, para aumentar el espacio disponible de la placa. A pesar de que el conector PCI tiene 120 pines en lugar de 104, se mantuvo el nombre establecido. La ubicación del conector PCI y el pinout son idénticos a PC/104-Plus. Dado que se omite el bus ISA, una placa PCI-104 es incompatible con el módulo periférico PC/104. Sin embargo, PCI-104 y PC/104-Plus son compatibles, ya que ambos utilizan el bus PCI. La mayoría de las placas PC/104-Plus pueden fabricarse como PCI-104 simplemente no rellenando el conector PC/104. PCI-104 utiliza el mismo esquema de selección de Número de Ranura PCI que PC/104-Plus. Cada dispositivo debe asignarse a un número de ranura único.
    \item \textbf{PCI/104-Express:} Incorpora el bus PCI Express (PCIe) además del bus PCI de la generación anterior. La especificación define un conector de montaje en superficie de 156 pines para las señales PCI Express. El nuevo conector ocupa la misma ubicación de la placa que el conector ISA PC/104 heredado. Además de PCI Express, las especificaciones también definen los pines en el conector para buses modernos adicionales, como USB y SATA.
    \item \textbf{PCIe/104:} Similar al estándar PCI/104-Express, pero omite el bus PCI heredado para aumentar el espacio disponible en la placa (similar a la relación entre PC/104-Plus y PCI-104). La ubicación del conector PCI Express y las opciones de asignación de patillas son las mismas que para PCI/104-Express (Tipo 1 y Tipo 2). Debido a que se omite el conector de bus PCI, una placa PCIe/104 es incompatible con los sistemas PC/104-Plus y PCI-104 (a menos que se use un dispositivo de puente PCIe a PCI).
\end{itemize}

El factor de forma 104 define el tamaño de la placa (90 * 96 mm), con orificios de montaje en las cuatro esquinas de la placa. Las especificaciones también permiten un área de 0.5 pulgadas (13 mm) más allá del borde de la PCB para los conectores de entrada y salida. La especificación PCI/104-Express y PCIe/104 introdujo el nombre ''104'' para distinguir el factor de forma del bus PC/104 heredado. A su vez, dichas especificaciones aceptan las opciones de uno (Fig. \ref{fig:pci104threebank}) o tres (Fig. \ref{fig:pci104onebank}) \textit{bancos}, agregando mayor flexibilidad al diseñador \cite{pci104express}.

\begin{figure}[!ht]
    \centering
    \includegraphics[width=\textwidth]{Img/pci104onebank}
    \caption{Board Layouts de PCI/104-Express y PCIe/104 con conector OneBank}
    \label{fig:pci104onebank}
\end{figure}
    

\fi

\newpage
\section{Resolución}
\subsection{Calibración de la unidad incercial}
\subsubsection{Calibración del giróscopo}
\textbf{ALLAN}
Para poder conocer el tamaño adecuado que permita obtener el \textit{bias} del giróscopo en el instante inicial, se utiliza la \textit{varianza de Allan}[20][8] $\sigma_{Allan}$, la cual mide la varianza de la diferencia entre promedios de intervalos consecutivos, siendo entonces
\begin{align}
    \sigma_{Allan} &= \frac{1}{2} E[(x(\tilde{t},k) - x(\tilde{t},k-1))^2] \\
    &= \frac{1}{2K}\sum_{k=1}^K(x(\tilde{t},k) - x(\tilde{t},k-1))^2
\end{align}
donde $x(\tilde{t},k)$ es el \textit{k}-ésimo intervalo promedio que abarca $\tilde{t}$ segundos, y K es el número de intervalos en que se segmenta el tiempo total considerado. Se computa la varianza de Allan para los tres ejes, y en el intervalo de tiempo en el que los tres convergen a un valor pequeño representa una buena elección para elegir el período de inicialización, $T_{init}$.

\textbf{INTEGRACION}
Resolver esta ecuación diferencial implica poder integrarla. Si bien existen varios métodos para hacerlo, ...


Esta función, en concreto, requiere de una integración de la velocidad angular en un tiempo discreto. Si bien existen diferentes métodos de integración numérica, es necesario que el mismo sea robusto y estable para mejorar la exactitud de la calibración. Por eso, el \textit{Runge-Kutta} $4^th$ \textit{order normalized method} (RK4n)[19] es el elegido.

Si la ecuación diferencial que describe a la cinemática del cuaternión se define como
\begin{equation}
    \bm{f}(\bm{q},t) = \dot{\bm{q}} = \frac{1}{2}\bm{\Omega}(\bm{\omega}(t))\bm{q}
\end{equation}
donde $\bm{\Omega}(\bm{\omega}(t))$ es el operador que convierte la velocidad angular tridimensional considerada en la representación de la matriz simétrica oblicua real, esto es,
\begin{equation}
    \bm{\Omega}(\bm{w}) = 
    \begin{bmatrix}
        0 & -w_x & -w_y & -w_z \\
        w_x & 0 & w_z & -w_y \\
        w_y & -w_z & 0 & w_x \\
        w_z & w_y & -w_x & 0
    \end{bmatrix}
\end{equation}
El algoritmo de integración RK4n es
\begin{align}
    \bm{q}_{k+1} &= \bm{q}_k + \Delta t\frac{1}{6}(\bm{k}_1 +\bm{k}_2 + \bm{k}_3 + \bm{k}_4) \\
    \bm{k}_i &= \bm{f}(\bm{q}^{(i)},t_k+c_i\Delta t) \\
    \bm{q}^{(i)} &= \bm{q}_k &&\text{para} &&&i=1 \\
    \bm{q}^{(i)} &= \bm{q}_k + \Delta t\sum_{j=1}^{i-1}a_{ij}\bm{k}_j &&\text{para} &&&i>1
\end{align}
donde todos los coeficientes necesarios, $c_i$ y $a_{ij}$ son
\begin{align*}
    c_1 &= 0,\ c_2 = \frac{1}{2},\ c_3 = \frac{1}{2},\ c_4 = 1 \\
    &a_{21} = \frac{1}{2},\ a_{31} = 0,\ a_{41} = 0, \\
    &a_{32} = \frac{1}{2},\ a_{42} = 0,\ a_{43} = 1
\end{align*}

Finalmente, en cada paso, es necesario normalizar el cuaternión $(k+1)$-ésimo, ya que puede derivar de la longitud de la unidad
\begin{equation}
    \bm{q}_{k+1} \rightarrow \frac{\bm{q}_{k+1}}{||\bm{q}_{k+1}||}
\end{equation}


\newpage
\section{Conclusiones}
\label{sec:5_concl}
\ifimagenes
En el trabajo presentado se logró en primera instancia realizar las comprobaciones prácticas en una cámara estereo para luego realizar un algoritmo de SLAM en base a los datos de una cámara RGB-D, con la capacidad de poder reconstruir un mapa símil al real, y obteniendo las poses relativas de cada iteración. Si bien es cierto que se realizaron las mediciones en base a simulaciones, el hecho de haber incorporado ROS permite que el código pueda adaptarse fácilmente no solo a un entorno real, sino también a cualquier robot, permitiendo entonces una gran flexibilidad a la hora de utilizarlo. 

Respecto a los tiempos de procesamiento del algoritmo, los mismos son suficientes para el movimiento de robots en velocidades moderadas, en cambio si se pretende utilizar al mismo a una gran velocidad es posible que no pueda obtener los resultados esperados, debido a que el algoritmo espera que la nube de puntos anterior no esté muy alejada de la nueva nube de puntos. 

\section{Trabajos futuros}
Este trabajo forma parte de un trabajo final de carrera en desarrollo, el cual pretende realizar un SLAM 2D y 3D aprovechando otros sensores para obtener el mapa final, en particular, una IMU y un LIDAR 2D. Por dicha razón es que el documente presentado es adecuado para dicha tarea ya que, por ejemplo, con los datos de un LIDAR 2D a la hora de correr el algoritmo en tiempo real, la pose anterior será la estimada por el LIDAR, haciendo que el punto de partida no esté muy lejos del resultado final, evitando así errores por no encontrar el mínimo global.

Como este proyecto se engloba dentro de un problema de SLAM con fusión sensorial entre la cámara, LIDAR 2D e IMU, se pretende que el algoritmo mejore considerablemente a la hora de realizar dicha fusión.
\else
En el trabajo presentado se logró, en primera instancia, diseñar una plataforma versátil para aplicaciones robóticas en general, ya sea para robots terrestres como para robots aéreos en primera instancia, gracias a los sensores que presenta la misma por defecto. A su vez, si se pretende extender a la misma a otras aplicaciones (incluso si se quiere agregar redundancia de sensores, por ejemplo), la misma cuenta con el factor de forma PC104, permitiendo así extender sus funcionalidades en caso de que sea necesario bajo un estándar conocido en la industria.

Luego, en base a los sensores seleccionados anteriormente, se realizaron distintas calibraciones de los mismos. Cabe destacar que, si bien los datos provistos de la calibración de la IMU no parece ser en primera instancia numéricamente significativa, el hecho de que se integre dos veces el valor de la aceleración por cada muestra del sensor hace que, si el dato no es lo mas próximo a la realidad, hará que aumente el error de dicha estimación de posición y, en consecuente, diverja al poco tiempo. Si bien el hecho de realizar la estimación de la posición en base a los datos de la IMU únicamente es una tarea muy compleja, la correcta calibración de la misma permitiría una mejor estimación de la pose en una futura fusión sensorial.

A continuación, se realizaron las comprobaciones prácticas en una cámara estéreo, desde la adquisición de los datos hasta la obtención de la nube de puntos arrojada por la misma.

Como no se dispuso de una forma de corroborar las poses obtenidas en cada instante mediante los datos reales, se optó por el uso de simulaciones mediante Gazebo, utilizando el modelo del robot ROSbot 2.0, ya que cuenta con una cámara RGB-D, un LIDAR 2D y una IMU, entre otros sensores.

Luego, se realizó un algoritmo de SLAM 3D en base a los datos de la cámara RGB-D y, utilizando los datos obtenidos de las simulaciones, se pudo reconstruir un mapa símil al real, obteniendo las poses relativas de cada iteración.

A su vez, con el uso de un LIDAR 2D ayudado por los datos de una IMU se consiguió resolver el problema de SLAM 2D, con tiempos considerables como para que el mismo pueda implementarse en tiempo real.

Si bien es cierto que se realizaron las mediciones en base a simulaciones, el hecho de haber incorporado ROS permite que el código pueda adaptarse fácilmente no solo a un entorno real, sino también a cualquier robot, permitiendo entonces una gran flexibilidad a la hora de utilizarlo.

Si bien este trabajo pretende realizar mejoras, las cuales se tratarán en la Sección \ref{sec:futureworks}, el código fuente realizado se encuentra disponible en el repositorio \textit{ramon\_slam}\footnote{\url{https://github.com/frand08/ramon_slam}} a todo aquel que pretenda incursionar ya sea con SLAM, calibraciones, simulaciones o la implementación de ROS en microcontroladores, entre otras características realizadas. En el mismo también se encuentra una guía de instalación de los distintos módulos a todo aquel que quiera utilizar el código.

%Respecto a los tiempos de procesamiento del algoritmo, los mismos son suficientes para el movimiento de robots en velocidades moderadas, en cambio si se pretende utilizar al mismo a una gran velocidad es posible que no pueda obtener los resultados esperados, debido a que el algoritmo espera que la nube de puntos anterior no esté muy alejada de la nueva nube de puntos. 

\fi

\newpage
% \section{Bibliografía}
\label{sec:biblio}

\begin{thebibliography}{9}
\bibitem{engel2012}
    J. Engel, J. Sturm, and D. Cremers,
    \textit{''Camera-based navigation of a low-cost quadrocopter''}.
    Conference on Intelligent Robots and Systems (IROS),
    2012.
    
\bibitem{engel2014}
    J. Engel, J. Sturm, and D. Cremers,
    ''\textit{Scale-aware navigation of a low-cost quadrocopter with a monocular camera''}.
    Robotics and Autonomous Systems, 
    2014.

\bibitem{hausman2016}
    K. Hausman, S. Weiss, R. Brockers, L. Matthies, and G.S. Sukhatme,
    \textit{''Self-calibrating multi-sensor fusion with probabilistic measurement validation for seamless sensor switching on a UAV''}.
    IEEE International Conference on Robotics and Automation (ICRA),
    2016.

\bibitem{qi2009}
    J. Qi, D. Song, L. Dai and J. Han,
    ''\textit{Design, Implement and Testing of a Rotorcraft UAV System''}.
    Aerial Vehicles, 
    January 2009.

\bibitem{delrosario2016}
    Del Rosario M, Lovell N, Redmond S,
    \textit{''Quaternion-based Complementary Filter for Attitude Determination of a Smartphone''}.
    IEEE Sensors Journal,
    Jun 2016.

\bibitem{caron2006}
    Caron F, Duflos E, Pomorski D, Vanheeghe P,
    \textit{''GPS/IMU data fusion using multisensor Kalman filtering: introduction of contextual aspects''}.
    Information Fusion,
    2006 Jun 30,
    7(2):221-30.
    
\bibitem{mirzaei2008}
    Mirzaei FM, Roumeliotis S,
    \textit{''A Kalman filter-based algorithm for IMU-camera calibration: Observability analysis and performance evaluation''}.
    Robotics, IEEE Transactions,
    2008 Oct,
    24(5):1143-56.

\bibitem{hesch2009}
    Hesch J, Mirzaei FM, Mariottini GL, Roumeliotis S,
    \textit{''A 3d pose estimator for the visually impaired''}.
    IEEE/RSJ International Conference,
    2009 Oct 10,
    (pp. 2716-2723),
    IEEE.
    
\bibitem{chambers2014}
    Chambers A, Scherer S, Yoder L, Jain S, Nuske S, Singh S,
    \textit{''Robust multi-sensor fusion for micro aerial vehicle navigation in GPS-degraded/denied environments''}.
    InAmerican Control Conference (ACC),
    2014 Jun 4, 
    pp. 1892-1899, IEEE.

\bibitem{hesch2013}
    Hesch JA, Kottas DG, Bowman SL, Roumeliotis SI,
    \textit{''Camera-IMU-based localization: Observability analysis and consistency improvement''}.
    The International Journal of Robotics Research,
    2013.

\bibitem{lee2016}
    S. Lee, D. Har, and D. Kum,
    \textit{''Drone-Assisted Disaster Management: Finding Victims via Infrared Camera and Lidar Sensor Fusion''}.
    3rd Asia-Pacific World Congress on Computer Science and Engineering, 
    2016.

\bibitem{shah2017}
    M. Shah, R. Kapdi,
    \textit{''Object detection using deep neural networks''}.
     International Conference on Intelligent Computing and Control Systems (ICICCS),
     2017.
     
\bibitem{autopilot}
    Aautopilot Project,
    \url{https://autopilot-project.eu/}.
    Última vez accedido el 03-08-2018.

\bibitem{pc104}
    PC/104 Embedded Consortium,
    \textit{''PC/104 Specification''}.
    Version 2.6,
    October 13, 2008.

\bibitem{pci104express}
    PC/104 Consortium,
    \textit{''PCI/104-Express \& PCIe/104 Specification''}.
    Version 3.0,
    February 17, 2015
    
\bibitem{endres2012}
    F. Endres, J. Hess, N. Engelhard,
    \textit{''An Evaluation of the RGB-D SLAM System''}.
    2012 IEEE International Conference on Robotics and Automation,
    2012
    
\bibitem{davison2007}
    A. Davison, I. Reid, O. Stasse
    \textit{''MonoSLAM: Real-Time Single Camera SLAM''},
    IEEE Trans. On Pattern Analysis And Machine Intelligence,
    2007
    
\bibitem{botterill2013}
    T. Botterill, S. Mills, R. Green,
    \textit{''Collecting Scale Drift by Object Recognition in Single-Camera SLAM''},
    IEEE Trans. Cybernetics, vol. 43, no.6, pp. 1767-1780,
    2013

\bibitem{sola2007}
    J. Solà,
    \textit{''Towards Visual Localization, Mapping and Moving Objects Tracking by a Mobile Robot: A Geometric and Probabilistic Approach},
    LAAS, Tolouse,
    2007
    
\bibitem{murray2007}
    G. Klein, D. Murray,
    \textit{''Parallel tracking and mapping for small AR workspaces''},
    Proc. Sixth IEEE and ACM International Symposium on Mixed and augmented Reality,
    2007
    
\bibitem{trivedi2013}
    P. Trivedi, T. Agarwal, K. Muthunagai,
    \textit{''MC-RANSAC: A Pre-processing Model for RANSAC using Monte Carlo method implemented on a GPU''}
    2013 International Conference on Advances in Computing, Communications and Informatics (ICACCI),
    2013
    
\bibitem{kaehler2017}
    A. Kaehler, G. Bradski,
    \textit{''Learning OpenCV3: Computer vision in C++ with the OpenCV library''},
    O'Reilly Media, Inc.,
    2017
    
\bibitem{carlson2010}
    J. Carlson
    \textit{''Mapping Large, Urban Environments with GPS-Aided SLAM''},
    The Robotics Institute, Carnegie Mellon University Pittsburgh, Pennsylvania,
    2010
    
\bibitem{weinmann2016}
    M. Weinmann,
    \textit{''Reconstruction and Analysis of 3D Scenes''},
    Institute of Photogrammetry and Remote Sensing Karlsruhe Institute of Technology Karlsruhe Germany,
    2016
    
\bibitem{garcia2017}
    F. Garcia,
    \textit{''Tools for 3D Point Cloud Registration''},
    Doctoral Program in Technology, Girona,
    2017
    
\bibitem{barazzetti2018}
    L. Barazzetti,
    \textit{''Point cloud occlusion recovery with shallow feedforward neural networks''},
    Advanced Engineering Informatics,
    2018
    
\bibitem{besl1992}
    P. Besl, N. McKay,
    \textit{''A method for registration of 3-D shapes''},
    IEEE Transactions on Pattern Analysis and Machine Intelligence (PAMI), vol. 14, no. 2, pp. 239– 256, 
    1992.
    
\bibitem{chen1992}
    Y. Chen, G. Medioni,
    \textit{''Object modelling by registra- tion of multiple range images''},
    Image Vision Comput., vol. 10, no. 3, pp. 145–155,
    1992
    
\bibitem{rusinkiewicz2001}
    S. Rusinkiewicz, M. Levoy,
    \textit{''Efficient variants of the ICP algorithm''},
    Proceedings Third International Conference on 3-D Digital Imaging and Modeling,
    2001
    
\bibitem{harris1988}
    C. Harris, M. Stephens,
    \textit{''A combined corner and edge detection''},
    4th Alvey Vision Conference, pp. 147-151,
    1988
    
\bibitem{schmid2000}
    C. Schmid, R. Mohr, C. Bauckhage,
    \textit{''Evaluation of interest point detectors''},
    Int. J. Comput. Vis. 37(2), 151–172,
    2000
    
\bibitem{sipiran2011}
    I. Sipiran, B. Bustos,
    \textit{''Harris 3D: A robust extension of the Harris operator for interest point detection on 3D meshes''},
    The Visual Computer,
    2011
    
\bibitem{jollife2016}
    I. Jollife, J. Cadima,
    \textit{''Principal component analysis: A review and recent developments''},
    Philosophical Transactions of the Royal Society A: Mathematical, Physical and Engineering Sciences,
    2016
    
\bibitem{fischler1981}
    M. Fischler, R. Bolles,
    \textit{''Random sample consensus: a paradigm for model fitting with applications to image analysis and automated cartography''},
    Commun. ACM, vol. 24, no. 6, pp. 381–395,
    1981.
    
\bibitem{horn1987}
    B. Horn,
    \textit{''Closed-form solution of absolute ori- entation using unit quaternions''}
    Journal of the Optical Society of America A, vol. 4, no. 4, pp. 629–642,
    1987.
    
\bibitem{arun1987}
    K. Arun, T. Huang, S. Blostein,
    \textit{''Least- squares fitting of two 3-D point sets''},
    IEEE Transactions on Pattern Analysis and Machine Intelligence (PAMI), vol. 9, no. 5, pp. 698–700,
    1987.
    
\bibitem{eggert1997}
    D. Eggert, A. Lorusso, R. Fischer,
    \textit{''Estimating 3-D rigid body transformations: a comparison of four major algorithms''},
    Mach. Vision Appl., vol. 9, no. 5-6, pp. 272–290,
    1997.
    
\bibitem{fitzgibbon2001}
    A. Fitzgibbon,
    \textit{''Robust registration of 2D and 3D point sets''},
    British Machine Vision Conference (BMVC), pp. 411–420,
    2001.
    
\bibitem{low2004}
    K. Low,
    \textit{''Linear least-squares optimization for point- to-plane ICP surface registration''},
    Technical Report TR04-004, Department of Computer Science, University of North Carolina at Chapel Hill,
    2004.
    
\bibitem{holz2015}
    D. Holz, A. Ichim, F. Tombari, R. Rusu, S. Behnke,
    \textit{''Registration with the point cloud library: A modular framework for aligning in 3-D''},
    IEEE Robotics and Automation Magazine,
    2015.
    
\bibitem{holzer2012}
    S. Holzer, R. Rusu, M. Dixon, S. Gedikli, N. Navab,
    \textit{''Real-Time Surface Normal Estimation from Organized Point Cloud Data Using Integral Images''},
    IEEE/RSJInternational Conference on IntelligentRobots and Systems (IROS), Vila Moura, Algarve, Portugal,
    2012.

\bibitem{nguyen2012}
    C. Nguyen, S. Izadi, D. Lovell,
    \textit{''Modeling kinect sensor noise for improved 3d reconstruction and tracking''},
    3D Imaging, Modeling, Processing, Visualiza- tion andTransmission (3DIMPVT), Second International Conference on, pp. 524 –530,
    2012
\end{thebibliography}


\bibliographystyle{unsrt}
\bibliography{thebibliography}

\ifimagenes
% \section{Sensores}
Para navegar de manera robusta a través de entornos desconocidos y no estructurados, los robots deben poder percibir y modelar su entorno. Es por esto que lo que se busca es construir mapas precisos y ubicarse en los mismos utilizando solo sensores a bordo. Dentro de los mismos pueden diferenciarse dos grandes grupos
\begin{itemize}
    \item En primer lugar, los sensores necesarios para obtener información de los \textit{alrededores} del robot, como pueden ser el LIDAR y la cámara. Estos se los denominan \textit{sensores exteroceptivos}. Un sistema SLAM mínimo requiere al menos de uno de ellos para poder realizar dicha tarea.
    \item Opcionalmente, aquellos sensores que miden el movimiento propio del robot, como pueden ser los acelerómetros y encoders, los que se denominan \textit{sensores propioceptivos}.
\end{itemize}

\subsection{LiDAR}
Light Detection And Ranging, conocido como LiDAR, es una tecnología que tiene su origen en la fusión de la tecnología láser junto con la tecnología RADAR (Radio Detection And Ranging), lo cual ha permitido mejorar en gran medida la precisión de los sistemas de detección, dando lugar a nuevas aplicaciones. Los sensores LIDAR son actualmente una de las opciones más confiables para SLAM robótico tanto en ambientes interiores como exteriores. Los mismos exhibien fuentes y tipos de ruido similares que pueden modelarse libremente como Gaussianos [133].

Muchos investigadores tienen sensores LiDAR de exploración servo-montados en configuraciones de cabeceo [116, 117] o de barrido [118] (Figura \ref{fig:lidars}.a) para producir exploraciones tanto planares como 3D, aunque para este último caso el mismo entrega escaneos cada 1Hz o menos. Este campo de visión, sobre todo el 3D, tiene el costo de una mayor complejidad (sincronización servo temporal) y una menor cobertura en direcciones críticas.
\textbf{[pfingsthron2012] [newman2006] [bosse2009]}

Con el fin de aumentar la frecuencia de recolección de datos, existen sensores que integran múltiples LiDAR en una sola unidad de escaneo (Figura \ref{fig:lidars}.b), de modo que se pueden obtener escaneados en 3D completos a altas velocidades.

\begin{figure}[!b]
    \centering
    \subfloat[Scanse Sweep]{\includegraphics[width=.35\textwidth]{Img/scanse-sweep}}
    \qquad
    \subfloat[Velodyne VLP-16]{\includegraphics[width=.5\textwidth]{Img/vlp-16}}
    \caption{Algunos LIDAR comerciales}
    \label{fig:lidars}
\end{figure}

\subsubsection{Principio de funcionamiento}
El  fundamento de los dispositivos basados en la tecnología LIDAR es el cálculo del tiempo de vuelo (\textit{ToF - Time  Of Flight}) de los pulsos láser, de manera que, conociendo la velocidad del mismo, las características angulares con las que fue emitido, y la diferencia de tiempos entre el rayo emitido y el reflejado, se puede determinar de manera sencilla la distancia a la que se encuentra el obstáculo/objeto con el que el rayo impactó. Esto permite, con gran exactitud, conocer las coordenadas de la posición de objetos o superficies con respecto del sistema de coordenadas del propio dispositivo.

A medida que cada pulso láser diverge, traza un volumen que es aproximadamente cónico. Si alguna parte de este volumen cónico se cruza con un objeto, parte de la luz reflejada puede regresar a través de la lente del LIDAR. Una vez que la parte frontal del receptor del LIDAR ha recogido suficiente luz (es decir, con un umbral) registra el retorno y calcula la distancia. Esta señal de entrada introduce efectos de acortamiento y alargamiento de rango.

\subsubsection{Fuentes de ruido}
Algunas de las fuentes de ruido que afectan a la medición del LIDAR son
\begin{itemize}
    \item \textit{Ángulo de incidencia:} Si un pulso LIDAR golpea una superficie perpendicularmente, todo el frente de la onda láser se refleja al mismo tiempo. Sin embargo, a medida que aumenta el ángulo de incidencia, parte del frente de onda se refleja antes y la señal recibida activará el umbral, tarde o temprano, dependiendo del diseño del detector frontal. A medida que el ángulo de incidencia se acerca a los 90 grados, una configuración común para el lidar montado horizontalmente, los pulsos lidar viajan casi paralelos al suelo. Estos retornos de ''pastoreo en el suelo'' son extremadamente sensibles al tono del robot, las variaciones en la superficie y otros ruidos. En esta configuración, incluso un robot sin movimiento puede producir mediciones lidar con medidores de ruido de rango. Estos retornos de pastoreo pueden presentar un desafío para SLAM con un lidar montado horizontalmente.
    \item \textit{Efectos de contorno:} están estrechamente relacionados con los retornos de pastoreo, los retornos espurios pueden ocurrir en el límite de los objetos, donde el frente de onda elíptica puede intersecar parte de uno o más objetos a medida que viaja. La medición del rango resultante a menudo se promedia y se crea una medición espuria "colgando" en el espacio vacío entre los objetos. La figura 2.4 demuestra este efecto.
    \item \textit{Propiedades de la superficie del objeto:} causa que la cantidad de luz láser reflejada varíe mucho. Un objeto altamente especular, como un espejo o agua quieta, reflejará la mayor parte de la luz láser y, a menudo, devolverá las mediciones a objetos más distantes (en el rumbo incorrecto). Cuanto más difusamente un objeto refleje la luz, mejor se puede medir en un rango más amplio de ángulos y distancias. Los objetos que absorben la luz infrarroja (que generalmente se ve negra para los humanos) a menudo no pueden reflejar suficiente luz, lo que limita los rangos de medición. Además, en algunos sensores lidar, los objetos blancos pueden aparecer un poco más cerca que los objetos negros, ya que el umbral del receptor se activa un poco antes [133].
    \item \textit{Luz ambiental:} La mayoría de los sensores comerciales LIDAR utilizan filtros de muesca infrarrojos para aumentar la relación señal a ruido. Si bien esto les permite funcionar al aire libre, la luz ambiental intensa, como la luz solar directa, disminuirá su alcance. Cuando la luz reflejada de un pulso lidar no es lo suficientemente fuerte como para activar una medición, un modelo de sensor típico supone que hay espacio libre hasta una fracción del rango máximo del sensor. Esta suposición de espacio libre debe variar según los niveles de luz ambiental, sin embargo, sin sensores adicionales, no es posible determinar cuándo la medición de lidar faltante se debe al espacio libre real o a la luz ambiental excesiva.
    \item \textit{Sincronización temporal:} cuando se monta en un robot en movimiento, el origen de un lidar se moverá a medida que el sensor giratorio complete cada escaneo. Moviéndose a $1m/s$, por ejemplo, un lidar de 40 Hz producirá hasta 25 mm de sesgo si no se compensa. La compensación se puede realizar utilizando un modelo de movimiento continuo, como [134], sin embargo, esto requiere una sincronización rigurosa y una marca de tiempo de los datos del sensor. La figura 2.4 muestra una nube de puntos tridimensional coloreada generada a partir de un lidar montado en un servomotor de movimiento rápido, con una cámara de obturación global y sincronización de microsegundos.
\end{itemize}

\subsubsection{Calibración}
MIRAR

\subsection{Cámara}
Las cámaras en su versión más simple tienen la característica de ser más económicas y sencillas de montar respecto al resto de los sensores utilizados para el SLAM (tal como el LiDAR), además de no emitir señales al entorno para obtener las características del mismo. A la hora de realizar el SLAM. el modelo de sensado de las cámaras consiste básicamente en un mapeo entre el entorno tridimencional y el plano de la imagen bidimensional. Dependiendo del tipo de cámara (monocular, estéreo, omnidireccional, RGB-D, entre otras), es posible utilizar diferentes modelos matemáticos que permitan relacionar puntos del mundo con su respectiva representación en la imagen.

Para las cámaras monoculares, la línea de base entre las imágenes debe estimarse a partir de la odometría, lo que da lugar al problema de SLAM monocular bien estudiado [124, 125]. Otra forma de realizarlo es mediante la percepción de alto nivel, donde se pueden utilizar señales como el tamaño relativo de un objeto para estimar la escala de la imagen y, por lo tanto, la profundidad [126].

En cambio, para las cámaras RGB-D (\textit{D: Depth} - Profundidad) (Figura \ref{fig:camaras}.b), uno de los métodos presentados en la literatura para la realización del SLAM a partir de las mismas [ENDERS2012] consiste en primero extraer características visuales de las imágenes de color entrantes. Luego, se comparan estas características con las características de imágenes anteriores. Al evaluar las imágenes de profundidad en las ubicaciones de estos puntos de características, se obtienen un conjunto de correspondencias 3D puntuales entre dos cuadros cualquiera. Basándose en estas correspondencias, se estima la transformación relativa entre los marcos utilizando RANSAC [TRIVEDI2013].

Por otro lado, las técnicas de visión estéreo utilizan dos cámaras con una separación entre las mismas fija y conocida (Figura \ref{fig:camaras}.a). Las correspondencias de características de la imagen se identifican entre las imágenes mediante la geometría epipolar[REF DE CASTRO] [111], y los rangos se calculan cuando se han medido las disparidades de imagen. Los enfoques pasivos sufren desde muchos modos de falla, desde agujeros de profundidad en áreas de imagen sin características visuales, hasta escenas que crean ambigüedades [123].

\begin{figure}
    \centering
    \includegraphics[width=.9\textwidth]{Img/bumblekinect}
    \caption{Cámaras: (a) Bumblebee 2, (b) Kinect}
    \label{fig:camaras}
\end{figure}

\subsection{Sensores Inerciales}
El término sensor inercial se usa para denotar la combinación de un acelerómetro de tres ejes y un giroscopio de tres ejes. Los dispositivos que contienen estos sensores se denominan comúnmente unidades de medición inercial (IMU, por \textit{intertial measurement unit}), los cuales en muchos casos incluyen también un magnetómetro de tres ejes. Estas unidades sensoriales suelen ser usadas para determinar la orientación y posición de objetos a los cuales se encuentran integrados, permitiendo así la incorporación de los mismos en gran número de aplicaciones [7, 59, 109, 156] [DE KOK2017].

Hoy en día, muchos de estos sensores se basan en la tecnología de sistemas microelectromecánicos (MEMS). Los componentes de MEMS son pequeños, ligeros, económicos, tienen un bajo consumo de energía y tiempos de arranque cortos. Su precisión ha aumentado significativamente con los años.

Un giroscopio mide la velocidad angular del sensor, es decir, la velocidad de cambio de la orientación del sensor. En cambio, un acelerómetro mide la fuerza específica externa que actúa sobre el sensor. La fuerza específica consiste tanto en la aceleración del sensor como en la gravedad de la Tierra. Por otro lado, los magnetómetros son los encargados de medir el campo magnético en el que el objeto se encuentra sumergido, obteniendo así información absoluta del ambiente y no referida únicamente al objeto en si. Todos estos sensores tienen el inconveniente de presentar un \textit{bias} variable en el tiempo que afecta su funcionamiento, errores de \textit{scaling} utilizados para convertir las salidas digitales de los sensores en cantidades físicas reales, desalineaciones entre ellos mismos en caso de que se trate de un circuito integrado incluyendo a dos o más de ellos, entre otros.

\subsubsection{Calibración}
En una IMU ideal, los grupos triaxiales deben compartir los mismos ejes de sensibilidad ortogonal 3D que abarcan un espacio tridimensional, mientras que el factor de escala debe convertir la cantidad digital medida por cada sensor en la cantidad física real. Desafortunadamente, las IMU basadas en MEMS de bajo costo generalmente se ven afectadas por el escalado no preciso, las desalineaciones del eje del sensor, las sensibilidades de los ejes cruzados y los sesgos distintos de cero. La calibración de la IMU se refiere al proceso de identificación de estas cantidades.

Sin embargo, las IMUs comerciales de costo elevado presentan también estos problemas (aunque en menor escala), con la diferencia de que el fabricante brinda los datos de la calibración, el cual es único para cada una de ellas, minimizando así los términos de incertidumbre. El factor de calibración de estas IMUs se suele realizar con métodos estándar, comparando los datos arrojados por la IMU con referencias conocidas, haciendo que el proceso sea lento y costoso debido al equipamiento necesario. A continuación se presentarán propuestas para resolver la calibración de los diferentes sensores

\paragraph{Acelerómetro-Giróscopo}
Para evitar el uso de equipamiento específico, en \textbf{[TEDALDI2014]} se propone utilizar solo el movimiento propio de la IMU y sus estados intermedios estáticos para calibrar tanto el giróscopo como el acelerómetro. Este procedimiento explota la idea básica del método de múltiples posiciones, presentado primero en [7] para la calibración de acelerómetros: en una posición estática, las normas de las aceleraciones medidas son iguales a las magnitudes de la gravedad más un multi-factor de error de origen (es decir, incluye \textit{bias}, desalineación, ruido, entre otros). Todas estas cantidades pueden estimarse a través de la minimización sobre un conjunto de actitudes estáticas. 

Después de la calibración de la tríada del acelerómetro, es posible utilizar las posiciones del vector de gravedad medidas por los acelerómetros como referencia para calibrar la tríada del giroscopio. Es por esto que la exactitud de la calibración del giroscopo depende fuertemente de la exactitud de la calibración del acelerómetro. Al integrar las velocidades angulares entre dos posiciones estáticas consecutivas, se consigue estimar las posiciones de gravedad en la nueva orientación. La calibración de los giroscopios finalmente se obtiene minimizando los errores entre estas estimaciones y las referencias de gravedad dadas por los acelerómetros calibrados.

Para una IMU ideal, los 3 ejes de la tríada de acelerómetros y los 3 ejes de la tríada de giroscopios definen un marco tridimensional único, compartido y ortogonal. En la realidad, en base a lo explicado anteriormente, los marcos de los acelerómetros y de los giróscopos no suelen ser ortogonales. A su vez, si se tiene a ambos dentro de un mismo chip, puede definirse un marco del cuerpo (o body frame), el cual es un marco ortogonal que representa, por ejemplo, el chasis del integrado. Este marco suele ser distinto a los otros dos marcos, pero como esta diferencia está dada por ángulos muy pequeños, la medición $\bm{m}$ tanto del acelerómetro como del giróscopo en un marco no ortogonal (o sea, el marco del sensor) puede llevarse al marco del cuerpo mediante
\begin{equation}
    \bm{m}^b = \bm{C}\bm{m}^s
\end{equation}
siendo $\bm{C}$ la matriz de rotación de la Expresión (\ref{eq:rpyinfinitesimal}).

Si se asume que el marco del cuerpo coincide con el marco ortogonal del acelerómetro, entonces para el caso de la aceleración
\begin{equation}
    \bm{a}^b = \bm{C}_a\bm{a}^s
\end{equation}
siendo
\begin{equation}
    \bm{C}_a =
    \begin{bmatrix}
        1 & \alpha_{12} & \alpha_{13} \\
        0 & 1 & \alpha_{23} \\
        0 & 0 & 1
    \end{bmatrix}
\end{equation}
Como se mencionó anteriormente, tanto las mediciones del acelerómetro como las del giróscopo deberían referir al mismo marco de referencia, en este caso, el marco del cuerpo. Por lo tanto,
\begin{equation}
    \bm{\omega}^b = \bm{C}_g\bm{\omega}^s
\end{equation}
siendo
\begin{equation}
    \bm{C}_g =
    \begin{bmatrix}
        1 & \gamma_{12} & \gamma_{13} \\
        \gamma_{21} & 1 & \gamma_{23} \\
        \gamma_{31} & \gamma_{32} & 1
    \end{bmatrix}
\end{equation}

Debido al error de conversión, ambos sensores se encuentran afectados por errores de \textit{scaling}
\begin{align}
    \bm{K}_a &=
    \begin{bmatrix}
        sa_x & 0 & 0 \\
        0 & sa_y & 0 \\
        0 & 0 & sa_z
    \end{bmatrix}
    \\
    \bm{K}_g &=
    \begin{bmatrix}
        sg_x & 0 & 0 \\
        0 & sg_y & 0 \\
        0 & 0 & sg_z
    \end{bmatrix}
\end{align}
 y de \textit{bias}
 \begin{align}
    \bm{b}_a &=
    \begin{bmatrix}
        ba_x \\
        ba_y \\
        ba_z
    \end{bmatrix}
    \\
     \bm{b}_a &=
     \begin{bmatrix}
        bg_x \\
        bg_y \\
        bg_z
     \end{bmatrix}
\end{align}

En consecuente, los modelos completos de los sensores resultan
\begin{align}
    \bm{a}^b &= \bm{C}_a\bm{K}_a\left(\bm{a}^s + \bm{b}_a + \bm{v}_a\right) \\
    \bm{\omega}^b &= \bm{C}_g\bm{K}_g\left(\bm{\omega}^s + \bm{b}_g + \bm{v}_g\right)
\end{align}
con $\bm{v}_i$ correspondiendo a los ruidos de medición de cada uno.

En concreto, para cada uno de los sensores se tendrán parámetros a estimar. Para el caso del acelerómetro
\begin{equation}
    \bm{\epsilon}_{a} = [\alpha_{12},\alpha_{13},\alpha_{23},sa_x,sa_y,sa_z,ba_x,ba_y,ba_z]
\end{equation}

Al aplicarse un promediado a cada intervalo de la calibración del acelerómetro, esto permite olvidar el ruido de la medición, y definir entonces
\begin{equation}
    \bm{a}^b = h(\bm{a}^s,\bm{\epsilon}_{a}) = \bm{T}_a\bm{K}_a(\bm{a}^s+\bm{b}_a)    
\end{equation}

Como en [7], se mueve a la IMU un set de $M$ rotaciones independientes y temporalmente estables, obteniendo entonces $M$ aceleraciones $\bm{a}^s_k$, las cuales son promediadas en cada uno de estos intervalos estáticos. La función de coste para minimizar los parámetros este caso es
\begin{equation}
    \mathscr{S}(\bm{\epsilon}_{a}) = \sum_{k=1}^M(||\bm{g}||^2-||h(\bm{a}^s_k,\bm{\epsilon}_{a})||^2)^2
\end{equation}
donde $||g||^2$ es la magnitud actual del vector de gravedad local, obtenido de tablas específicas. Al tratarse de una regresión no lineal, la misma puede minimizarse utilizando el algoritmo de \textit{Levenberg-Marquardt}.

Para calibrar la triada del giróscopo, puede asumirse al mismo libre de \textit{bias} ya que este sesgo puede obtenerse promediando sobre una cantidad de datos consecutivos de tamaño conveniente en el instante inicial estacionario. Por ello, el vector de parámetro desconocidos resulta
\begin{equation}
    \bm{\epsilon}_{g} = \left[\gamma_{12},\ \gamma_{13},\ \gamma_{21},\ \gamma_{23},\ \gamma_{31},\ \gamma_{32},\ sg_x,\ sg_y,\ sg_z\right]
\end{equation}

Una vez definido el \textit{bias}, como para la calibración del giróscopo se toma al acelerómetro como referencia conocida, se requiere convertir los datos de dicho giróscopo en un vector de aceleraciones. Para ello, se define el operador $\psi$, 
\begin{equation}
    \bm{u}_{g,k} = \psi\left[\bm{\omega}_i^s,\bm{u}_{a,k-1}\right]
\end{equation}
que toma como entrada una secuencia de $n$ lecturas del giróscopo $\bm{\omega}_i^s$ y el versor de gravedad inicial $\bm{u}_{a,k-1}$ obtenido del acelerómetro ya calibrado, y retorna el versor de gravidad final $\bm{u}_{g,k}$. 

Una vez obtenido $\bm{u}_{g,k}$, para el caso del giróscopo, la función de coste será entonces
\begin{equation}
    \mathscr{S}(\bm{\epsilon}_{g}) = \sum_{k=2}^M ||\bm{u}_{a,k} - \bm{u}_{g,k}||^2
\end{equation}
siendo $M$ el número de los intervalos estáticos, $\bm{u}_{a,k}$ es el versor de la aceleración medido promediando en la ventana temporal obtenida en el $k$-ésimo intervalo estático, y $\bm{u}_{g,k}$ corresponde al versor de aceleración en base a los datos del giróscopo computada anteriormente. Se obtiene, entonces, $\bm{\epsilon}_{g}$ minimizando la Expresión anterior utilizando, por ejemplo, el algoritmo de Levenberg-Marquardt.

\subparagraph{Ecuación diferencial ordinaria de la velocidad angular}
Si se tiene un vector $\bm{r}$ de longitud constante, la velocidad angular del mismo en base a las leyes físicas puede obtenerse realizando la derivada del mismo
\begin{equation}
    \frac{d\bm{r}}{dt} = \bm{\omega}\times \bm{r}
\end{equation}
siendo $\times$ el producto externo. Como para el caso evaluado la velocidad angular es perpendicular al vector $\bm{r}$ (producto intero $\bm{\omega}\cdot\bm{r} = 0$), tal como puede verse en la Figura \ref{fig:angularvelocity}. Por lo tanto, puede escribirse a la misma en la forma de cuaterniones
\begin{equation}
    \frac{d\bm{r}}{dt} = \bm{\omega} \otimes \bm{r}
\end{equation}
teniendo en cuenta que tanto $\bm{\omega}$ como $\bm{r}$ se obtienen mediante la Expresión (\ref{eq:vectorquaternionform}).

\textbf{REVISAR SI ESTA BIEN EL GRAFICO, LA IDEA !!!}
\begin{figure}[!ht]
    \centering
    \includegraphics[width=0.5\textwidth]{Img/AngularVelocity.png}
    \caption{Velocidad angular respecto al vector}
    \label{fig:angularvelocity}
\end{figure}

Ahora, siendo que $\bm{r}$ puede definirse mediante la Expresión (\ref{eq:quaternionvectorrotation}), y teniendo en cuenta las propiedades de los cuaterniones
\begin{align}
    \bm{r} &= \bm{q}\otimes \bm{r}_0 \otimes \bm{q}^{-1} \\
    \frac{d\bm{r}}{dt} &= \frac{d}{dt}\left[\bm{q}\otimes\bm{r}_0\otimes\bm{q}^{-1}\right] \\
    &= \dot{\bm{q}}\otimes\bm{r}_0\otimes\bm{q}^{-1} + \bm{q}\otimes\bm{r}_0\otimes\dot{\bm{q}^{-1}} = \bm{\omega}\otimes\bm{r}
\end{align}

Como $\dot{\bm{q}^{-1}}$ puede generar problemas, es posible obtenerlo realizando la derivada del conjugado del cuaternión con su conjugado
\begin{align}
    \frac{d}{dt}\left(\bm{q}\otimes\bm{q}^{-1}\right) &= \frac{d}{dt}1 \\
    \dot{\bm{q}}\otimes\bm{q}^{-1} + \bm{q}\otimes\dot{\bm{q}^{-1}} &= 0
\end{align}
entonces
\begin{equation}
    \dot{\bm{q}^{-1}} = -\bm{q}^{-1}\otimes\dot{\bm{q}}\otimes\bm{q}^{-1}
\end{equation}
sustituyendo y poniéndolo en función de $\bm{r}$
\begin{equation}
        \frac{d}{dt}\left[\bm{q}\otimes\bm{r}_0\otimes\bm{q}^{-1}\right] = \dot{\bm{q}}\otimes\bm{q}^{-1}\otimes\bm{r} - \bm{r}\otimes\dot{\bm{q}}\otimes\bm{q}^{-1}
\end{equation}
donde dicha Expresión tiene la forma de la operación conmutador $\left[\bm{p},\bm{q}\right]$. Finalmente, puede llegarse a la ecuación diferencial
\begin{align}
    \dot{\bm{q}}\otimes\bm{q}^{-1}\otimes\bm{r} - \bm{r}\otimes\dot{\bm{q}}\otimes\bm{q}^{-1} &= \bm{\omega}\otimes\bm{r} \\
    \left[\dot{\bm{q}}\otimes\bm{q}^{-1},\bm{r}\right] &= \bm{\omega}\otimes\bm{r} \\
    2\dot{\bm{q}}\otimes\bm{q}^{-1}\otimes\bm{r} &= \bm{\omega}\otimes\bm{r} \\
    \dot{\bm{q}} &= \frac{1}{2}\bm{\omega}\otimes\bm{q}
\end{align}
donde $\bm{\omega}(t)$ es la velocidad angular en el marco fijo global. En muchos casos es útil expresar la Expresión () en base a la velocidad angular en el marco del sensor. La velocidad angular en este marco es la velocidad angular global rotada en el marco del cuerpo, dado por la Expresión (\ref{eq:quaternionvectorrotation}). En consecuencia,
\begin{equation}
    \dot{\bm{q}} = \frac{1}{2}\bm{q}\otimes\bm{\omega}^s\otimes\bm{q}^{-1}\otimes\bm{q}
\end{equation}
y recordando la Propiedad (\ref{eq:quaternioninverse})
\begin{equation}
    \dot{\bm{q}} = \frac{1}{2}\bm{q}\otimes\bm{\omega}^s 
\end{equation}
Finalmente, en base a la Expresión (\ref{eq:quaternionproductmatrixright}), se llega a
\begin{equation}
    \dot{\bm{q}} = \frac{1}{2}\bm{\Omega}(\bm{\omega}(t))\bm{q}
    \label{eq:edoquaternion}
\end{equation}
siendo $\bm{\Omega}(\bm{\omega}(t))$ la matriz de rotación en base a la velocidad angular obtenida del giróscopo
\begin{equation}
    \bm{\Omega}(\bm{\omega}(t)) =
    \begin{bmatrix}
        0 & -\bm{\omega} \\
        \bm{\omega}^T & -\left[\bm{\omega}\right]_\times
    \end{bmatrix}
    =
    \begin{bmatrix}
        0 & -\omega_x & -\omega_y & -\omega_z \\
        \omega_x & 0 & \omega_z & -\omega_y \\
        \omega_y & -\omega_z & 0 & \omega_x \\
        \omega_z & \omega_y & -\omega_x & 0
    \end{bmatrix}
\end{equation}

\paragraph{Magnetómetro}
Para la calibración del magnetómetro, en cambio, los enfoques tradicionales suponen que hay un sensor de referencia disponible que puede proporcionar información de rumbo precisa. Un ejemplo bien conocido de esto es el balanceo de la brújula [3]. Sin embargo, para permitir que cualquier usuario realice la calibración, se han desarrollado una gran cantidad de enfoques que eliminan la necesidad de una fuente de información de orientación. Una clase de estos algoritmos de calibración de magnetómetro se enfoca en minimizar la diferencia entre la magnitud del campo magnético medido y la del campo magnético local[4]. Este enfoque también se conoce como verificación escalar[5]. Otra clase formula el problema de calibración como un problema de ajuste de elipsoide, es decir, como el problema de mapear un elipsoide de datos a una esfera[6] - [8]. El beneficio de usar esta formulación es que existe una vasta literatura sobre cómo resolver problemas de ajuste de elipsoides,[9][10]. Fuera de estas dos clases, también está disponible un gran número de otros enfoques de calibración[11], donde se consideran diferentes formulaciones del problema de calibración en términos de un problema de máxima verosimilitud. La ventaja de estos métodos es que utilizan sólo la información provista por el magnetómetro.

Sin embargo, si lo que se busca es calibrar un magnetómetro para mejorar la estimación del rumbo en combinación con sensores inerciales, la alineación de los ejes sensoriales de los sensores de inercia y el magnetómetro es crucial en este caso. Se puede ver que esta alineación determina la orientación de la esfera azul de los datos del magnetómetro calibrado en la Fig. 1. Los algoritmos que solo usan datos del magnetómetro pueden asignar el elipsoide rojo de los datos a una esfera, pero sin información adicional, la rotación de esta esfera permanece desconocido.

Varios enfoques más modernos incluyen un segundo paso en el algoritmo de calibración para determinar la desalineación[6], [12] - [14] entre diferentes ejes del sensor. Una opción común para alinear los ejes del sensor de inercia y el magnetómetro es utilizar mediciones de acelerómetro a partir de períodos de aceleraciones bastante pequeñas [12], [13]. La desventaja de este enfoque es que se debe determinar un umbral para usar las mediciones del acelerómetro. Además, se omiten los datos del giroscopio. En [15], por otro lado, el problema se reformula en términos del cambio de orientación, lo que permite el uso directo de los datos del giroscopio. \textbf{[DE KOK 2014]}


\subsection{Encoders}
El objetivo de los encoders es medir las rotaciones relativas de la rueda a la que están asociados. Los encoders normalmente se fijan a la salida del motor o de la caja de cambios, con opciones de diseño que a menudo cambian la resolución angular con la máxima velocidad. Los efectos de resolución y muestreo a menudo producen mediciones ruidosas que requieren filtrado digital. La odometría basada en ruedas supone que no hay deslizamiento entre las ruedas y el terreno. En la práctica, los robots móviles con frecuencia superan la fricción estática y rodante, y cuando una o más ruedas giran o se deslizan, pueden ocurrir grandes errores de odometría. La pendiente del terreno, la fricción y otras propiedades pueden variar, incluso entre ruedas individuales. Los robots de accionamiento diferencial con cuatro ruedas a menudo experimentan grandes errores de odometría cuando giran, ya que se requiere un deslizamiento de las ruedas para girar [150].

\subsection{GPS}
El GPS es un sistema de navegación georreferenciado basado en satélites mediante los cuales estima la latitud, longitud y altitud del objeto en cuestión. Debido a esto, el mismo tiene su campo de aplicación principalmente en ambientes al aire libre, donde mediante la triangulación entre 4 o más satélites puede determinar la ubicación del objeto en cuestión. Debido a esto y a que, entre otros factores, su principio de funcionamiento se basa en medir el tiempo que tardó la señal disparada por un satélite en ser recibida por el propio sensor, el mismo cuenta con precisiones variables dependiendo de la posición de los satélites [162]. Estos sesgos y errores de grandes pasos hacen que la navegación robótica con GPS sea problemática, especialmente con grandes equipos de robots que operan en entornos urbanos y durante muchas horas. [CARLSON2010]
\else
    \newpage
    \section{Apéndices}
\label{sec:apendix}

\subsection{Circuitos electricos}
\label{sec:9_circ}
\includepdf[pages=-]{PDF/HW_Dron_V2_1.pdf}

\subsection{Hojas de Datos}
\label{sec:10_hd}

\subsection{Programas Fuente}
\label{sec:11_soft}

\subsection{Layout PCB}
\label{sec:12_pcblay}

\fi

\end{document}