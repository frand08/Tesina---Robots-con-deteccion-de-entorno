\section{Introducción}
\label{sec:2_marcoteorico}
Si bien el uso de robots tiene sus orígenes principalmente en fábricas para el ensamblaje de automóviles, en los últimos años la electrónica moderna ha permitido introducirlos en otras áreas, ya sea desde electrodomésticos y juguetes de la vida cotidiana, hasta viajes espaciales con el fin de explorar planetas desconocidos. Esta expansión es debido a que los mismos permiten reducir la interacción humana no solo en tareas que presentan un riesgo a la integridad de la persona, sino también en aquellas que tienen cierto grado de repetitividad.

Yendo a un caso más concreto, en el último tiempo muchas personas han adquirido los llamados robots aspiradoras, los cuales consiguen limpiar la superficie de las casas en un tiempo medianamente razonable, aunque este tiempo no suele ser una preocupación ya que al ser el mismo completamente autónomo, uno puede seguir con sus actividades cotidianas. Ahora bien, si se analiza el recorrido que realizan la mayoría de estos robots, se puede apreciar que el mismo es completamente aleatorio, y por ende se tendería a creer que no podrá pasar por toda la superficie y limpiarla completamente. La realidad es que, como están mucho tiempo circulando, logran pasar por todos lados, pero esto genera que circulen más veces por unos lugares que por otros, haciendo ineficiente el trabajo, tal como se observa en la Figura \ref{fig:vaccumrobot}.

\begin{figure}%
    \centering
    \subfloat{{\includegraphics[width=0.47\textwidth]{Img/Roomba}}}%
    \qquad
    \subfloat{{\includegraphics[width=0.47\textwidth]{Img/VaccumRobot.png}}}%
    \caption{Ejemplo del recorrido de un robot aspiradora comercial}
    \label{fig:vaccumrobot}
\end{figure}

A pesar de que se busque reducir la interacción humana bajo estas circunstancias, no todos los robots hoy en día son autónomos, principalmente debido a la falta de robustez del algoritmo de control involucrado en el proceso. Tomando por caso el ejemplo anterior, si bien el robot aspiradora es autónomo, su eficiencia respecto a la forma óptima de realizar la tarea no es una garantía en todos los casos.

Es útil distinguir entre robots que están inmóviles, como un brazo robótico en una fábrica, y robots que son móviles, como un auto sin conductor. En este trabajo se hará hincapié en los robots móviles. Usamos este término para describir un robot impulsado por sus propios medios que puede moverse cinemáticamente entre ubicaciones en su entorno. Cuando se habla de la posición y orientación combinadas del robot, esto se define como su \textit{pose}.

Los robots móviles pueden referirse a robots que se mueven sobre el suelo, bajo el agua, a través del aire y en entornos de microgravedad, tales como los que pueden observarse en la Figura \ref{fig:mobilerobots}. Si bien este trabajo busca aplicar en parte a cualquiera de estos entornos, el enfoque del mismo se refiere principalmente a los robots móviles que permanecen en contacto con el suelo. El término vehículos terrestres no tripulados (con sus siglas en inglés, UGV) a menudo se usa más específicamente para describir robots móviles terrestres, mientras que el término vehículos aéreos no tripulados (del inglés, UAV) refieren principalmente a los drones.

Los robots móviles se mueven a través de entornos grandes y potencialmente dinámicos, lo que hace que la percepción sea mucho más difícil que los robots industriales con entornos de trabajo limitados y parámetros operativos rígidos. Los robots móviles requieren sensores adicionales, una mejor percepción y mayores grados de autonomía para operar en entornos del mundo real que cambian con frecuencia.


\begin{figure}%
    \centering
    \subfloat[Wheeled robot]{{\includegraphics[width=0.25\textwidth]{Img/WheeledRobot.jpeg}}}%
    \qquad
    \subfloat[Legged robot]{{\includegraphics[width=0.25\textwidth]{Img/LeggedRobot.jpeg}}}%
    \qquad
    \subfloat[Tracked robot]{{\includegraphics[width=0.25\textwidth]{Img/TrackedRobot.jpeg}}}%
    \qquad
    \subfloat[Drone]{{\includegraphics[width=0.3\textwidth]{Img/Drone.jpeg}}}%
    \qquad
    \subfloat[Water-based robot]{{\includegraphics[width=0.5\textwidth]{Img/WaterBasedRobot.jpg}}}%
    \caption{Tipos de robot móviles}
    \label{fig:mobilerobots}
\end{figure}

Dentro de lo que respecta a robótica móvil, uno de sus desafíos actuales es lograr la completa autonomía del vehículo. Una gran parte de las empresas de la industria automotriz han dedicado mucho tiempo y dinero para lograr el objetivo. La investigación ha avanzado hasta el punto de que en la actualidad hay autos semi-autónomos disponibles en el mercado, y se cree ampliamente que en el futuro cercano casi todos los vehículos serán completamente autónomos. Para lograr esto, obviamente hay una necesidad de sensores y funciones avanzadas. Uno de los principales problemas es el hecho de que el vehículo debe ser consciente de su posición actual dentro de un entorno desconocido. Esto se conoce como la localización y mapeo simultáneos, comúnmente abreviado como SLAM.

El SLAM es un problema difícil de solucionar, así como lo fue para los humanos en el pasado. Tiempo atrás, los marineros usaban los llamados registros de fichas para estimar su velocidad y extrapolaban esta velocidad con el tiempo para calcular su posición u orientación. Este proceso de navegación por estima (en inglés, \textit{dead reckoning}) conduce inevitablemente a errores graves de posicionamiento con el tiempo y, por lo tanto, siempre se ha respaldado mediante el uso de puntos de referencia (o \textit{landmarks}) para la navegación. Quedándonos en este caso particular, con el uso de las estrellas los marineros lograban saber donde estaba el Norte, permitiendo corregir su dirección. Sin embargo, no siempre se podía observar este patrón por días debido al cielo nublado, entonces tenían que confiar solamente en su navegación por estima, haciendo que la incertidumbre crezca día tras día. Para el caso de los robots, cuando el mismo recorre un  mapa completamente desconocido, acumulará error debido a las predicciones que debe tomar respecto a su pose y entorno en el que está, hasta el momento en que entra a un área con puntos de referencia conocidos, puediendo corregir su estimación de posición. Esto es lo denominado cierre de ciclo (o \textit{loop closure}), como puede verse en la Figura \ref{fig:poseloopclcorr}.

\begin{figure}
    \centering
    \includegraphics[width=\textwidth]{Img/Pose_LoopClosureCorr.png}
    \caption{Estimación de pose y su correción mediante el loop closure}
    \label{fig:poseloopclcorr}
\end{figure}

Volviendo al ejemplo del robot de limpieza, si el mismo conociera el mapa en el que se encuentra, podría entonces trazar la ruta mas óptima para realizar la limpieza del lugar. Como no todos los ambientes son iguales y cada uno tiene, por ejemplo, muebles ubicados en distintos lugares, dicho robot podría hacer primero un reconocimiento del lugar previo a la limpieza para tener una noción de todo el espacio, o bien obtener el mapa y la ubicación en la que se encuentra en base al movimiento que realiza, pasando por los lugares que le faltaron aspirar. Para cualquiera de las dos opciones, por lo tanto, necesitará tanto de tomar los datos del lugar como de controlar al vehículo para que vaya circulando por el ambiente.

La estimación de la pose, entonces, es una parte no separable de aplicaciones como el control del vehículo y el mapeo. Varios sensores se utilizan comúnmente para estimar la pose de un robot. Las unidades de medición inercial (IMU), la cámara, la odometría de las ruedas\footnote{Se mide la rotación de las ruedas para estimar cambios de la posición del vehículo a lo largo del tiempo} para el caso de ciertos robots móviles  y el LiDAR se encuentran entre los sensores más populares para la localización en interiores \cite{delrosario2016}, mientras que para exteriores suele sumarse el GPS a estos sensores, corrigiendo al mismo mediante acción de la IMU \cite{caron2006}. Entre estos sensores, el LiDAR ha recibido menos atención para la estimación de la pose. Esto se debe a que los LiDAR bidimensionales no pueden ofrecer una estimación completa de la pose\footnote{De lo que refiere a la orientación y posicion en tres dimensiones} y los LiDAR tridimensionales son voluminosos y costosos.

En la localización al aire libre, la señal del sistema de posicionamiento global (GPS) suele estar disponible y, dado que la posición se puede obtener con precisión mediante GPS, es la opción común para la fusión con los datos obtenidos de la IMU \cite{engel2014}. Sin embargo, en un entorno denegado por GPS, como dentro de un edificio, se deben usar otros sensores para corregir las estimaciones de IMU. Un enfoque común para abordar este problema es la fusión IMU-cámara \cite{mirzaei2008}, \cite{hesch2009}, \cite{chambers2014}, \cite{hesch2013}, y en ciertos casos combinadas con LiDAR \cite{lee2016}.

Un algoritmo SLAM robusto es esencial para que cualquier robot móvil navegue de manera segura a través de un entorno no estructurado. El algoritmo SLAM con frecuencia define un marco de coordenadas global para que el robot opere; uno que generalmente es utilizado por todas las funcionalidades de alto nivel, como navegación, planificación de rutas, exploración, identificación de objetos, seguimiento de objetos y manipulación de los mismos. Estas dependencias hacen que el algoritmo SLAM sea una parte central de cualquier arquitectura de robot móvil, y las fallas irrecuperables son altamente indeseables. Es probable que cualquier robot desorientado por una falla de SLAM no pueda realizar su tarea, o peor, puede poner en peligro a los humanos, a sí mismo o al medio ambiente.

Si bien se han demostrado varios algoritmos SLAM en laboratorios, es más difícil producir soluciones robustas en entornos no estructurados del mundo real. El deslizamiento de las ruedas, por ejemplo, puede producir mediciones de odometría ruidosas y sesgadas, mientras que los sensores del sistema de posicionamiento global (GPS) a menudo producen una localización global muy ruidosa y sesgada. El SLAM robusto del mundo real se vuelve aún más difícil cuando estos errores de detección aleatorios y sistemáticos ocurren en entornos que tienen geometrías redundantes u objetos en movimiento.