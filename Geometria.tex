\section{Geometría tridimensional}
En la presente sección se presentará la geometría tridimensional y específicamente los conceptos de rotación, traslación y algunas de sus representaciones. Presta especial atención al establecimiento de marcos de referencia. {\big Sastry (1999)} es una referencia integral sobre el control de la robótica que incluye un trasfondo sobre geometría tridimensional. {\big Hughes (1986)} también proporciona una buena base de primeros principios.

Los robots móviles generalmente son libres de trasladar y rotar. Matemáticamente, tienen seis grados de libertad: tres en traslación y tres en rotación. Esta configuración geométrica de seis grados de libertad se conoce como la \textit{pose} (posición y orientación) del robot. Algunos robots pueden tener múltiples cuerpos conectados entre sí; en este caso cada cuerpo tiene su propia pose. Consideraremos solo el caso de un solo cuerpo aquí.

\subsection{Marco de referencia}

