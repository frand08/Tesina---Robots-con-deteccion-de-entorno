
\section{Simultaneous Localization And Mapping}
Esta sección describe los conocimientos básicos respecto a la Localización y Mapeo Simultáneos (SLAM), a la vez de introducir parte de los algoritmos y sensores conocidos.

\subsection{Introducción al SLAM}
El problema de localización y mapeo simultáneos (SLAM) se basa en el proceso de un robot que construye un mapa de su entorno desconocido (\textit{mapping}) mientras este lo explora conociendo su ubicación en el mismo (\textit{localization}). Dicho problema puede ser expresado como el dilema del huevo o la gallina, ya que para conocer la ubicación del robot es necesario determinar el mapa que lo rodea, sin embargo, para que el mismo pueda estimar el mapa en el cual se encuentra, necesita primero conocer su ubicación dentro de ese entorno. A partir de la detección y seguimiento de marcas naturales del ambiente (\textit{landmarks}), los sistemas SLAM estiman tanto la posición del robot como la ubicación de las marcas en el entorno. El mapa se construye con las estimaciones de las posiciones de dichas marcas, las cuales van siendo ajustadas a medida que son observadas desde distintas posiciones.

El SLAM es un problema difícil ya que, debido al ruido de los sensores utilizados, ninguna de las mediciones es perfecta. Esto significa que ni el movimiento del robot ni la estructura del entorno se conocen de manera absolutamente precisa, sino solo hasta cierto grado de incertidumbre. Con el fin de hacer frente a estas incertidumbres, el SLAM generalmente se entiende y se aborda mediante técnicas y modelos probabilísticos. Las diferentes formas en que se representan las funciones de densidad probabilística constituyen las diferencias en cada enfoque. Muchas de ellas utilizan filtros de Bayes, tal como lo es el Filtro de Kalman Extendido (EKF) 
\begin{large}
[DURRANT-WHYTE2006]
[KALMAN1960]
\end{large}, o en versiones más modernas, por ejemplo, el Fast-SLAM
\begin{large}
[PONER FASTSLAM BIEN]
\end{large}. 
Sin embargo, el uso de enfoques de estimación de estado para describir el problema de SLAM implica resolver una serie de inconvenientes que se generan a partir del mismo, siendo de los más destacados la \textit{asociación de datos}\textbf{[ref 103 y 104 de reid2016]} y el \textit{cierre de ciclo}.

\subsubsection{Asociación de datos}
En su forma más simple, la \textit{asociación de datos} refiere a relacionar diferentes mediciones con los objetos que se encuentran en el entorno. Si bien los sensores utilizado para este tipo de aplicaciones puede tomar miles de mediciones del entorno por segundo, generalmente no produce información suficiente para resolver estas concordancias directamente.

Con una incertidumbre siempre presente en la posición del robot, normalmente no se conoce la dirección exacta a la que apunta un sensor. Esta incertidumbre, junto con la incapacidad para identificar de qué objeto proviene cada medición, crea incertidumbres y posibles ambigüedades referidas a la distancia que se encuentra el objeto respecto al robot.

\subsubsection{Cierre de ciclo}
Cuando un robot sigue un camino largo, mientras mide su progreso con sensores incorporados, el mismo acumula incertidumbre debido al ruido inherente de los sensores. Si el robot sigue visitando nuevos lugares, la incertidumbre de la posición seguirá creciendo sin límite. Cuando el robot vuelve a visitar un lugar en el que ha estado anteriormente, se producen eventos de \textit{cierre de ciclo}
\begin{large}
[REFS DE REID Y VER CASTRO]
\end{large}, otorgando información valiosa respecto a la relación entre las estimaciones llevadas a cabo por los sistemas de localización. Tomar conocimiento de que se ha efectuado un ciclo en la trayectoria permite calcular el error cometido en la estimación de la posición y da origen a una serie de procesos que permiten corregir tanto la localización actual del robot como el mapa hasta ese momento construido.

\begin{large}
VER DE QUE MAS PONERRRRRRR
\end{large}{}

\subsection{Algoritmos de SLAM}
Los mapas SLAM se construyen a partir de millones de lecturas de sensores, que se comparan entre sí en un paso de asociación de datos que depende completamente de las estimaciones de \textit{pose} actuales. La evaluación y la reevaluación de estas asociaciones de datos, al mismo tiempo que se calcula y actualiza todo el historial de \textit{poses} del robot, describe el problema de \textit{full SLAM} \textbf{[ref 6 de reid2016]}. Si bien este problema no puede resolverse para entornos no triviales, los enfoques descritos en la literatura a menudo producen resultados útiles en entornos del mundo real. En concreto, con el fin de reducir la complejidad del algoritmo se toman como supuestos comunes que el \textit{entorno es estático}, es decir, que el mapa no varía con el tiempo, y por segundo que las taryectorias de los robots \textit{pueden predecirse}, logrando así que los modelos de movimiento puedan predecir dónde es probable que esté un robot, permitiendo que la búsqueda de asociaciones de datos comience cerca del óptimo global.

En esta sección se describen algunos de los enfoques principales que se encuentran en la literatura respecto al tema en cuestión.

\subsubsection{GraphSLAM}

\subsubsection{EKF SLAM}
El filtro de Kalman es uno de los filtros basados en la estimación Bayesiana más conocidos, el cual asume que los modelos matemáticos utilizados son lineales y las distribuciones de sus componentes Gaussianas. En SLAM, sin embargo, los modelos utilizados suelen ser no lineales, por lo que se requiere realizr una linealización apropiada. Mediante el uso de una serie de Taylor alrededor el estado estimado actual, el \textit{Filtro de Kalman Extendido} (EKF) linealiza dicho modelo no lineal.

El enfoque del EKF para solucionar el problema de SLAM es mediante el uso de datos de sensores recopilados a partir del movimiento y la rotación del robot. Esto se puede hacer, por ejemplo, utilizando encoders y una IMU. Además, también es necesario recopilar información del entorno mediante, por ejemplo, el uso de un LIDAR. Con estos datos, el algoritmo realiza un seguimiento del lugar donde probablemente se ubica el robot dentro de un mapa, así como un seguimiento de los puntos de referencia específicos observados.

Al igual que con el filtro de Bayes general, el EKF itera un ciclo de \textit{prediction-update}:

\paragraph{Prediction} Predecir el estado actual expresado por la media predicha y la covarianza predicha. Estos se calculan en función de la entrada de control, las matrices, la covarianza anterior y la media anterior.

\paragraph{Update} La idea detrás del paso de corrección es la asociación de datos el cual, como se dijo anteriormente, tiene el objetivo principal de distinguir entre un hito observado anteriormente y uno recientemente observado. Cuando se produce una nueva medición, en primer lugar se almacenan las ubicaciones de los puntos de referencia recientemente observados. Después de eso, se asocian puntos de referencia observados previamente con los observados recientemente. Además, se calcula la ganancia de corrección, también conocida como la ganancia de Kalman. Básicamente, es un factor de ganancia de corrección necesario para actualizar la media y la covarianza actuales del estado actual. Sin embargo, la media actual y la covarianza no solo dependen de la ganancia de Kalman, sino que también toman en cuenta la media predicha y la covarianza predicha y la medición.

Además de la complejidad cuadrática, este filtro cuando se aplica al problema SLAM tiene como desventaja sustancial \big{[Csorba 1997 DE SU - FAST-SLAM]}
la sensibilidad a fallas que presenta en las asociaciones de datos. Este problema con el EKF se aplica en situaciones en las que se desconocen las asociaciones de datos. El EKF mantiene una única hipótesis de asociación de datos por observación, típicamente elegida usando una heurística de máxima verosimilitud. Si la asociación de datos elegida por esta heurística es incorrecta, el efecto de incorporar esta observación en el EKF nunca se puede eliminar.

\subsubsection{FastSLAM}
